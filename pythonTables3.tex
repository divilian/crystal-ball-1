
\chapter[Tables in Python (3 of 3)]{\huge\selectfont{Tables in Python (3 of
3)}}

\section{Queries}

\label{queries}
\index{query}
\index{match (a query)}

Back in section~\ref{seriesQueries} (p.~\pageref{seriesQueries}), we learned
how to write simple \textbf{queries} to selectively match only certain elements
of a \texttt{Series}. The same technique is available to us with
\texttt{DataFrame}s, only it's more powerful since there are more columns to
work with at a time.

\index{simpsons@The Simpsons}

Let's return to the Simpsons example from p.~\pageref{finalSimpsons}, which is
reproduced here:

\begin{Verbatim}[fontsize=\footnotesize,samepage=true,frame=leftline,framesep=5mm,framerule=1mm]
       species  age gender            fave     IQ          hair   salary
name                                                                    
Homer    human   36      M            beer   74.0          none  52000.0
Marge    human   34      F  helping others  120.0  stacked tall      0.0
Bart     human   10      M      skateboard   90.0          buzz      0.0
Lisa     human    8      F       saxophone  200.0         curly      0.0
Maggie   human    1      F        pacifier  100.0         curly      0.0
SLH        dog    4      M             NaN  100.0        shaggy      0.0
\end{Verbatim}

\index{boxies (square brackets)}
\index{[]@\texttt{[]} (boxies)}

We can filter this on certain rows by including a query in boxies:

\index{adults@\texttt{adults}}

\begin{Verbatim}[fontsize=\normalsize,samepage=true,frame=single,framesep=3mm]
adults = simpsons[simpsons.age > 18]
\end{Verbatim}
\vspace{-.2in}

\begin{Verbatim}[fontsize=\small,samepage=true,frame=leftline,framesep=5mm,framerule=1mm]
      species  age gender            fave     IQ          hair   salary
name                                                                   
Homer   human   36      M            beer   74.0          none  52000.0
Marge   human   34      F  helping others  120.0  stacked tall      0.0
\end{Verbatim}

As with \texttt{Series}es, we can't forget to repeat the name of the variable
(``\texttt{simpsons}'') before giving the query criteria (``\texttt{> 18}'').
Unlike with \texttt{Series}es, we also specify a column name
(``\texttt{.age}'') that we want to query.

\index{compound condition}
\index{and (compound condition)}
\index{kids@\texttt{kids}}

We can also provide compound conditions in just the same way as before
(section~\ref{seriesCompoundQueries}, p.~\pageref{seriesCompoundQueries}). If
we want only human children, we say:

\begin{Verbatim}[fontsize=\small,samepage=true,frame=single,framesep=3mm]
kids = simpsons[(simpsons.age <= 18) & (simpsons.species == "human")]
\end{Verbatim}
\vspace{-.2in}

\begin{Verbatim}[fontsize=\small,samepage=true,frame=leftline,framesep=5mm,framerule=1mm]
       species  age gender        fave     IQ   hair  salary
name                                                        
Bart     human   10      M  skateboard   90.0   buzz     0.0
Lisa     human    8      F   saxophone  200.0  curly     0.0
Maggie   human    1      F    pacifier  100.0  curly     0.0
\end{Verbatim}

\index{or (compound condition)}
\index{kids@\texttt{old\_andor\_wise}}
whereas if we want everybody who's smart and/or old, we say:

\begin{Verbatim}[fontsize=\small,samepage=true,frame=single,framesep=3mm]
old_andor_wise = simpsons[(simpsons.IQ > 100) | (simpsons.age > 30)]
\end{Verbatim}
\vspace{-.2in}

\begin{Verbatim}[fontsize=\small,samepage=true,frame=leftline,framesep=5mm,framerule=1mm]
      species  age gender            fave     IQ          hair   salary
name                                                                   
Homer   human   36      M            beer   74.0          none  52000.0
Marge   human   34      F  helping others  120.0  stacked tall      0.0
Lisa    human    8      F       saxophone  200.0         curly      0.0
\end{Verbatim}

To narrow it down to only specific columns, we can combine our query with the
syntax from section~\ref{accessIndividualRowsColsOfDataFrame}
(p.~\pageref{accessIndividualRowsColsOfDataFrame}). You see, our query gave us
another (shorter) \texttt{DataFrame} as a result, which has the same rights and
privileges as any other \texttt{DataFrame}. So tacking on another pair of
boxies gives us just a column:

\begin{Verbatim}[fontsize=\small,samepage=true,frame=single,framesep=3mm]
print(simpsons[simpsons.age > 18]['fave'])
\end{Verbatim}
\vspace{-.2in}

\begin{Verbatim}[fontsize=\small,samepage=true,frame=leftline,framesep=5mm,framerule=1mm]
name
Homer              beer
Marge    helping others
Name: fave, dtype: object
\end{Verbatim}

\index{double boxies@``double boxies''}
while tacking on \textit{double} boxies gives us column\textbf{s}:

\begin{Verbatim}[fontsize=\small,samepage=true,frame=single,framesep=3mm]
print(simpsons[simpsons.age > 18][['fave','gender','IQ']])
\end{Verbatim}
\vspace{-.2in}

\begin{Verbatim}[fontsize=\small,samepage=true,frame=leftline,framesep=5mm,framerule=1mm]
                 fave gender     IQ
name                               
Homer            beer      M   74.0
Marge  helping others      F  120.0
\end{Verbatim}

Note that in the first of these cases, we got a \textit{\texttt{Series}} back,
whereas in the second (with the double boxies) we got a multi-column
\texttt{DataFrame}.

Combining all these operations takes practice, but lets you slice and dice a
\texttt{DataFrame} up in innumerable different ways.

\section{The \texttt{.groupby()} method}

\index{groupby@\texttt{.groupby()} method (Pandas)}
\index{subset (of a data set)}

One of the most useful methods in the whole \texttt{DataFrame} repertoire is
\texttt{.groupby()}. It applies when you want summary statistics (mean,
quantile, max/min, \textit{etc.}) not for the \textit{whole} data set, but for
each \textit{subset} of the data set, where the subsets split on the values of 
one of the variables.

Here's an example in action. It's old news that we could find, say, the median
IQ of the Simpsons family overall:

\begin{Verbatim}[fontsize=\small,samepage=true,frame=single,framesep=3mm]
print(simpsons['IQ'].median())
\end{Verbatim}
\vspace{-.2in}

\begin{Verbatim}[fontsize=\small,samepage=true,frame=leftline,framesep=5mm,framerule=1mm]
95.0
\end{Verbatim}

But it's new news that we can do this for each gender separately, via:

\begin{Verbatim}[fontsize=\normalsize,samepage=true,frame=single,framesep=3mm]
print(simpsons.groupby('gender')['IQ'].median())
\end{Verbatim}
\vspace{-.2in}

\begin{Verbatim}[fontsize=\small,samepage=true,frame=leftline,framesep=5mm,framerule=1mm]
gender
F    120.0
M     74.0
Name: IQ, dtype: float64
\end{Verbatim}

\index{categorical variable}
We give a \textit{categorical} variable as the argument to \texttt{.groupby()},
and specify a \textit{numeric} variable as the column we wish to analyze.
Finally, we choose the summary statistic we want (\texttt{.median()} in the
above case). 

All this produces a resulting \textit{\texttt{Series}}. Think hard: the
\textit{keys} of the resulting \texttt{Series} are the \textit{values} of the
categorical variable (in the original \texttt{Series}) that we grouped by; and
the \textit{values} of the resulting \texttt{Series}, are the results of
applying the summary statistic function to each of the subsets separately.

So now, in addition to knowing that the overall Simpson family median IQ is 95,
we also know that among Simpson boys and men, it's only 74, whereas among girls
and women, it's an impressive 120.

Another example: let's find the maximum age for each hair style:

\begin{Verbatim}[fontsize=\small,samepage=true,frame=single,framesep=3mm]
print(simpsons.groupby('hair')['age'].max())
\end{Verbatim}
\vspace{-.2in}

\begin{Verbatim}[fontsize=\small,samepage=true,frame=leftline,framesep=5mm,framerule=1mm]
hair
buzz            10
curly            8
none            36
shaggy           4
stacked tall    34
Name: age, dtype: int64
\end{Verbatim}

(Since there's so many different hairstyles present, Maggie turns out to be the
only one whose age is not represented here.)

% TODO: explain .groupby() as "split/apply/combine" with figure

\section{Looping with \texttt{DataFrame}s}

\index{loop}
\index{itertuples@\texttt{.itertuples()} (Pandas)}

Just as we wrote \texttt{for} loops to iterate through the elements of a NumPy
array (section~\ref{arrayLoops}) and a Pandas \texttt{Series}
(section~\ref{seriesLoops}), so we can iterate through the rows of a
\texttt{DataFrame}. We'll do so using the weirdly-named \texttt{.itertuples()}
method.

\index{loop!header}
\index{loop!body}
By using \texttt{.itertuples()} in the loop header, we have available to us in
the loop body a \texttt{Series} representing \textit{the current row}.
(``Current row'' just means ``the row we're on as we successively go through
all the rows in sequence.'') We can access individual elements of it by using
column names and the dot (or boxie) syntax, as follows:

\begin{Verbatim}[fontsize=\small,samepage=true,frame=single,framesep=3mm]
for row in simpsons.itertuples():
    print("A certain {}-year old has {} hair.".format(row.age, row.hair))
\end{Verbatim}
\vspace{-.2in}

\begin{Verbatim}[fontsize=\small,samepage=true,frame=leftline,framesep=5mm,framerule=1mm]
A certain 36-year old has none hair.
A certain 34-year old has stacked tall hair.
A certain 10-year old has buzz hair.
A certain 8-year old has curly hair.
A certain 1-year old has curly hair.
A certain 4-year old has shaggy hair.
\end{Verbatim}

I called the loop variable ``\texttt{row}'' because it represents a row of the
\texttt{DataFrame} (duh), although you can call it anything you want to. (I
could have named it ``\texttt{family\_member}'' or ``\texttt{Simpson}''
instead.)

This is very intuitive. Slightly less intuitive is that if we want the index
column (in \texttt{simpsons}, it's called ``\texttt{name},'' remember) we can't
use the name of the index column. Instead, we have to literally say
``\texttt{.Index},'' and yes that's a \textit{capital} `\texttt{I}'.

To illustrate:

\begin{Verbatim}[fontsize=\small,samepage=true,frame=single,framesep=3mm]
for family_member in simpsons.itertuples():
    print("{} Simpson, {} years of age, has {} hair.".format(
        family_member.Index, family_member.age, family_member.hair))
\end{Verbatim}
\vspace{-.2in}

\begin{Verbatim}[fontsize=\small,samepage=true,frame=leftline,framesep=5mm,framerule=1mm]
Homer Simpson, 36 years of age, has none hair.
Marge Simpson, 34 years of age, has stacked tall hair.
Bart Simpson, 10 years of age, has buzz hair.
Lisa Simpson, 8 years of age, has curly hair.
Maggie Simpson, 1 years of age, has curly hair.
SLH Simpson, 4 years of age, has shaggy hair.
\end{Verbatim}

