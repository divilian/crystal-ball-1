
\chapter[Associative arrays in Python (3 of 3)]{\huge\selectfont{Associative
arrays in Python (3 of 3)}}
\label{ch:assocArraysInPython3}

\index{min@\texttt{.min()} (Pandas)}
\index{max@\texttt{.max()} (Pandas)}
\index{idxmin@\texttt{.idxmin()} (Pandas)}
\index{idxmax@\texttt{.idxmin()} (Pandas)}
\index{index@index (pl:~indices)}
\index{key-value pair}

But wait, there's more! We can also use methods like \texttt{.min()},
\texttt{.max()}, \texttt{.idxmin()}, and \texttt{.idxmax()} to get the
``extremes'' of a \texttt{Series} -- \textit{i.e.} the lowest and highest
values in a \texttt{Series}, or their keys (indexes). Note that
\texttt{.idxmin()} does \textit{not} give you the lowest key in the
\texttt{Series}! Instead, it gives you \textit{the key of the lowest value}.
Study this code snippet and its output to test your understanding of this:

\begin{Verbatim}[fontsize=\small,samepage=true,frame=single,framesep=3mm]
understanding = pd.Series([15,4,13,3,7], index=[4,10,2,12,9])
print(understanding)
print("The min is {}.".format(understanding.min()))
print("The max is {}.".format(understanding.max()))
print("The idxmin is {}.".format(understanding.idxmin()))
print("The idxmax is {}.".format(understanding.idxmax()))
\end{Verbatim}

\begin{Verbatim}[fontsize=\small,samepage=true,frame=leftline,framesep=5mm,framerule=1mm]
4     15
10     4
2     13
12     3
9      7
dtype: int64

The min is 3.
The max is 15.
The idxmin is 12.
The idxmax is 4.
\end{Verbatim}

The \texttt{idxmin} and \texttt{idxmax} are \texttt{12} and \texttt{4},
respectively, since the smallest value in the series (the \texttt{3}) has a key
of \texttt{12}, and the largest value (the \texttt{15}) has a key of
\texttt{4}.

\index{index@\texttt{.index} syntax (Pandas)}

If we did actually want the lowest (or highest) key, we could use the
\texttt{.index} syntax (see p.~\pageref{dotIndex}) to achieve that:

\begin{Verbatim}[fontsize=\small,samepage=true,frame=single,framesep=3mm]
print("The lowest key is {}.".format(understanding.index.min()))
print("The highest key is {}.".format(understanding.index.max()))
\end{Verbatim}

\begin{Verbatim}[fontsize=\small,samepage=true,frame=leftline,framesep=5mm,framerule=1mm]
The lowest key is 2.
The highest key is 12.
\end{Verbatim}

And remember that ``lowest''/``highest'' for string data means alphabetical
order.

