
\chapter[\small Assoc.~arrays in Python (3 of 3)]{\LARGE\selectfont{Associative
arrays in Python (3 of 3)}}
\label{ch:assocArraysInPython3}

\index{min@\texttt{.min()} (Pandas)}
\index{max@\texttt{.max()} (Pandas)}
\index{idxmin@\texttt{.idxmin()} (Pandas)}
\index{idxmax@\texttt{.idxmax()} (Pandas)}
\index{index@index (pl:~indices)}
\index{key-value pair}

But wait, there's more! We can also use methods like \texttt{.min()},
\texttt{.max()}, \texttt{.idxmin()}, and \texttt{.idxmax()} to get the
``extremes'' of a \texttt{Series} -- \textit{i.e.} the lowest and highest
values in a \texttt{Series}, or their keys (indexes). Note that
\texttt{.idxmin()} does \textit{not} give you the lowest key in the
\texttt{Series}! Instead, it gives you \textit{the key of the lowest value}.
Study this code snippet and its output to test your understanding of this:

\index{understanding@\texttt{understanding}}

\begin{Verbatim}[fontsize=\footnotesize,samepage=true,frame=single,framesep=3mm]
understanding = pd.Series([15,4,13,3,7], index=[4,10,2,12,9])
print(understanding)
print("The min is {}.".format(understanding.min()))
print("The max is {}.".format(understanding.max()))
print("The idxmin is {}.".format(understanding.idxmin()))
print("The idxmax is {}.".format(understanding.idxmax()))
\end{Verbatim}

\begin{Verbatim}[fontsize=\footnotesize,samepage=true,frame=leftline,framesep=5mm,framerule=1mm]
4     15
10     4
2     13
12     3
9      7
dtype: int64

The min is 3.
The max is 15.
The idxmin is 12.
The idxmax is 4.
\end{Verbatim}

The \texttt{idxmin} and \texttt{idxmax} are \texttt{12} and \texttt{4},
respectively, since the smallest value in the series (the \texttt{3}) has a key
of \texttt{12}, and the largest value (the \texttt{15}) has a key of
\texttt{4}.

\index{index@\texttt{.index} (little \texttt{i}) syntax}

If we did actually want the lowest (or highest) key, we could use the
\texttt{.index} syntax (see p.~\pageref{dotIndex}) to achieve that:

\begin{Verbatim}[fontsize=\footnotesize,samepage=true,frame=single,framesep=3mm]
print("The lowest key: {}.".format(understanding.index.min()))
print("The highest key: {}.".format(understanding.index.max()))
\end{Verbatim}

\begin{Verbatim}[fontsize=\footnotesize,samepage=true,frame=leftline,framesep=5mm,framerule=1mm]
The lowest key: 2.
The highest key: 12.
\end{Verbatim}

And remember that ``lowest''/``highest'' for string data means alphabetical
order.

\section{Queries}

\label{seriesQueries}
\index{query}
\index{match (a query)}
\index{filter}

One of the most powerful things we'll do with a data set is to \textbf{query}
it. This means that instead of specifying (say) a particular key, or something
like ``the minimum'' or ``the maximum,'' we provide our own custom criteria and
ask Pandas to give us all values that \textbf{match}. This kind of operation is
also sometimes called \textbf{filtering}, because we're taking a long list of
items and sifting out only the ones we want.

\index{boxies (square brackets)}
\index{[]@\texttt{[]} (boxies)}

\index{condition (of a query)}

The syntax is interesting: you still use the boxies (like you do when
giving a specific key) but inside the boxies you put a \textbf{condition} that
will be used to select elements. It's best seen with an example. Re-using the
\texttt{understanding} variable from above, we can query it and ask for all the
elements greater than 5:

\begin{Verbatim}[fontsize=\small,samepage=true,frame=single,framesep=3mm]
more_than_five = understanding[understanding > 5]
print(more_than_five)
\end{Verbatim}

\begin{Verbatim}[fontsize=\small,samepage=true,frame=leftline,framesep=5mm,framerule=1mm]
4    15
2    13
9     7
dtype: int64
\end{Verbatim}

The new thing here is the ``\texttt{understanding > 5}'' thing inside the
boxies. The result of this query is itself a \texttt{Series}, but one in which
everything that doesn't match the condition is filtered out. Thus we only have
three elements instead of five. Notice the keys didn't change, and they also
had nothing to do with the query: our query was about \textit{values}.

\index{index@\texttt{.index} (little \texttt{i}) syntax}

We could change this, if we were interested in putting a restriction on
\textit{keys} instead, using the \texttt{.index} syntax:

\begin{Verbatim}[fontsize=\footnotesize,samepage=true,frame=single,framesep=3mm]
index_more_than_five = understanding[understanding.index > 5]
print(index_more_than_five)
\end{Verbatim}

\begin{Verbatim}[fontsize=\small,samepage=true,frame=leftline,framesep=5mm,framerule=1mm]
10    4
12    3
9     7
dtype: int64
\end{Verbatim}

See how tacking on ``\texttt{.index}'' in the query made all the difference.

\subsection{Query operators}

\index{wakkas (angle brackets)}
\index{<>@\texttt{<>} (wakkas)}

Now I have a surprise for you. It makes perfect sense to use the character
``\texttt{>}'' (called ``greater-than,'' ''right-angle-bracket,'' or simply
''wakka'') to mean ``greater than.'' And the character ``\texttt{<}'' makes
sense as ``less than.'' Unfortunately, the others don't make quite as much
sense. See the top table in Figure~\ref{fig:queryOps}.

\index{bang (``\texttt{"!}'')}
\index{"!@\texttt{"!} (bang)}
\index{bang-equals (``\texttt{"!=}'')}

``Greater/less than or equal to'' isn't hard to remember, and it's a good thing
Python doesn't require symbols like ``$\le$'' or ``$\ge$'' since those are hard
to find on your keyboard. You just type both symbols back-to-back, with no
space. More problematic are the last two entries in the top table. The
``\texttt{!=}'' operator (pronounced ``bang-equals'') is used as a stand-in
for ``$\ne$'' which also isn't keyboard friendly. And that one doesn't have a
good mnemonic; you just have to memorize it.

\index{double-equals (\texttt{==})}
\index{==@\texttt{==} (double-equals)}

By far the most error-prone of this set is the ``\texttt{==}''
(\textbf{double-equals}) operator, which simply means ``equals.'' \textbf{Yes,
you do have to use double-equals instead of single-equals in your queries, and
yes it matters.} As additional incentive, let me inform you that if you use
single-equals when you needed to use double-equals, \textit{it will seem to
work at first}, but you will silently get the wrong answer.

\textbf{Memorize this fact!} Failing to use double-equals is quite possibly the
single most common programming error for beginners.

\begin{figure}[ht]
\centering
\small
\bigskip
\underline{\textit{Simple query operators:}}\\
\smallskip
\begin{tabular}{c|c}
Symbol & Meaning \\
\hline
\texttt{>} & greater than \\
\hline
\texttt{<} & less than \\
\hline
\texttt{>=} & greater than or equal to \\
\hline
\texttt{<=} & less than or equal to \\
\hline
\texttt{!=} & \textit{not} equal to \\
\hline
\texttt{==} & equal to \\
\end{tabular}

\bigskip
\bigskip
\underline{\textit{Compound query operators:}}\\
\smallskip
\begin{tabular}{c|c}
Symbol & Meaning \\
\hline
\texttt{\&} & and \\
\hline
\texttt{|} & or \\
\hline
\texttt{\textasciitilde} & not \\
\end{tabular}

\bigskip
\normalsize
\caption{Query operators: simple and compound}
\label{fig:queryOps}
\end{figure}


Here are some more examples to test your understanding. Make sure you
understand why each output is what it is.

\begin{Verbatim}[fontsize=\footnotesize,samepage=true,frame=single,framesep=3mm]
understanding = pd.Series([15,4,13,3,7], index=[4,10,2,12,9])
print(understanding[understanding <= 7])
\end{Verbatim}
\vspace{-.3in}

\begin{Verbatim}[fontsize=\small,samepage=true,frame=leftline,framesep=5mm,framerule=1mm]
10    4
12    3
9     7
dtype: int64
\end{Verbatim}

\begin{Verbatim}[fontsize=\small,samepage=true,frame=single,framesep=3mm]
print(understanding[understanding != 13])
\end{Verbatim}
\vspace{-.3in}

\begin{Verbatim}[fontsize=\small,samepage=true,frame=leftline,framesep=5mm,framerule=1mm]
4     15
10     4
12     3
9      7
dtype: int64
\end{Verbatim}

\begin{Verbatim}[fontsize=\small,samepage=true,frame=single,framesep=3mm]
print(understanding[understanding == 3])
\end{Verbatim}
\vspace{-.3in}

\begin{Verbatim}[fontsize=\small,samepage=true,frame=leftline,framesep=5mm,framerule=1mm]
12    3
dtype: int64
\end{Verbatim}

\begin{Verbatim}[fontsize=\small,samepage=true,frame=single,framesep=3mm]
print(understanding[understanding.index >= 9])
\end{Verbatim}
\vspace{-.3in}

\begin{Verbatim}[fontsize=\small,samepage=true,frame=leftline,framesep=5mm,framerule=1mm]
10    4
12    3
9     7
dtype: int64
dtype: int64
\end{Verbatim}


\subsection{Compound queries}

\label{seriesCompoundQueries}
\index{compound condition}

Often, your query will involve more than one criterion. This is called a
\textbf{compound condition}. It's not as common with \texttt{Series}es as it
will be with \texttt{DataFrame}s in a couple chapters, but there are still uses
for it here.

\index{and (compound condition)}

Suppose I want all the key/value pairs of \texttt{understanding} where the
value is \textit{between 5 and 14}. This is really two conditions masquerading
as one: we want all pairs where (1) the value is greater than 5, \textbf{and}
also (2) the value is less than 14. I put the word ``\textbf{and}'' in boldface
in the previous sentence because that's the operator called for here. We only
want elements in our results where \textit{both} things are true, and
therefore, we ``\textit{and} together the two conditions.'' (``And'' is being
used as a verb here.)

The way to achieve this is as follows. The syntax is nutty, so pay close
attention:

\begin{Verbatim}[fontsize=\footnotesize,samepage=true,frame=single,framesep=3mm]
x = understanding[(understanding > 5) & (understanding < 14)]
print(x)
\end{Verbatim}

\begin{Verbatim}[fontsize=\small,samepage=true,frame=leftline,framesep=5mm,framerule=1mm]
2    13
9     7
dtype: int64
\end{Verbatim}

First, notice that we put each of our two conditions \textit{in bananas} here.
This is \textit{not} optional, as it turns out: you'll get a non-obvious error
message if you omit them. Second, see how we combined the two with the
``\texttt{\&}'' operator from the bottom half of Figure~\ref{fig:queryOps}. The
result, then, was only the elements that satisfied \textit{both} conditions.

\index{or (compound condition)}
\index{pipe (vertical bar, or ``\texttt{|}'')}
\index{|@\texttt{|} (pipe)}

It can be tricky to figure out whether you want an \textbf{and} or an
\textbf{or}. Unfortunately they don't always correspond to their colloquial
English usage. Let's see what happens if we switch the ``\texttt{\&}'' symbol to
a ``\texttt{|}'' (pronounced ``pipe''):

\begin{Verbatim}[fontsize=\footnotesize,samepage=true,frame=single,framesep=3mm]
y = understanding[(understanding > 5) | (understanding < 14)]
print(y)
\end{Verbatim}

\begin{Verbatim}[fontsize=\small,samepage=true,frame=leftline,framesep=5mm,framerule=1mm]
4     15
10     4
2     13
12     3
9      7
dtype: int64
\end{Verbatim}

You can see that we got everything back. That's because \textbf{or} means
``only give me the elements where \textit{either one} of the conditions,
\textit{or both}, are true.'' In this case, this is guaranteed to match
everything, because if you think about it, \textit{every} number is either
greater than five, or less than fourteen, or both. (Think deeply.)

Even though in this example it didn't do anything exciting, an ``\textbf{or}''
does sometimes return a useful result. Consider this example:

\begin{Verbatim}[fontsize=\scriptsize,samepage=true,frame=single,framesep=3mm]
z = understanding[(understanding.index > 10) | (understanding > 5)]
print(z)
\end{Verbatim}

\begin{Verbatim}[fontsize=\small,samepage=true,frame=leftline,framesep=5mm,framerule=1mm]
4     15
2     13
12     3
dtype: int64
\end{Verbatim}

Here we're asking for all key/value pairs in which \textit{either} the key is
greater than ten, \textit{or} the value is greater than ten, or both. This
reeled in exactly three fish as shown above. If we changed this ``\texttt{|}''
to an ``\texttt{\&}'', we'd have caught \textit{no} fish. (Take a moment to
convince yourself of that.)

\index{not (query condition)}
\index{squiggle (tilde, or ``\texttt{\textasciitilde}'')}
\index{~@\texttt{\textasciitilde} (squiggle)}

The last entry in Figure~\ref{fig:queryOps} is the ``\texttt{\textasciitilde}''
sign, which is pronounced ``tilde,'' ``twiddle,'' or ``squiggle.'' It
corresponds to the English word \textbf{not}, although in an unusual place in
the sentence. Here's an example:

\begin{Verbatim}[fontsize=\scriptsize,samepage=true,frame=single,framesep=3mm]
a = understanding[~(understanding.index > 10) | (understanding > 10)]
print(a)
\end{Verbatim}

\begin{Verbatim}[fontsize=\small,samepage=true,frame=leftline,framesep=5mm,framerule=1mm]
4     15
10     4
2     13
9      7
dtype: int64
\end{Verbatim}

Search for and stare at the squiggle in that line of code. In English, what we
said was ``give me elements where either the key is \textit{not} greater than
ten, or the value is greater than ten, or both.'' The four matching elements
are shown above. 

Changing the ``or'' back to an ``and'' here gives us this output instead:

\begin{Verbatim}[fontsize=\scriptsize,samepage=true,frame=single,framesep=3mm]
b = understanding[~(understanding.index > 10) & (understanding > 10)]
print(b)
\end{Verbatim}

\begin{Verbatim}[fontsize=\small,samepage=true,frame=leftline,framesep=5mm,framerule=1mm]
4     15
2     13
dtype: int64
\end{Verbatim}

These are the only two rows where \textit{both} conditions are true (and
remember that the first one is ``not-ted.'')

It can be tricky to get compound queries right. As with most things, it just
takes some practice.

\subsection{Queries on strings}

So far our examples have involved only numbers. Pandas also lets us perform
queries on text data, specifying constraints on such things as the length of
strings, letters in certain positions, and case (upper/lower).

\index{Marvel comics}

Let's return to the Marvel-themed series from section~\ref{marvelSeries}:

\begin{Verbatim}[fontsize=\footnotesize,samepage=true,frame=single,framesep=3mm]
alter_egos = pd.Series(['Hulk','Spidey','Iron Man','Thor'],
    index=['Bruce','Peter','Tony','Thor'])
\end{Verbatim}

\index{str@\texttt{.str} suffix (for Pandas queries)}
\index{len@\texttt{len()}}

By appending ``\texttt{.str}'' to the end of the variable name, we can get
access to most of the string-based methods we'd like to use. For instance, find
all the values with exactly four letters:

\begin{Verbatim}[fontsize=\small,samepage=true,frame=single,framesep=3mm]
four_letter_names = alter_egos[alter_egos.str.len() == 4]
print(four_letter_names)
\end{Verbatim}
\vspace{-.3in}

\begin{Verbatim}[fontsize=\small,samepage=true,frame=leftline,framesep=5mm,framerule=1mm]
Bruce    Hulk
Thor     Thor
dtype: object
\end{Verbatim}

\index{contains@\texttt{contains()}}
or all the values that contain a space:

\begin{Verbatim}[fontsize=\small,samepage=true,frame=single,framesep=3mm]
spaced_out = alter_egos[alter_egos.str.contains(' ')]
print(spaced_out)
\end{Verbatim}
\vspace{-.3in}

\begin{Verbatim}[fontsize=\small,samepage=true,frame=leftline,framesep=5mm,framerule=1mm]
Tony    Iron Man
dtype: object
\end{Verbatim}

or all the keys whose first character is a \texttt{T}:

\begin{Verbatim}[fontsize=\footnotesize,samepage=true,frame=single,framesep=3mm]
to_a_tee = alter_egos[alter_egos.index.str.startswith('T')]
print(to_a_tee)
\end{Verbatim}
\vspace{-.3in}

\begin{Verbatim}[fontsize=\small,samepage=true,frame=leftline,framesep=5mm,framerule=1mm]
Tony    Iron Man
Thor        Thor
dtype: object
\end{Verbatim}

or all entries where either the value is greater than five letters long or the
key is the same as the value:

\begin{Verbatim}[fontsize=\small,samepage=true,frame=single,framesep=3mm]
huh = alter_egos[(alter_egos.str.len() > 5) |
    (alter_egos.index == alter_egos)]
print(huh)
\end{Verbatim}
\vspace{-.3in}

\begin{Verbatim}[fontsize=\small,samepage=true,frame=leftline,framesep=5mm,framerule=1mm]
Peter    Spidey
Thor       Thor
dtype: object
\end{Verbatim}

The possibilities are endless. Some of the more common functions are summarized
in Figure~\ref{fig:stringQueryFunctions}.

\setlength\extrarowheight{5pt}

\begin{figure}[ht]
\centering
\footnotesize
\begin{tabular}{c|p{2.8in}}
Function & Description \\
\hline

\textsl{ser}\texttt{.str.len()} & Set a condition on the length of a string. \\

\textsl{ser}\texttt{.str.startswith(}\textsl{str}\texttt{)} & Request only
strings that begin with certain letter(s). \\

\textsl{ser}\texttt{.str.endswith(}\textsl{str}\texttt{)} & Request only
strings that end with certain letter(s). \\

\textsl{ser}\texttt{.str.contains(}\textsl{str}\texttt{)} & Request only
strings that contain certain letter(s) somewhere in them. \\

\textsl{ser}\texttt{.str.isupper()} & Request only strings that are in all
upper-case. \\

\textsl{ser}\texttt{.str.islower()} & Request only strings that are in all
lower-case. \\

\end{tabular}
\bigskip
\caption{Common query methods for string data.}
\label{fig:stringQueryFunctions}
\end{figure}

\subsection{Last word}

A couple things before we move on. You've noticed that in all the above
examples, it was necessary to type the \texttt{Series} variable name several
times:

\begin{Verbatim}[fontsize=\scriptsize,samepage=true,frame=single,framesep=3mm]
understanding[(understanding < 12) | (understanding > 18)]
alter_egos[(alter_egos.str.isupper()) & (alter_egos.str.len() < 10)]
\end{Verbatim}

There's really no way around that, sorry; you just have to get used to it. A
very common beginner error is to try and write this:

\begin{Verbatim}[fontsize=\small,samepage=true,frame=single,framesep=3mm]
understanding[understanding < 12 | > 18]
\end{Verbatim}

This seems to make perfect sense, especially since it mimics the natural
English sentence: ``give me all values where \texttt{understanding} is
\textit{less than 12 or greater than 18}.'' Unfortunately, it doesn't work like
that in Python. The rule is: each side of an \textsl{and} or an \textsl{or}
must be a complete sentence. The phrase ``\texttt{understanding} is greater
than 18'' counts as a complete sentence, but ``is greater than 18'' does not.

\index{boxies (square brackets)}
\index{[]@\texttt{[]} (boxies)}
\index{of@``of'' (array/\texttt{Series} access)}

Also, whenever I see a line of code that specifies a key to a \texttt{Series}
(or array), I mentally pronounce the opening boxie (``\texttt{[}'') as the word
``of''. So when I read:

\begin{Verbatim}[fontsize=\small,samepage=true,frame=single,framesep=3mm]
print(x[5])
\end{Verbatim}
\vspace{-.2in}

I say to myself ``print x \textbf{of} five.''

\index{where@``where'' (array/\texttt{Series} query)}
However, whenever I see a \textit{query}, I mentally pronounce the 
boxie as the word ``where''. So when I read:

\begin{Verbatim}[fontsize=\small,samepage=true,frame=single,framesep=3mm]
print(x[x > 12])
\end{Verbatim}
\vspace{-.2in}

I say to myself ``print x \textbf{where} x is greater than 12.'' I've found
this helpful in making sense of the meaning of queries, since they're
complicated enough as it is!
