
\chapter{Evaluating a classifier}

\index{performance (of a classifier)}

Once we've built a classifier -- whether it's a decision tree or any other kind
-- the next step is to evaluate it to see how well it performs. This is
sometimes called the classifier's \textbf{performance}. It will determine
whether we deem it accurate enough to set it loose in the field, and if so, how
accurate we can expect its predictions to be.

At the risk of repeating myself (recall my stern lecture from
section~\ref{cantTestOnTrainingData}) you must evaluate your classifier by
testing it on data that was \textit{not} used to train it. On
p.~\pageref{sampleRows} we learned how to randomly divide a data set into
separate training and test sets.

Suppose we've done that. Suppose the \texttt{students} \texttt{DataFrame} from
chapters~\ref{ch:decisionTrees1} and \ref{ch:decisionTrees2} was the result of
randomly choosing 70\% of the labeled examples from an original data set, and
that we have preserved the remaining 30\% of the rows in a \texttt{DataFrame}
called \texttt{students\_test}, which contains:

\begin{Verbatim}[fontsize=\small,samepage=true,frame=single,framesep=3mm,xleftmargin=2.4cm,xrightmargin=2.5cm]
   Major     Age Gender   VG
 0  CPSC   young      F   No
 1  CPSC     old      O   No
 2  MATH     old      F   No
 3  CPSC  middle      M   No
 4  PSYC  middle      F  Yes
 5  PSYC   young      F   No
 6  MATH     old      M  Yes
 7  CPSC  middle      M  Yes
\end{Verbatim}

Our question is: ``how well does our classifier do on this test data?''

\section{What ``doing well'' means}

\index{accuracy, classification}
\index{classification!accuracy}

The most common (and simplest) way to measure a classifier's performance is to
simply count how many of the test points it correctly classifies, and divide by
the total number of test points. This gives us the classification
\textbf{accuracy} as a fraction between 0 and 1 (or, if we want to multiply by
100, a ``percentage accuracy'' from 0\% to 100\%.) It's possible to do this
because, as you'll remember, our test data is comprised of \textit{labeled}
examples, just like our training data is. Therefore, we know the ``right
answer'' for each test point, and we can simply compare it to our classifier's
prediction.

Even though this is the most common approach, it's worth taking a moment to
consider alternatives. The key assumption of this accuracy measure is that
\textit{all kinds of prediction errors are equal.} In the videogame case, we're
saying that mistakenly labeling a videogamer as a non-videogamer is ``just as
bad'' as mistakenly labeling a non-videogamer as a videogamer. And that might
be just the right thing for our gaming company to do.

But consider other settings. Suppose that our classifier's inputs are features
from an MRI image, and our prediction is ``cancer'' or ``no cancer.'' Now, it's
a much different story. Mistakenly predicting that a certain patient has cancer
when they actually don't might throw a needless scare into them. That's bad.
But it's far worse in the other direction: mistakenly giving a clean bill of
health to a patient who actually has early stage cancer risks losing a life. In
cases like this, we would need to penalize our classifier more harshly for
false negatives than for false positives.

It's also a different story when the labels aren't equally represented. Recall
the NFL fan prediction problem from Figure~\ref{fig:labeledTrainTest}
(p.~\pageref{fig:labeledTrainTest}). Consider if we performed fan prediction in
a city like Dallas, which is comprised of (say) 99\% Cowboys fans and only 1\%
Redskins fans. If we were to penalize a classifier equally for
mistaken-Cowboy-predictions and mistaken-Redskin-predictions, a one-line
classifier could earn a pretty good score:

\begin{Verbatim}[fontsize=\footnotesize,samepage=true,frame=single,framesep=3mm]
def predict(age, hometown, current_residence, yrs_in_residence):
    return "Cowboys"
\end{Verbatim}
 
It's not even worth trying hard to ferret out the few Redskins fans if we're
going to be docked a full point every time we dare to predict one. They're just
too rare. The only way to get a classifier to be bold and try to identify the
tiny population of Redskins fans is to penalize it more heavily for missing
them than for falsely identifying them.

Anyway, for the rest of this chapter, we'll use the vanilla ``count all
prediction mistakes equally'' approach, but it's worth remembering that this
doesn't make sense in all situations.

\section{Calculating accuracy in Python}

Calculating your classifier's accuracy is actually a snap. Once your
classifier's code is in a function, you just need a loop.

Return to the videogame example from last chapter, and the decision tree
classifier we wrote on p.~\pageref{fig:completeVgTreePython}. We'll use a
counter variable, initialized to zero, that will keep track of our number of
correct predictions. We'll then loop through each row of the test set, feeding
that row's features to the classifier function. If the return value from the
classifier matches the value of that row's target, ka-ching! We increment our
counter to increase our score. If it doesn't, we don't. At the end, we divide
by the number of test points to get our percentage. Simple!

\begin{Verbatim}[fontsize=\footnotesize,samepage=true,frame=single,framesep=3mm]
count = 0
for row in students_test.itertuples():
    if predict(row.Major, row.Age, row.Gender) == row.VG:
        count += 1

accuracy = count / len(students_test) * 100
print("Our accuracy on the test set was {}%.".format(accuracy,
    count, len(students_test)))
\end{Verbatim}
\vspace{-.3in}

\begin{Verbatim}[fontsize=\small,samepage=true,frame=leftline,framesep=5mm,framerule=1mm]
Our accuracy on the test set was 87.5%.
\end{Verbatim}

\smallskip

If we want more detail, we could print a message for each prediction, and flag
the incorrect ones for easy identification:

\begin{Verbatim}[fontsize=\footnotesize,samepage=true,frame=single,framesep=3mm]
count = 0
for row in students_test.itertuples():
    if predict(row.Major, row.Age, row.Gender) == row.VG:
        print("  Predicted {}/{}/{} right!".format(row.Major,
            row.Age, row.Gender))
        count += 1
    else:
        print("X Predicted {}/{}/{} wrong. :(".format(row.Major,
            row.Age, row.Gender))

accuracy = count / len(students_test) * 100
print("Our accuracy on the test set was {}% ({}/{}).".format(
    accuracy, count, len(students_test)))
\end{Verbatim}
\vspace{-.3in}

\begin{Verbatim}[fontsize=\footnotesize,samepage=true,frame=leftline,framesep=5mm,framerule=1mm]
  Predicted CPSC/young/F right!
  Predicted CPSC/old/O right!
  Predicted MATH/old/F right!
X Predicted CPSC/middle/M wrong. :(
  Predicted PSYC/middle/F right!
  Predicted PSYC/young/F right!
  Predicted MATH/old/M right!
  Predicted CPSC/middle/M right!
Our accuracy on the test set was 87.5% (7/8).
\end{Verbatim}

Not too shabby. As you can see, the only test point we missed was the male
\texttt{middle}-aged \texttt{CPSC} major, which our classifier figured would
be a videogamer. Live and learn.

The data size here is laughably small so that I can fit everything on the page.
But it's worth considering these three quantities anyway:

\begin{center}
\begin{tabular}{r|c}
Classifier's performance on training set & 94.1\% (16/17) \\
\hline
Classifier's performance on test set & 82.5\% (7/8) \\
\hline
Just using the prior on test set & 62.5\% (5/8) \\
\end{tabular}
\end{center}

\index{contradiction (in a training set)}

These three quantities will nearly always be in this order from top to bottom.
When we test our classifier on the very data it was trained on, we get an
inflated view of its accuracy -- for decision trees, recall, it will always be
100\% less any contradictions. Testing it on the data it has not yet seen gives
the truer (more realistic) picture. Finally, your classifier had better
outperform just using the prior (here, choosing ``No'' because the majority of
training points were ``No'') or this whole thing is a pretty useless
enterprise!
