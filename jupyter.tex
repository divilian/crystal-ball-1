
\chapter{A trip to Jupyter}

\label{programmingEnvironment}
\index{execute}
\index{edit}
\index{language}
\index{environment!programming}
\index{programming environment}
Python is a ridiculously popular \textbf{language} for programming and data
science (currently the third most widely used in the world\footnote{See
\url{https://www.tiobe.com/tiobe-index/}.}) which is one of many reasons we're
using it for this course. The language itself is different from the
\textbf{programming environment} used to write code in it, just as ``English''
is different from ``Microsoft Word'' and ``Google Docs.'' A programming
environment is just a fancy name for a tool or application used to write
programs. At a minimum, it must include a way to \textbf{edit} (write and
revise) code, and a way to \textbf{execute} (run) it.

\index{Jupyter Notebooks}
\index{IDE}
\index{Spyder}
There are many different programming environments data scientists use to write
Python code, just as there are many different word processing apps people use
to write English. The choice largely comes down to personal preference. Some
use full-blown \textbf{IDE}s (``integrated development environments'') like
Spyder or Atom; some use text-based tools like Notepad++ or \texttt{vim}. In
this class, we're going to use the friendly and minimalistic ``\textbf{Jupyter
Notebooks}'' environment since it's appropriate for an intro experience.

\section{Jupyter Notebooks}

\index{cell}

The concept of a Jupyter Notebook is simple: it's a Web page with editable
``\textbf{cell}s.'' Each cell is a little text window you can type in.

\pagebreak
There are three kinds of cells in Jupyter Notebooks:
\vspace{-.15in}

\begin{description}
\index{cell!raw}
\item{\textbf{Raw.}} ``Raw'' is dumb. Never use it.
\index{Markdown}
\index{render}
\index{cell!Markdown}
\item{\textbf{Markdown.}} ``Markdown'' cells are for \textit{English text}, not Python
code. They're mostly used to describe and annotate what you're doing in the
code cells, like a running commentary. You can type plain-ol' text in a
Markdown cell, plus various cutesy formatting adornments like boldface (putting
double-splats (\texttt{**}) around a word or phrase), italics (single splats),
outline headings (prefacing a line with one or more hashtags (like \texttt{\#}
or \texttt{\#\#\#}), and so forth.\footnote{For a complete list of formatting
options, see
\url{https://github.com/adam-p/markdown-here/wiki/Markdown-Cheatsheet}.} When
you type in a Markdown cell, you see the raw text and formatting; to actually get Jupyter to
\textbf{render} your cells and make them pretty, you choose ``Run All'' from
the \index{run all@``Run All''} \index{cell menu@``Cell'' CoCalc menu} ``Cell'' menu.

\index{cell!Code}
\item{\textbf{Code.}} The most important cells are ``Code'' cells which contain (duh)
code. When executed (again, by choosing ``Run All'' from the ``Cell'' menu) they actually carry out
the Python instructions you have typed in that cell, and display any results.

\end{description}

By the way, a common snafu is to somehow accidentally click in a way that
changes the type of a cell from ``Code'' to one of the other types. If you do
this, the Python code in that cell won't execute until you change the type back
to ``Code'' (more on this below).

\index{cell!type dropdown}
\index{CoCalc}
Figure~\ref{fig:jupyterNotebook} shows a Jupyter Notebook hosted by the
\textbf{CoCalc} cloud computing platform, which we'll use this semester. It has
two cells, one Markdown and one Code. Note carefully the \textbf{cell-type
dropdown} which is kind of hidden in the middle of the page: it currently reads
``Code'' because the second cell is the one that's highlighted. (If we clicked
to highlight and edit the top cell, that dropdown would change to
``Markdown.'')

\begin{figure}[!h]
\centering
\includegraphics[width=0.85\textwidth]{firstNotebook.png} \\
\bigskip
\includegraphics[width=0.85\textwidth]{firstNotebook2.png}
\medskip
\caption{A Jupyter Notebook with one Markdown cell and one Code cell. In the
top image, the two cells have been edited but not yet ``run'' -- hence
the Markdown formatting is unrendered and the code has not been executed. The
bottom pane shows both cells after the use has chosen ``Run All'' from the
``Cell'' menu.}
\label{fig:jupyterNotebook}
\end{figure}

\index{code snippet}
\index{snippet}
\index{output}
\index{*@\texttt{*} (splat)}
\index{splat}
The top figure shows the two cells before the user has done a ``Run All'' from
the ``Cell'' menu: all\index{run all@``Run All''} \index{cell menu@``Cell'' CoCalc menu} 
the Markdown is unrendered (see the literal splats and hashtags) and the code
is just sitting there. After ``Run All,'' the picture changes:
you see the formatted message in the top cell, and the \textbf{output} of the
Python code snippet after it runs. (The latter is easy to miss; stare at that
bottom picture and find the ``\texttt{Our country is 243 years old!}'' message.
That's the ``output.'') We haven't yet covered what that Python code means
(that's the main subject of this book) but you can probably guess a great deal
about what it's doing.

%(Incidentally, you can see that when you Shift-Enter the bottom cell to run the
%code, Jupyter also creates a new empty cell at the end of the Notebook for you
%to type in. I find this annoying, but whatever.)

\section{Code and output}

Incredibly, that's about it. Everything else in this book is going to concern
what to type in those Code cells and how to interpret its output.

From now on, whenever I give example Python code in this book, I'll write it in
a box like this:

% TODO: update year from 2019 to 2021.

\begin{Verbatim}[fontsize=\small,samepage=true,frame=single,framesep=3mm]
founding = 1776
usa_age = 2019 - founding
print("Our country is {} years old!".format(usa_age))
\end{Verbatim}

That box means ``this stuff goes in a Code cell of a Notebook.''

When I write the corresponding output (\textit{i.e.}, what gets printed on the
page immediately below the code cell when ``Run All'' is chosen from the
``Cell'' menu), I'll write it like this:
\index{run all@``Run All''} \index{cell menu@``Cell'' CoCalc menu} 
\begin{Verbatim}[fontsize=\small,samepage=true,frame=leftline,framesep=5mm,framerule=1mm]
Our country is 243 years old!
\end{Verbatim}

That vertical bar means ``this stuff is the printed result of executing the
code cell.''

Easy enough. Onward!
