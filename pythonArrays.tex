
% np.array([]), np.arange(), np.zeros(),
% np.loadtxt("file", dtype=object, delimiter='!!!')
% +, -, *, /
% np.sqrt()
% np.round( , num_decimals)
% np query syntax

% indexing, slices

\chapter{Arrays in Python: NumPy \texttt{ndarray}s}

\index{array!in NumPy}
\index{ndarray@\texttt{ndarray} (NumPy)}
\index{list@\texttt{list}}
There are several candidates in the Python language for representing the type
of array structure we introduced in chapter~\ref{ch:aggregateData}. One is the
plain-ol' Python \textbf{\texttt{list}}, which you may have used if you've
taken a computer science course in Python. Turns out, \texttt{list}s are going
to be too slow for us once we start dealing with a lot of data, plus there are
a lot of things that it won't do for us automatically that are handy to have.
Another choice is the Pandas \texttt{Series} which we'll actually cover in
chapter~\ref{ch:pythonAssocArrays} -- oddly, that one turns out to do to
\textit{much}, rather than too little, for our purposes here. A happy medium is
the \textbf{\texttt{ndarray}} from the NumPy package. Before we do that,
however, we need to learn what a \textbf{package} actually is, and how to use
one.

\section{Packages}
\index{package}

Back in my day (circa 1990's) when someone wanted to write a computer program,
they wrote the entire thing themselves, line by line.

type()
dtype
