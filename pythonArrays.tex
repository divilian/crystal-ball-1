
% np.array([]), np.arange(), np.zeros(),
% np.loadtxt("file", dtype=object, delimiter='!!!')
% +, -, *, /
% np.sqrt()
% np.round( , num_decimals)
% np query syntax

% indexing, slices

\chapter{Arrays in Python: NumPy \texttt{ndarray}s}

\index{array!in NumPy}
\index{ndarray@\texttt{ndarray} (NumPy)}
\index{list@\texttt{list}, plain-ol'}
There are several candidates in the Python language for representing the type
of array structure we introduced in chapter~\ref{ch:aggregateData}. One is the
plain-ol' Python \textbf{\texttt{list}}, which you may have used if you've
taken a computer science course in Python. Turns out, \texttt{list}s are going
to be too slow for us once we start dealing with a lot of data, plus there are
a lot of things that it won't do for us automatically that are handy to have.
Another choice is the Pandas \texttt{Series} which we'll actually cover in
chapter~\ref{ch:pythonAssocArrays} -- oddly, that one turns out to do too
\textit{much}, rather than too little, for our purposes here. A happy medium is
the \textbf{\texttt{ndarray}} from the NumPy package\footnote{Most people seem
to pronounce this ``NUM-pie,'' although I've heard ``NUM-pee'' as well. Pick
your poison.}. Before we do that, however, we need to learn what a ``package''
actually is, and how to use one.

\section{Packages}
\index{package}

Back in my day (circa 1990's) when someone wanted to write a computer program,
they wrote the entire thing themselves, line by line. Everything you needed to
do -- from something complex like making a remote network connection to
something simple like computing the average of some numbers -- was up to you to
build. Code sharing over the Internet just wasn't much of a thing.

Today, the reverse is true. When you write a complex data analysis program,
\textit{most} of the code will actually be written by others, if you do it
right. This is because many, many smart people across the globe have written
snippets of code to do all the common (and some not-so-common) things you'll
want to do, and your job is to string them all together. Put another way:
you're given most of the Legos\textsuperscript{\textregistered} -- and even a
bunch of pre-assembled chunks made with dozens of
Legos\textsuperscript{\textregistered} each -- and your job is to construct
your masterpiece out of those building blocks.

\index{package}
\index{importing (a package)}
\index{package}
\index{calling a function@``calling'' a function}
\index{calling a method@``calling'' a method (on a variable)}
In Python, a \textbf{package} is a repository of useful functions and methods
that someone else has written. By \textbf{import}ing a package into your
program, you're making all those useful things available to you. Your own code
can then call those functions/methods whenever you see fit. It's the modular,
organized, and elegant way to do things, in addition to saving a ton of time.

The first package we'll use is called \textbf{NumPy}, which stands for
``Numerical Python.'' To import it, you should include this exact line of code
in the \textit{first} Code cell of your Notebook:

\begin{Verbatim}[fontsize=\small,samepage=true,frame=single,framesep=3mm]
import numpy as np
\end{Verbatim}

Note that it's in all lower-case letters. Once that cell has been executed, you
now have access to all the NumPy ``stuff,'' which is the subject of this
chapter.

\section{The NumPy \texttt{ndarray}}

\index{array}
\index{ndarray@\texttt{ndarray} (NumPy)}
\index{matrix}
The actual data type that the NumPy package provides is called an
\texttt{ndarray}, which stands for ``n-dimensional array.'' If that sounds
heady, it kind of is, although in this course we're only ever going to use a
\textit{one}-dimensional array, which is super simple to understand. In fact,
it looks exactly like the examples in Figure~\ref{fig:array}
(p.~\pageref{fig:array}). ``One-dimensional'' just means that there is a single
index number, and the elements are all in a line.\footnote{A two-dimensional
array is a spreadsheety-looking thing also called a \textbf{matrix}. Each
element has \textit{two} index numbers: a row and a column. A three-dimensional
array is a cube, with three index numbers needed to specify an element.
\textit{Etc.}}

\subsection{Creating \texttt{ndarray}s}

There are many different ways to create an \texttt{ndarray}. We'll learn four
of them.

\subsubsection{Way 1: \texttt{np.array([])}}

\index{array@\texttt{array()} function (NumPy)}
The first is to use the \texttt{array()} function of the NumPy package, and
give it all the values explicitly. Here's the code to reproduce the
Figure~\ref{fig:array} examples:

\begin{Verbatim}[fontsize=\scriptsize,samepage=true,frame=single,framesep=3mm]
followees = np.array(['@katyperry','@rihanna','@Cristiano','@TheEllenShow'])
balances = np.array([1526.73, 98774.91, 1000000, 4963.12, 123.19])
\end{Verbatim}

\index{boxies (square brackets)}
\index{[]@\texttt{[]} (boxies)}
\index{bananas (parentheses)}
\index{()@\texttt{()} (bananas)}
It's simple, but don't miss the syntactical gotcha: \textit{you must include a
pair of boxies inside the bananas.} Why? Reasons.\footnote{For the experienced
reader, what we're actually doing here is creating a plain-ol' Python list
(with the boxies), and then calling the \texttt{array()} function with that
list as an argument.} For now, just memorize that for this function -- and this
function only -- we use ``\texttt{([}\textsl{...stuff...}\texttt{])}'' instead
of ``\texttt{(}\textsl{...stuff...}\texttt{)}'' when we call it.

\index{package}
\index{function}
\index{method}
By the way, the attentive reader might object to me calling \texttt{array()} a
function, instead of a method. Isn't there a word-and-a-dot before it, and
isn't that a ``method thing?'' Shrewd of you to think that, but actually no,
and the reason is that ``\texttt{np}'' isn't the name of a variable, but the
name of a \textit{package}. When we say ``\texttt{np.array()}'' what we're
saying is: ``Python, please call the \texttt{array()} function \textit{from the
\texttt{np} package}.'' The word-and-dot syntax does double-duty.

\index{type@\texttt{type()}}
We can call the \texttt{type()} function, as we did back on
p.~\pageref{typeFunction}, to verify that yes indeed we have created
\texttt{ndarray}s:

\begin{Verbatim}[fontsize=\small,samepage=true,frame=single,framesep=3mm]
print(type(followees))
print(type(balances))
\end{Verbatim}

\begin{Verbatim}[fontsize=\small,samepage=true,frame=leftline,framesep=5mm,framerule=1mm]
numpy.ndarray
numpy.ndarray
\end{Verbatim}

This is useful, but sometimes we want to know what underlying atomic data type
the array is comprised of. To do that, we attach ``\texttt{.dtype}''
(confusingly, \textit{without} bananas this time) to the end of the variable
name. ``\texttt{.dtype}'' stands for ``data type.'' Here goes:

\begin{Verbatim}[fontsize=\small,samepage=true,frame=single,framesep=3mm]
print(type(followees))
print(type(balances))
\end{Verbatim}

\begin{Verbatim}[fontsize=\small,samepage=true,frame=leftline,framesep=5mm,framerule=1mm]
dtype('<U13')
dtype('float64')
\end{Verbatim}

Whoa, what does \textit{that} stuff mean? It's a bit hard on the eyes, but let
me explain. The underlying data type of \texttt{followees} is (bizarrely)
``\texttt{<U13}'' which in English means ``strings of Unicode
characters\footnote{A ``Unicode character'' is just a fancy way of saying ``a
character, which might not be English.'' NumPy is capable of storing more than
just a-b-c's in its strings; it can store symbols from Greek, Arabic, Chinese,
\textit{etc.} as well.}, each of which is 13 characters long or less.'' (If you
bother to count, you'll discover that the longest string in our
\texttt{followees} array is the last one, \texttt{'@TheEllenShow'}, which is
exactly 13 characters long.) The ``\texttt{float64}'' thing means
``\texttt{float}s, each of which is represented with 64 bits\footnote{A ``bit''
-- which is short for ``binary digit'' -- is the tiniest piece of information a
computer can store: it's a single 0 or 1.} in memory.

You don't need to worry about any of those details. All you need to know is: if
an array's \texttt{dtype} has ``\texttt{<U}'' in it, then it's composed of
strings; and if it has the word ``\texttt{int}'' or ``\texttt{float}'' in it,
it means one of those two old friends from chapter~\ref{ch:atomicData}.

\index{homogeneous}
\index{list@\texttt{list}, plain-ol'}
Incidentally, you'll recall from chapter~\ref{ch:aggregateData} that an array
is \textit{homogeneous}, which means all its elements are of the same type.
NumPy enforces this. If you try to combine them:

\begin{Verbatim}[fontsize=\small,samepage=true,frame=single,framesep=3mm]
weird = np.array([3, 4.9, 8])
strange = np.array([18, 73.0, 'bob', 22.8])
\end{Verbatim}

you'll discover that NumPy converts them to all be of the same type:

\begin{Verbatim}[fontsize=\small,samepage=true,frame=single]
print(weird)
print(weird.dtype)
print(strange)
print(strange.dtype)
\end{Verbatim}

\begin{Verbatim}[fontsize=\small,samepage=true,frame=leftline,framesep=5mm,framerule=1mm]
[ 3.   4.9  8. ]
dtype('<U3')
['18' '73.0' 'bob' '22.8']
dtype('float64')
\end{Verbatim}

See how the \texttt{int}s \texttt{3} and \texttt{8} from the first array were
converted into the \texttt{float}s \texttt{3.}~and \texttt{8.}; meanwhile, all
of the numerical elements of the second array got converted to \texttt{str}s.
(If you think about it, that's the only direction the conversions could go.)

\subsubsection{Way 2: \texttt{np.zeros()}}

\index{increment}
\index{decrement}
\index{counter variable}

It will often be useful to create an array, possibly a large one, with all
elements equal to \textit{zero} initially. Among other scenarios, we often need
to use a bunch of counter variables to, well, count things. Suppose, for
example, that we had a giant array that held the numbers of likes that each
Instagram photo had. When someone likes a photo, that photo's appropriate
element in the array should be \textbf{incremented} (raised in value) by one.
Similarly, if someone unlikes it, then its value in the array should be
\textbf{decremented} by one.

\index{zeros@\texttt{zeros()} function (NumPy)}
An easy way to do this is NumPy's \texttt{zeros()} function:

\begin{Verbatim}[fontsize=\small,samepage=true,frame=single,framesep=3mm]
photo_likes = np.zeros(40000000000)
\end{Verbatim}

(although I'll bet you don't have enough memory on your laptop to actually
store an array this size! Instagram sure has a lot of pics...) When I do this
on my Data Science cluster, I get this:

\begin{Verbatim}[fontsize=\small,samepage=true,frame=single,framesep=3mm]
print(photo_likes)
print(type(photo_likes))
print(photo_likes.dtype)
\end{Verbatim}

\begin{Verbatim}[fontsize=\small,samepage=true,frame=leftline,framesep=5mm,framerule=1mm]
array([ 0.,  0.,  0., ...,  0.,  0.,  0.])
numpy.ndarray
float64
\end{Verbatim}

Don't miss the ``\texttt{...}'' in the middle: it means ``there are
(potentially) a lot of elements here that are not being shown, for
conciseness.'' Also notice that \texttt{zeroes()} makes an array of
\texttt{float}s, not \texttt{int}s.


