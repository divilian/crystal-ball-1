
% np.array([]), np.arange(), np.zeros(),
% np.loadtxt("file", dtype=object, delimiter='!!!')
% +, -, *, /
% np.sqrt()
% np.round( , num_decimals)
% np query syntax

% indexing, slices

\chapter{Arrays in Python: NumPy \texttt{ndarray}s}

\index{array!in NumPy}
\index{ndarray@\texttt{ndarray} (NumPy)}
\index{list@\texttt{list}}
There are several candidates in the Python language for representing the type
of array structure we introduced in chapter~\ref{ch:aggregateData}. One is the
plain-ol' Python \textbf{\texttt{list}}, which you may have used if you've
taken a computer science course in Python. Turns out, \texttt{list}s are going
to be too slow for us once we start dealing with a lot of data, plus there are
a lot of things that it won't do for us automatically that are handy to have.
Another choice is the Pandas \texttt{Series} which we'll actually cover in
chapter~\ref{ch:pythonAssocArrays} -- oddly, that one turns out to do to
\textit{much}, rather than too little, for our purposes here. A happy medium is
the \textbf{\texttt{ndarray}} from the NumPy package. Before we do that,
however, we need to learn what a \textbf{package} actually is, and how to use
one.

\section{Packages}
\index{package}

Back in my day (circa 1990's) when someone wanted to write a computer program,
they wrote the entire thing themselves, line by line. Everything you needed to
do -- from something complex like making a remote network connection to
something simple like computing the average of some numbers -- was up to you to
build. Code sharing over the Internet just wasn't much of a thing.

\index{legos@Legos\textsuperscript{\textregistered}} -- and even a bunch of
Today, the reverse is true. When you write a complex data analysis program,
\textit{most} of the code will actually be written by others, if you do it
right. This is because many, many smart people across the globe have written
snippets of code to do all the common (and some not-so-common) things you want
to do, and your job is to string them all together. Put another way: you're
given most of the Legos\textsuperscript{\textregistered} -- and even a bunch of
pre-assembled chunks with dozens of Legos\textsuperscript{\textregistered} each
-- and your job is to construct your masterpiece out of those building blocks.

\index{package}
\index{importing (a package)}
\index{package}
\index{calling a function@``calling'' a function}
\index{calling a method@``calling'' a method (on a variable)}
In Python, a \textbf{package} is a repository of useful functions and methods
that someone else has written. By \textbf{import}ing a package into your
program, you're making all those useful things available to you. Your own code
can then call those functions/methods whenever you see fit. It's the modular,
organized, and elegant way to do things, in addition to saving a ton of time.

The first package we'll use is called \textbf{NumPy}, which stands for
``Numerical Python.'' To import it, you should include this exact line of code
in the \textit{first} Code cell of your Notebook:

\begin{Verbatim}[fontsize=\small,samepage=true,frame=single,framesep=3mm]
import numpy as np
\end{Verbatim}

Note that it's in all lower-case letters. Once that cell has been executed, you
now have access to all the NumPy ``stuff,'' some of which is the subject of
this chapter.

\section{The NumPy \texttt{ndarray}}

type()
dtype
