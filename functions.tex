
\chapter{Functions}
\label{ch:functions}

\index{function}
\index{writing a function@writing a function}
\index{non-linear}

And now for the very last ``pure programming'' lesson of the book: writing
\textbf{function}s. This is more or less the final tool in the programmer's
toolkit, and as I've learned over my years of teaching, it often causes the
most trouble.

Now you might be thinking, ``hey waitaminit, we've known about functions since
all the way back on p.~\pageref{function}. This is something new?'' Yes it is.
Previously in this book, we've done a lot of \textit{calling} functions -- from
\texttt{len()} to \texttt{np.append()} to \texttt{pd.read\_csv()} to
\texttt{scipy.stats.chi2\_contingency()} -- that someone else has written for
us. In this chapter, we look behind the curtain and join the production staff:
we write our \textit{own} functions.

\section{Why do all this}

It turns out there's a lot of syntactic nonsense involved to get all the wiring
right when you do this. It can cause students to pull their hair out. So it's
worth asking at the outset: what do we get for all this pain?

\index{modular}

The answer is subtle, and can seem underwhelming at first, but it's crucial. It
essentially boils down to this lesson: any complex creative work (including a
computer program) should be \textbf{modular} in its design. This means that it
should be composed of smaller building blocks, which are in turn composed of
smaller building blocks, and that the whole thing should comprise an organized
whole.

Any other way of doing it leads to madness.

\index{car engine}

Think of a car engine. When a mechanic opens up the hood, he or she doesn't see
just one big monolithic thing called ``The Engine,'' but rather piston
assemblies, spark plugs, water pumps, drive shafts, and lots of other
subsystems. It's what allows piece by piece investigation of problems, and
piece by piece replacement of bad parts.

\index{creativity}
\index{rock 'n' roll}

Or think of a rock 'n' roll tune. It's not just an impenetrable mass of sound.
Instead, it's a collection of recognizable bass lines, drum sequences, vocal
patterns, and variations on common guitar riffs. This isn't to minimize the
creativity involved in its orchestration; in fact, the novel combination of the
myriad possible building blocks \textit{is} the creativity. If it were just an
impermeable wall of sound, it would be noise, not music.

% TODO Susan Polgar
% I like to tell the story of Susan Polgar, and the ``simul'' match I saw her
% play.

\index{spaghetti code@``spaghetti code''}

In the same way, once your data analysis code approaches a certain size, it
really must be written in a modular way or it will become a hopelessly tangled
mess, what programmers refer to as ``spaghetti code.'' And the way to achieve
this is by writing it in terms of functions that you then call at the
appropriate time.

\index{reusable}
\index{wheel, reinventing}

One other advantage to this approach, by the way, is that functions are
\textbf{reusable}. Think of how many different programmers all over the world
have had reason to call \texttt{np.sort()}, or \texttt{scipy.stats.pearsonr()}!
The same function becomes applicable in a variety of different contexts, so
that nobody has to reinvent the wheel.

\section{The \texttt{def} statement}

\index{def statement@\texttt{def} statement}

Okay, down to brass tacks. The way to create (not call) a function in Python is
to use the \texttt{def} statement. For our first example, let's write a
function to compute an (American) football team's score in a game:

\index{football\_score@\texttt{football\_score}}

\begin{Verbatim}[fontsize=\small,samepage=true,frame=single,framesep=3mm]
def football_score(num_tds, num_fgs):
    return num_tds * 7 + num_fgs * 3
\end{Verbatim}

For those not familiar with football scoring, each ``touchdown'' (or TD for
short) a team scores is worth seven points, and each ``field goal'' (or FG) is
worth three. (For those who \textit{are} familiar with football scoring, please
forgive the simplifications here -- extra point kicks, safeties, \textit{etc.}
It's a first example.)

\index{argument}
\index{bananas (parentheses)}
\index{()@\texttt{()} (bananas)}

As you can see from the code snippet, above, the word \texttt{def} (which
stands for ``define,'' I think, since we're ``defining'' -- a.k.a.~writing -- a
function) is followed by the \textit{name} of our function, which like a
variable name can be any name we choose. After the name is the list of
\textbf{argument}s to the function, in bananas.

\index{function!header}
\index{function!body}
\index{indentation}

All that is the \textbf{header} of the function. The \textbf{body}, like other
``bodies'' we've seen (p.~\pageref{loopBody}, p.~\pageref{ifBody}) is
\textit{indented} underneath. The \texttt{football\_score} body is just one
line long, but it can be as many lines as necessary.

\index{return value}
\index{output!of a function}
\index{input!of a function}
\index{return statement@\texttt{return} statement}

Finally, we see the word ``\texttt{return}'' on that last line. This is how we
control the \textbf{return value} which is given back to the code that called
our function (review section~\ref{returnValues} on p.~\pageref{returnValues} if
you need a refresher on this). Whenever a \texttt{return} statement is
encountered during the execution of a function, the function immediately
\textit{stops} executing, and the specified value is handed back to the calling
code. More on that in a minute.

\section{Writing vs.~calling}

Now here's one of the most perplexing things for beginners. 
