
\chapter{Functions}
\label{ch:functions}

\index{function}
\index{writing a function@writing a function}
\index{non-linear}

And now for the very last ``pure programming'' lesson of the book: writing
\textbf{function}s. This is more or less the final tool in the programmer's
toolkit, and as I've learned over my years of teaching, it often causes the
most trouble.

Now you might be thinking, ``hey waitaminit, we've known about functions since
all the way back on p.~\pageref{function}. This is something new?'' Yes it is.
Previously in this book, we've done a lot of \textit{calling} functions -- from
\texttt{len()} to \texttt{np.append()} to \texttt{pd.read\_csv()} to
\texttt{scipy.stats.chi2\_contingency()} -- that someone else has written for
us. In this chapter, we look behind the curtain and join the production staff:
we write our \textit{own} functions.

\section{Why do all this}

\section{The \texttt{def} statement}

\index{def statement@\texttt{def} statement}

