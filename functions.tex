
\chapter{Functions (1 of 2)}
\label{ch:functions}

\index{function}
\index{writing a function@writing a function}
\index{non-linear}

And now for the very last ``pure programming'' lesson of the book: writing
\textbf{function}s. This is more or less the final tool in the programmer's
toolkit, and as I've learned over my years of teaching, it often causes the
most trouble.

Now you might be thinking, ``hey waitaminit, we've known about functions since
all the way back on p.~\pageref{function}. This is something new?'' Yes it is.
Previously in this book, we've done a lot of \textit{calling} functions -- from
\texttt{len()} to \texttt{np.append()} to \texttt{pd.read\_csv()} to
\texttt{scipy.stats.chi2\_contingency()} -- that someone else has written for
us. In this chapter, we look behind the curtain and join the production staff:
we write our \textit{own} functions.

\section{Why do all this}

It turns out there's a lot of syntactic nonsense involved to get all the wiring
right when you do this. It can cause students to pull their hair out. So it's
worth asking at the outset: what do we get for all this pain?

\index{modular}

The answer is subtle, and can seem underwhelming at first, but it's crucial. It
essentially boils down to this lesson: any complex creative work (including a
computer program) should be \textbf{modular} in its design. This means that it
should be composed of smaller building blocks, which are in turn composed of
still smaller building blocks. The entire thing should comprise an organized
whole.

Any other way of doing it leads to madness.

\index{car engine}

Think of a car engine. When a mechanic opens up the hood, he or she doesn't see
just one big monolithic thing called ``The Engine,'' but rather piston
assemblies, spark plugs, water pumps, drive shafts, and lots of other
subsystems. It's what allows piece by piece investigation of problems, and
piece by piece replacement of bad parts.

\index{creativity}
\index{rock 'n' roll}

Or think of a rock 'n' roll tune. It's not just an impenetrable mass of sound.
Instead, it's a collection of recognizable bass lines, drum sequences, vocal
patterns, and variations on common guitar riffs. I don't mean to minimize the
creativity involved in its orchestration; in fact, the novel combination of the
myriad possible building blocks \textit{is} the creativity. If the song were
just an impermeable wall of sound, it would be noise, not music.

% TODO Susan Polgar
% I like to tell the story of Susan Polgar, and the ``simul'' match I saw her
% play.

\index{spaghetti code@``spaghetti code''}

In the same way, once your data analysis code approaches a certain size, it
really must be written in a modular way or it will become a hopelessly tangled
mess, what programmers refer to as ``spaghetti code.'' And the way to achieve
this is by writing it in terms of functions that you then call at the
appropriate time.

\index{reusable}
\index{wheel, reinventing}

One other advantage to this approach, by the way, is that functions are
\textbf{reusable}. Think of how many different programmers all over the world
have had reason to call \texttt{np.sort()}, or \texttt{scipy.stats.pearsonr()}!
The same function becomes applicable in a variety of different contexts, so
that nobody has to reinvent the wheel.

\section{The \texttt{def} statement}

\index{def statement@\texttt{def} statement}

Okay, down to brass tacks. The way to create (not call) a function in Python is
to use the \texttt{def} statement. For our first example, let's write a
function to compute an (American) football team's score in a game:

\index{football\_score@\texttt{football\_score}}

\begin{Verbatim}[fontsize=\small,samepage=true,frame=single,framesep=3mm]
def football_score(num_tds, num_fgs):
    return num_tds * 7 + num_fgs * 3
\end{Verbatim}

For those not familiar with football scoring, each ``touchdown'' (or TD for
short) a team scores is worth seven points, and each ``field goal'' (or FG) is
worth three. (For those who \textit{are} familiar with football scoring, please
forgive the simplifications here -- extra point kicks, safeties, \textit{etc.}
It's a first example.)

\index{argument}
\index{bananas (parentheses)}
\index{()@\texttt{()} (bananas)}

As you can see from the code snippet, above, the word \texttt{def} (which
stands for ``define,'' since we're ``defining'' -- a.k.a.~writing -- a
function) is followed by the \textit{name} of our function, which like a
variable name can be any name we choose. After the name is the list of
\textbf{argument}s to the function, in bananas.

\index{function!header}
\index{function!body}
\index{indentation}

All that is the \textbf{header} of the function. The \textbf{body}, like other
``bodies'' we've seen (p.~\pageref{loopBody}, p.~\pageref{ifBody}) is
\textit{indented} underneath. The \texttt{football\_score} body is just one
line long, but it can be as many lines as necessary.

\index{return value}
\index{output!of a function}
\index{input!of a function}
\index{return statement@\texttt{return} statement}
\label{returnImmediatelyReturns}

Finally, we see the word ``\texttt{return}'' on that last line. This is how we
control the \textbf{return value} which is given back to the code that called
our function (review section~\ref{returnValues} on p.~\pageref{returnValues} if
you need a refresher on this). Whenever a \texttt{return} statement is
encountered during the execution of a function, the function immediately
\textit{stops} executing, and the specified value is handed back to the calling
code. More on that in a minute.

\section{Writing vs.~calling}

Now here's one of the most perplexing things for beginners. Consider this code:

\begin{Verbatim}[fontsize=\small,samepage=true,frame=single,framesep=3mm]
team_name = "Broncos"
num_tds = 3
num_fgs = 2

def football_score(num_tds, num_fgs):
    return num_tds * 7 + num_fgs * 3
\end{Verbatim}

It surprises many to learn that this code snippet \textit{does not compute
anything}, football score or otherwise. The reason? We only \textit{wrote} a
function; we didn't actually \textit{call} it.

This is sort of like building an impressive machine but then never pushing the
``On'' button. The above code says to do three things:

\begin{compactenum}
\item Create a \texttt{num\_tds} variable and set its value to the integer 3.
\item Create a \texttt{num\_fgs} variable and set its value to the integer 2.
\item Create a function called \texttt{football\_score} which, \textit{if it is
ever called in the future}, will compute and return the score of a football
game.
\end{compactenum}

In other words, that last step is just preparatory. It tells Python: ``by the
way, in case you see any code later on that calls a function named
`\texttt{football\_score},' here's the code you should run in response.''

To actually call your function, you have to use the same syntax we learned on
p.~\pageref{function}, namely:

\begin{Verbatim}[fontsize=\small,samepage=true,frame=single,framesep=3mm]
team_name = "Broncos"
num_tds = 3
num_fgs = 2

the_score = football_score(num_tds, num_fgs)
print("The {} scored {} points.".format(team_name, the_score))
\end{Verbatim}
\vspace{-.2in}

\begin{Verbatim}[fontsize=\small,samepage=true,frame=leftline,framesep=5mm,framerule=1mm]
The Broncos scored 27 points.
\end{Verbatim}

\index{main program}

Follow the thread of execution closely here. First, the three variables are
created, in what I'll often call ``the main program.'' By ``main,'' I really
just mean the stuff that's all the way flush-left, and thus not inside any
``\texttt{def}.'' It's the main program in the sense that when you execute the
cell, it's what immediately happens without needing to be explicitly called.

Then, after those three variables are created, the \texttt{football\_score()}
function is called, at which point \textit{the flow of execution is transferred
to the inside of the function.} Since this simple function has only one line of
code in its body (the \texttt{return} statement), executing it is really quick;
but it's still important to realize that for a moment, Python isn't ``in'' that
Broncos cell at all. Instead it jumps to the function, carries out the code
inside it, and then \texttt{return}s the value...

...right back into the waiting arms of the main program, which stores that
returned value (an integer 27, as it turns out) in a new variable named
\texttt{the\_score}. Then the flow continues, and the \texttt{print()}
statement executes as normal.

\index{bananas (parentheses)}
Bottom line: every time you want to run your function's code -- whether that's
a hundred times, once, or not at all -- you need to call it by typing the name
of the function (with no ``\texttt{def}'') followed by a banana-separated list
of arguments.

\section{Naming arguments}

\index{passing an argument@``passing'' an argument}
\index{argument}

Speaking of arguments, here's the next thing many students have trouble with.
The names of the variables that your main program passes to the function are
normally \textit{not} the same as the arguments defined by the function itself.

What? Yeah.

Consider this example:

\index{Colts, Baltimore}
\index{Jets, New York}
\index{Superbowl III}
\index{Namath, Joe}

\begin{Verbatim}[fontsize=\small,samepage=true,frame=single,framesep=3mm]
jets_touchdowns = 1
jets_field_goals = 3
jets_total = football_score(jets_touchdowns, jets_field_goals)

colts_tds = 1
colts_fgs = 0
colts_total = football_score(colts_tds, colts_fgs)

print("The Jets won {} to {}.".format(jets_total, colts_total))
\end{Verbatim}
\vspace{-.2in}

\begin{Verbatim}[fontsize=\small,samepage=true,frame=leftline,framesep=5mm,framerule=1mm]
The Jets won 16 to 7.
\end{Verbatim}

This code contains two calls to the \texttt{football\_score()} function. In the
first call, the variables \texttt{jets\_touchdowns} (with value 2) and
\texttt{jets\_field\_goals} (0) were passed. In the second,
\texttt{colts\_tds} (1) and \texttt{colts\_fgs} (3) were. In
\textit{neither} case were the arguments literally named ``\texttt{num\_tds}''
and ``\texttt{num\_fgs}'', which were the function's own argument names.

To be crystal clear: whenever \texttt{football\_score()} is called, two
arguments are passed to it. The function chooses to name the first one it
receives ``\texttt{num\_tds}'' and the second one it receives
``\texttt{num\_fgs}''. But these are \textit{its own personal names.} They
normally have nothing to do with what the calling code chooses to name them.

Why does it work this way? Perhaps this example makes clear the reason. If the
main program had to name its variables exactly the same as the (indented)
function did, then the function would not be reusable. In order for it to be
called with different values in different contexts, there needs to be this
flexible decoupling of variable names.

To reinforce the lesson, note too that you can call a function without even
having variables in the main program at all:

\begin{Verbatim}[fontsize=\small,samepage=true,frame=single,framesep=3mm]
x = football_score(5, 2)
print("Some mythical team scored {} points today.".format(x))
\end{Verbatim}
\vspace{-.2in}

\begin{Verbatim}[fontsize=\small,samepage=true,frame=leftline,framesep=5mm,framerule=1mm]
Some mythical team scored 41 points today.
\end{Verbatim}

\index{slot \#1}
\index{slot \#2}
Here we literally passed the values 5 and 2 to the function, instead of
creating variables to hold them. The function doesn't mind: it just says, ``hey
man, whatever's given to me in slot \#1, I'm going to name `\texttt{num\_tds},'
and whatever's delivered through slot \#2, I'm going to call
`\texttt{num\_fgs}.' I don't care what the outside world's variable names
are...or even whether they have names at all. I just work here.''

\section{Passing aggregate data to functions}

\index{atomic}
\index{aggregate data}
\index{IQR (interquartile range)}

Even though the previous example involved passing atomic data to a function,
you can totally pass aggregate data as well. Suppose we'd like to be able to
easily compute the IQR (recall p.~\pageref{IQR}) of a univariate data set.
Writing a function to do that is a snap:

\begin{Verbatim}[fontsize=\small,samepage=true,frame=single,framesep=3mm]
def IQR(some_data):
    return some_data.quantile(.75) - some_data.quantile(.25)
\end{Verbatim}

\index{NCAA}
\index{YouTube}
\index{num\_plays@\texttt{num\_plays}}
We can now call it on anything we like, like our examples from
p.~\pageref{YouTubeData} and p.~\pageref{NCAAData}:

\begin{Verbatim}[fontsize=\small,samepage=true,frame=single,framesep=3mm]
print("The IQR of the YouTube plays data is {}".format(IQR(num_plays)))
print("The IQR of the NCAA scoring data is {}".format(IQR(pts)))
\end{Verbatim}
\vspace{-.2in}

\begin{Verbatim}[fontsize=\small,samepage=true,frame=leftline,framesep=5mm,framerule=1mm]
The IQR of the YouTube plays data is 412.
The IQR of the NCAA scoring data is 15.
\end{Verbatim}

Again, we named the function's own argument (\texttt{some\_data}) something
totally different than the variables it was called with (\texttt{num\_plays}
and \texttt{pts}). This is a happy and healthy thing.

\section{Returning text}

So far, our functions have returned numeric answers. But they can certainly
return text as well. Here's a function which assembles a person's full name out
of his or her constituent components:

\begin{Verbatim}[fontsize=\small,samepage=true,frame=single,framesep=3mm]
def full_name(last_name, first_name, middle_initial):
    return first_name + " " + middle_initial + ". " + last_name

my_full_name = full_name("Davies", "Stephen", "C")
her_full_name = full_name("Clinton", "Hillary", "R")

print("Your author's full name is: {}".format(my_full_name))
print("Another person's full name is: {}".format(her_full_name))
\end{Verbatim}
\vspace{-.2in}

\begin{Verbatim}[fontsize=\small,samepage=true,frame=leftline,framesep=5mm,framerule=1mm]
Your author's full name is: Stephen C. Davies
Another person's full name is: Hillary R. Clinton
\end{Verbatim}

(Recall from p.~\pageref{concatenatingStrings} that the ``\texttt{+}'' operator
is used for the concatenation of strings.)

\section{Returning \texttt{True} or \texttt{False}}

\index{if statement@\texttt{if} statement}
\index{condition (of an \texttt{if} statement)}
It's common for a programmer to want a function which, instead of returning a
number or text, tells her \textit{whether or not something is true.} This lets
her use the return value of such a function as the condition of an \texttt{if}
statement.

\index{is\_positive@\texttt{is\_positive()}}
Here's a trivial example:

\begin{Verbatim}[fontsize=\small,samepage=true,frame=single,framesep=3mm]
def is_positive(number):
    if number > 0:
        return True
    else:
        return False

x = is_positive(13)
if x:
    print("Yes, 13 is positive!")
else:
    print("Alas, 13 isn't positive.")

if is_positive(-5):
    print("Yes, -5 is positive!")
else:
    print("Alas, -5 isn't positive.")
\end{Verbatim}
\vspace{-.2in}

\begin{Verbatim}[fontsize=\small,samepage=true,frame=leftline,framesep=5mm,framerule=1mm]
Yes, 13 is positive!
Alas, -5 isn't positive.
\end{Verbatim}

\index{boolean value}
\index{Boole, George}
The values \texttt{True} and \texttt{False} are called \textbf{boolean value}s,
after the 19th-century mathematician George Boole. Note that in Python they
must begin with \textit{capital} letters.

The somewhat odd-looking ``\texttt{if x:}'' line is possible because \texttt{x}
was set to the return value of a call to \texttt{is\_positive()}, and that
function returned a boolean value.

\index{return value}
\index{return statement@\texttt{return} statement}
\index{if/else statement@\texttt{if}/\texttt{else} statement}
Another thing this brief example illustrates is the presence of more than one
\texttt{return} statement in a function. You might have thought this was
useless, since on p.~\pageref{returnImmediatelyReturns} I mentioned that as
soon as a \texttt{return} is encountered during execution, the function is
immediately completed. Why, then, would one ever have more than one -- the
second one could never be reached, right? Wrong. The branching nature of the
\texttt{if}/\texttt{else} statement (above) means that the first
\texttt{return} will be skipped over in some situations (negative arguments,
\textit{etc.}) so it's perfectly sensible to have more than one.

\index{salutations}
\index{salutation@\texttt{salutation()}}
A more complicated example would be the ``salutation'' algorithm from
section~\ref{salutation} (p.~\pageref{salutation}), this time embodied in a
function:

\begin{Verbatim}[fontsize=\small,samepage=true,frame=single,framesep=3mm]
def salutation(gender, marital_status, degree):
    if degree == "PhD" or degree == "MD":
        return "Dr."
    elif gender == "male":
        return "Mr."
    elif gender == "female":
        if marital_status == "married":
            return "Mrs."
        elif marital_status == "single":
            return "Miss"
        else:
            return "Ms."
    else:
        return "Mx."

my_salutation = salutation("male", "married", "PhD")
print("Why hello, {} Davies.".format(my_salutation))
print("And hello, {} Davies.".format(salutation("female", "married", "BS")))
\end{Verbatim}
\vspace{-.2in}

\begin{Verbatim}[fontsize=\small,samepage=true,frame=leftline,framesep=5mm,framerule=1mm]
Why hello, Dr. Davies.
And hello, Mrs. Davies.
\end{Verbatim}

Wow, all that code is in \textit{one} function? Yeah. That's not unusual at
all, although you should strive to make functions as compact as they can be.
(The \texttt{salutation()} function \textit{is} as compact as it can be,
actually: there's no way to shorten it without changing what it does.)

\index{return value}
\index{return statement@\texttt{return} statement}

This function, as you can see, has a veritable crap-ton of \texttt{return}
statements -- six in fact. But only one will ever be reached, because as soon
as one \textit{is} reached, the function is officially finished, and returns
that value.

\section{Returning nothing at all}

Does it ever make sense for a function to return \textit{nothing}? Oddly, yes:
if it has a useful side effect. One obvious one is printing something to the
screen:

\begin{Verbatim}[fontsize=\small,samepage=true,frame=single,framesep=3mm]
def cheer_for(team):
    if team != "Christopher Newport":
        print("Go {} go!!".format(team))
    else:
        print("Uhh...no.")

cheer_for("Mary Washington")
cheer_for("Lady Eagles")
cheer_for("Christopher Newport")
\end{Verbatim}
\vspace{-.2in}

\begin{Verbatim}[fontsize=\small,samepage=true,frame=leftline,framesep=5mm,framerule=1mm]
Go Mary Washington go!!
Go Lady Eagles go!!
Uhh...no.
\end{Verbatim}

Here, instead of ``\texttt{something = cheer\_for(...)}'' we type plain-old
``\texttt{cheer\_for(...)}''. That's because there's no return value to
capture, so there's no point in setting it equal to anything. We call the
function just to enjoy its messages.

\section{Calling a function from another function}

Finally, note that we can put any code we desire inside a function's body,
including one or more calls to other functions! Check out this bad boy:

\index{greet@\texttt{greet()}}
\index{full\_name@\texttt{full\_name()}}
\index{salutation@\texttt{salutation()}}
\index{Merkel, Angela}
\index{Sharapova, Maria}
\index{Thunberg, Greta}
\index{Kasparov, Garry}
\index{single}
\index{married}
\index{Ph.D.}
\index{Dr.}
\index{Mr./Mrs./Miss/Ms./Mx.}

\begin{Verbatim}[fontsize=\small,samepage=true,frame=single,framesep=3mm]
def greet(first_n, last_n, middle_i, gender, status, deg, lang):
    sal = salutation(gender, status, deg)
    if lang == "Swedish":
        greeting = "Hej"
    elif lang == "Russian":
        greeting = "Privet"
    elif lang == "Hindi":
        greeting = "Namaste"
    else:
        greeting = "Yo"
    full = full_name(last_n, first_n, middle_i)
    print("{}, {} {}!".format(greeting, sal, full))

greet("Greta", "Thunberg", "F", "female", "single", "none", "Swedish")
greet("Maria", "Sharapova", "S", "female", "single", "none", "Russian")
greet("Garry", "Kasparov", "K", "male", "married", "BA", "Russian")
greet("Angela", "Merkel", "D", "female", "married", "PhD", "German")
\end{Verbatim}
\vspace{-.2in}

\begin{Verbatim}[fontsize=\small,samepage=true,frame=leftline,framesep=5mm,framerule=1mm]
Hej, Miss Greta F. Thunberg!
Privet, Miss Maria S. Sharapova!
Privet, Mr. Garry K. Kasparov!
Yo, Dr. Angela D. Merkel!
\end{Verbatim}

In the course of its duties, \texttt{greet()}'s function body calls both
\texttt{salutation()} and \texttt{full\_name()} for help. They each produce
components of its complete solution. Good teamwork!

% full name, using salutation

