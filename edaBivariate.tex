
\chapter{Exploratory Data Analysis: bivariate data}
\label{ch:edaBivariate}

\index{Exploratory Data Analysis (EDA)}
\index{bivariate}

In this chapter, we'll extend our EDA repertoire to cover \textbf{bivariate
data}, which means studying the relationships between \textit{pairs} of
variables, rather than focusing only on one variable at a time. This is where
most of the action is: you'll be awed and impressed by how much more we can dig
out of a data set in this chapter.

\index{table}

Bivariate data analysis is especially suited to the \textbf{table}s (in Python,
\texttt{DataFrame}s) from section~\ref{tables} and
chapters~\ref{tablesInPython1}--\ref{tablesInPython3}. This is because each
column of a table is a variable that matches one-for-one with every
\textit{other} column in the table.

\index{simpsons@The Simpsons}

In the Simpsons example (p.~\pageref{finalSimpsons}), the fourth
\texttt{species} value corresponds to Lisa, as does the fourth \texttt{age}
value, the fourth \texttt{fave} value, the fourth \texttt{gender} value,
the fourth \texttt{fave} value, the fourth \texttt{IQ} value, the fourth
\texttt{hair} value, and the fourth \texttt{salary} value. This means that if
we examine any two columns, we know that matching indices go together
(\textit{i.e.}, represent the same person). This implicit connection is what
allows us to meaningfully examine a pair of variables.

\section{The concept of statistical significance}

\index{statistical significance}

Before we get to the details, we need to face head-on what is probably the
single most important concept in statistics, that of \textbf{statistical
significance} (or ``stat sig,'' for short). It is so immensely important that
I'm going to ask you to put down whatever snack you're eating right now, fold
your hands, and pay very close attention.

\index{association}
\index{correlation}

All forms of bivariate analysis are variations on a single theme, namely:
discovering whether or not an \textit{association} exists between two
variables. Recall from section~\ref{association} (p.~\pageref{association})
that an association means that two variables are correlated in some way: that
certain values of one tend to go more often with certain values of the other.
To make it concrete, let's say one of our variables is \texttt{sex} (at birth,
male or female) and the other is \texttt{height} (in inches, say). We want to
know: ``are taller people more often male, and shorter people more often
female, or is there no connection between \texttt{sex} and \texttt{height}?''

\index{sample}
\index{mean}
\index{height}
\index{sex}

Now the first thing you think of doing, of course, is getting a \textbf{sample}
(recruiting volunteers, say) of both males and females, measuring their
heights, and taking the average (mean). Let's say you do that, and you come up
with the following numbers:

\begin{center}
Females -- average height: 65.5 inches\\
Males -- average height: 69.3 inches
\end{center}

Clearly, in your \textit{sample}, males were on average somewhat taller -- 3.8
inches taller, in fact. A careless thinker would immediately conclude: ``aha!
My hypothesis is confirmed. I scientifically carried out my study, and
mathematically computed the results, and now here is some hard data proving the
conclusion that generally speaking, men tend to be taller than women.''

Are you convinced by that reasoning?

\index{IQ}

I hope you're not. Here's why. Let's change the example, and suppose that
instead of height, we measured our volunteers' IQ. Taking the averages as
before, we come up with these numbers:

\begin{center}
Females -- average IQ: 102.4\\
Males -- average IQ: 98.6
\end{center}

In this case, the average of the females in the sample was higher than the
males was. Shall we conclude that in general, women tend to be smarter than
men?

\subsection{Confirmation bias}

\index{confirmation bias}

If you're like most people, you'll accept that first finding as confirmation of
men's tallness, and you'll reject the second finding as just a fluke of the
sample. Undoubtedly, this is because you went into the question already having
an opinion about the matter. You just \textit{know in your heart} that men
\textit{do} tend to be taller than women (you've observed thousands of both
sexes, in fact, and have in fact noticed that trend) whereas you know in your
heart that neither sex has an advantage in intelligence (ditto). This leads you
to reason as follows:

\begin{compactenum}
\item ``Well of course my male volunteers were taller than the female ones.
I've known all along that males tend to be taller in general, and this just
confirms it!''
\item ``Aw, c'mon, we only sampled a few people and measured their IQs. Sure,
these particular women might have been a bit smarter than these particular men,
but if I ran the experiment again on different volunteers, it might just as
easily go the other way. It'd be silly to draw a grand conclusion from that.''
\end{compactenum}

Psychologists call this fallacy of reasoning ``\textbf{confirmation bias}.'' We
have a natural tendency to interpret information in a way that affirms our
prior beliefs. Data that seems to contradict it, we simply talk our way out of.

\index{groupthink}

Confirmation bias is one of the most insidious enemies of humankind. It leads
to wrong reasoning, the entrenchment of beliefs, dangerous overconfidence,
polarization, and in the worst cases, \textbf{groupthink}. When a group of
people succumbs to groupthink, ``orthodox'' viewpoints are encouraged, while
alternative viewpoints are dismissed and suppressed. Every piece of evidence
that conforms to the group's consensus belief is hailed as evidence confirming
it, and evidence that contradicts it is chalked up to mere statistical anomaly.

\index{WMD}

One of many examples of this phenomenon was the CIA circa 2002: from the top to
the bottom, nearly every member of the organization was \textit{certain} that
Sadaam Hussein's terrible regime in Iraq possessed weapons of mass destruction
(WMDs). Later, when it was inexplicably discovered that this ``fact'' wasn't
true after all (\textit{after} we had made irreversible decisions based on it),
analysts pored over the CIA's decision-making process to try and make sense of
it. Confirmation bias was perhaps the key ingredient.

Be aware of it in your own thinking, and at all costs steer yourself away from
it!

\subsection{The perils of eyeballing it}

\section{Three bivariate scenarios}

\index{scales of measure}
\index{categorical variable}
\index{nominal variable}
As in chapter~\ref{ch:edaUnivariate}, different kinds of plots and statistics
are appropriate depending on the variable's scale of measure -- categorical or
numeric. There are thus three different cases for bivariate analysis:

\begin{compactitem}
\item Two categorical variables
\item One categorical variable and one numeric variable
\item Two numeric variables
\end{compactitem}

We'll consider each case in turn. Throughout all the remaining sections, we'll
use this fictitious data set, called \texttt{people}:

\begin{Verbatim}[fontsize=\small,samepage=true,frame=leftline,framesep=5mm,framerule=1mm]
   gender  salary   color  followers
0    male   54.94  purple         26
1  female   72.48  purple         22
2    male    9.47    blue         27
3   other   60.08     red         22
4    male   37.62     red         13
                .
                .
                .
\end{Verbatim}

Each row represents one fictional person we interviewed, and includes their 
\texttt{gender}, their \texttt{salary} (in thousands of dollars per year),
their favorite \texttt{color}, and the number of \texttt{followers} they have
on some unspecified social media website.

The \texttt{DataFrame} has 5000 rows, and no special ``index'' variable: none
of the columns that we collected are unique, so we just let Pandas default to
indexing the rows by number, 0 through 4,999.


\section{Two categorical variables}

\index{value\_counts@\texttt{.value\_counts()} method (Pandas)}
\index{bar chart}

Let's begin with a bivariate analysis of the \texttt{gender} and \texttt{color}
columns. The first thing we should do, of course, is inspect each one
individually, using \texttt{.value\_counts()} and perhaps a bar chart from
sections~\ref{categoricalDataValueCounts} and \ref{categoricalDataBarCharts}.
Let's say we've done that.

\index{association}

The next obvious question: is there an \textit{association} between the two
variables? In other words, are there particular values of one that tend to go
with particular values of the other? In still other words, do the genders tend
to have different favorite colors?

\index{contingency table}

\subsection{Contingency tables}

The first tool to get at this question is called a \textbf{contingency table}.
This is very much like \texttt{.value\_counts()}, but for two variables instead
of one. Our function is \texttt{crosstab()} from the Pandas package: if we give
it two columns as arguments, it computes the complete set of counts from all
possible combinations of variables. Here's what it looks like:

\begin{Verbatim}[fontsize=\small,samepage=true,frame=single,framesep=3mm]
pd.crosstab(people.gender, people.color)
\end{Verbatim}
\vspace{-.2in}

\begin{Verbatim}[fontsize=\small,samepage=true,frame=leftline,framesep=5mm,framerule=1mm]
color   blue  green  pink  purple  red  yellow
gender                                        
female   240    402   665     644  289     378
male    1403      0     0     248  463     258
other      1      2     2       2    1       2
\end{Verbatim}

Interpreting this is straightforward. Every cell in the matrix tells us how
many people had a particular gender and a particular favorite color. For
instance, there were 378 females who named yellow as their favorite color, and
no males at all chose green.

% Margins? Or is that 219 only?

\subsection{The $\chi^2$ test}

