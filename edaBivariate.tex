
\chapter{Exploratory Data Analysis: bivariate data}
\label{ch:edaBivariate}

\index{Exploratory Data Analysis (EDA)}
\index{bivariate}

In this chapter, we'll extend our EDA repertoire to cover \textbf{bivariate
data}, which means studying the relationships between \textit{pairs} of
variables, rather than focusing only on one variable at a time. This is where
most of the action is: you'll be awed and impressed by how much more we can dig
out of a data set in this chapter.

\index{table}

Bivariate data analysis is especially suited to the \textbf{table}s (in Python,
\texttt{DataFrame}s) from section~\ref{tables} and
chapters~\ref{tablesInPython1}--\ref{tablesInPython3}. This is because each
column of a table is a variable that matches one-for-one with every
\textit{other} column in the table.

\index{simpsons@The Simpsons}

In the Simpsons example (p.~\pageref{finalSimpsons}), the fourth
\texttt{species} value corresponds to Lisa, as does the fourth \texttt{age}
value, the fourth \texttt{fave} value, the fourth \texttt{gender} value,
the fourth \texttt{fave} value, the fourth \texttt{IQ} value, the fourth
\texttt{hair} value, and the fourth \texttt{salary} value. This means that if
we examine any two columns, we know that matching indices go together
(\textit{i.e.}, represent the same person). This implicit connection is what
allows us to meaningfully examine a pair of variables.

\section{Three bivariate scenarios}

\index{scales of measure}
\index{categorical variable}
\index{nominal variable}

\section{Categorical data: counts of occurrences}

\index{faves@\texttt{faves}}
\index{Swift@Taylor Swift}
\index{Perry@Katy Perry}

