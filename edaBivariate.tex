
\chapter{Exploratory Data Analysis: bivariate data}
\label{ch:edaBivariate}

\index{Exploratory Data Analysis (EDA)}
\index{bivariate}

In this chapter, we'll extend our EDA repertoire to cover \textbf{bivariate
data}, which means studying the relationships between \textit{pairs} of
variables, rather than focusing only on one variable at a time. This is where
most of the action is: you'll be awed and impressed by how much more we can dig
out of a data set in this chapter.

\index{table}

Bivariate data analysis is especially suited to the \textbf{table}s (in Python,
\texttt{DataFrame}s) from section~\ref{tables} and
chapters~\ref{tablesInPython1}--\ref{tablesInPython3}. This is because each
column of a table is a variable that matches one-for-one with every
\textit{other} column in the table.

\index{simpsons@The Simpsons}

In the Simpsons example (p.~\pageref{finalSimpsons}), the fourth
\texttt{species} value corresponds to Lisa, as does the fourth \texttt{age}
value, the fourth \texttt{fave} value, the fourth \texttt{gender} value,
the fourth \texttt{fave} value, the fourth \texttt{IQ} value, the fourth
\texttt{hair} value, and the fourth \texttt{salary} value. This means that if
we examine any two columns, we know that matching indices go together
(\textit{i.e.}, represent the same person). This implicit connection is what
allows us to meaningfully examine a pair of variables.

\section{The concept of statistical significance}

\index{statistical significance}

Before we get to the details, we need to face head-on what is probably the
single most important concept in statistics, that of \textbf{statistical
significance} (or ``stat sig,'' for short). It is so immensely important that
I'm going to ask you to put down whatever snack you're eating right now, fold
your hands, and pay very close attention.

\index{association}
\index{correlation}

All forms of bivariate analysis are variations on a single theme, namely:
discovering whether or not an \textit{association} exists between two
variables. Recall from section~\ref{association} (p.~\pageref{association})
that an association means that two variables are correlated in some way: that
certain values of one tend to go more often with certain values of the other.
To make it concrete, let's say one of our variables is \texttt{sex} (at birth,
male or female) and the other is \texttt{height} (in inches, say). We want to
know: ``are taller people more often male, and shorter people more often
female, or is there no connection between \texttt{sex} and \texttt{height}?''

\index{sample}
\index{mean}
\index{height}
\index{sex}

Now the first thing you think of doing, of course, is getting a \textbf{sample}
(recruiting volunteers, say) of both males and females, measuring their
heights, and taking the average (mean). Let's say you do that, and you come up
with the following numbers:

\begin{center}
Females -- average height: 65.5 inches\\
Males -- average height: 69.3 inches
\end{center}

Clearly, in your \textit{sample}, males were on average somewhat taller -- 3.8
inches taller, in fact. A careless thinker would immediately conclude: ``aha!
My hypothesis is confirmed. I scientifically carried out my study, and
mathematically computed the results, and now here is some hard data proving the
conclusion that generally speaking, men tend to be taller than women.''

Are you convinced by that reasoning?

\index{IQ}

I hope you're not. Here's why. Let's change the example, and suppose that
instead of height, we measured our volunteers' IQ. Taking the averages as
before, we come up with these numbers:

\begin{center}
Females -- average IQ: 102.4\\
Males -- average IQ: 98.6
\end{center}

In this case, the average of the females in the sample was higher than the
males was. Shall we conclude that in general, women tend to be smarter than
men?

\subsection{Confirmation bias}

\index{confirmation bias}

If you're like most people, you'll accept that first finding as confirmation of
men's tallness, and you'll reject the second finding as just a fluke of the
sample. Undoubtedly, this is because you went into the question already having
an opinion about the matter. You just \textit{know in your heart} that men
\textit{do} tend to be taller than women (you've observed thousands of both
sexes, in fact, and have in fact noticed that trend) whereas you know in your
heart that neither sex has an advantage in intelligence (ditto). This leads you
to reason as follows:

\begin{compactenum}
\item ``Well of course my male volunteers were taller than the female ones.
I've known all along that males tend to be taller in general, and this just
confirms it!''
\item ``Aw, c'mon, we only sampled a few people and measured their IQs. Sure,
these particular women might have been a bit smarter than these particular men,
but if I ran the experiment again on different volunteers, it might just as
easily go the other way. It'd be silly to draw a grand conclusion from that.''
\end{compactenum}

Psychologists call this fallacy of reasoning ``\textbf{confirmation bias}.'' We
have a natural tendency to interpret information in a way that affirms our
prior beliefs. Data that seems to contradict it, we simply talk our way out of.

\index{groupthink}

Confirmation bias is one of the most insidious enemies of humankind. It leads
to wrong reasoning, the entrenchment of beliefs, dangerous overconfidence,
polarization, and in the worst cases, \textbf{groupthink}. When a group of
people succumbs to groupthink, ``orthodox'' viewpoints are encouraged, while
alternative viewpoints are dismissed and suppressed. Every piece of evidence
that conforms to the group's consensus belief is hailed as evidence confirming
it, and evidence that contradicts it is chalked up to mere statistical anomaly.

\index{WMD}

One of many examples of this phenomenon was the CIA circa 2002: from the top to
the bottom, nearly every member of the organization was \textit{certain} that
Sadaam Hussein's terrible regime in Iraq possessed weapons of mass destruction
(WMDs). Later, when it was inexplicably discovered that this ``fact'' wasn't
true after all (\textit{after} we had made irreversible decisions based on it),
analysts pored over the CIA's decision-making process to try and make sense of
it. Confirmation bias was perhaps the key ingredient.

Be aware of it in your own thinking, and at all costs steer yourself away from
it!

\subsection{The perils of eyeballing it}

Okay. Let's suppose that we've freed ourselves from confirmation bias, and
we're actually looking at the numbers objectively. The first question still
remains: \textit{does} an association exist between the two variables?

Men were an average of 3.8 inches taller in our sample. That's a
difference...but is it \textit{enough} of a difference? Women were smarter by
3.8 IQ points. That's a difference...but is it \textit{enough}?

\index{enough@``enough''}
\index{sample}
\index{population}

To clarify, when we say ``enough,'' what we mean is: ``\textit{enough of a
difference to generalize our findings to the population at large.}'' Here's a
paradox: what we have is a sample; yet oddly, it's never the sample we actually
care about. We care about what the sample tells us \textit{about the
\textbf{population}}.

\index{Warren, Elizabeth}
\index{Trump, Donald}

Think about it. Nobody cares whether the 14 females we surveyed are smarter on
average than the 17 males we surveyed. But if we make the claim: ``\textit{in
general} women are smarter than men,'' suddenly everybody cares. To use another
example, nobody cares that 58\% of the 2,000 people in our phone sample said
they intend to vote for Elizabeth Warren, and only 39\% said Donald Trump. But
if we say ``\textit{across the country}, we predict more Warren votes than
Trump votes by such-and-such margin,'' this is big news.

\index{eyeballing it}

Now the first law to beat into your head is that \textit{you absolutely cannot
reliably eyeball it.} This is what everyone who hasn't taken a Data Science or
Statistics class tries to do. They squint at the difference (3.8 IQ points,
\textit{e.g.}) and bite their lip and mutter, ``well, that sure \textit{seems}
(or \textit{doesn't seem}) like a pretty big difference. I'll bet this says (or
doesn't say) something about intelligence among the sexes in general.''

Stop. You cannot. People are demonstrably very bad at judging whether or not a
difference between groups is ``enough.'' Part of the problem is that the answer
to the question turns on three separate things: how big the difference is, how
large your sample size is, and -- importantly -- how variable the data is
(meaning, how widely the points you sample differ from each other). All three
of these factors need to be mixed into a soup in just the right way in order to
properly judge, and human intuition is just flat terrible at doing that.

So eyeballing is a non-starter. But happily, it turns out that statistics
provides us an iron-clad, dependable, quantitative, take-it-to-the-bank method
for determining whether the pattern in a data set is ``enough'' to justifiably
claim an association between variables. And that is the concept of statistical
significance.

\subsection{A ``statistically significant difference''}

The correct way to determine whether a difference is ``enough'' is to use the
appropriate \textbf{statistical test}. A statistical test is a standard
procedure for incorporating all three factors I previously mentioned (the
degree of difference, the sample size, and the amount of variance among data
points) to come up with a principled, defensible answer to this question: is
the pattern I see in my sample a \textit{statistically significant} one? Can we
be reasonably confident that it will also be true of the population as a whole?

\index{statistical test}
\index{p-value@\textit{p}-value}

All the statistical tests we'll learn (in the next chapter) have a common
output: a \textbf{\textit{p}-value}. That's nice because you don't have to
memorize a lot of different rules for interpreting a lot of different tests.

So what the heck is a ``\textit{p}-value?'' There have been controversies
galore\footnote{For instance:\\
\vspace{-.15in}
\begin{compactitem}
\item Colquhoun D (2017). ``$p$-values.'' \textit{Royal Society Open Science.}
\textbf{4}(12): 171085.
\item Murtaugh, Paul A. (2014). ``In defense of $p$-values.'' \textit{Ecology}.
\textbf{95}(3): 611--617
\item Wasserstein, Ronald L.; Lazar, Nicole A. (2016). ``The ASA's Statement on
$p$-Values: Context, Process, and Purpose.'' \textit{The American Statistician.}
\textbf{70}(2): 129-133
\end{compactitem}
\vspace{-.15in}
}, and even entire
books\footnote{Vickers, A. J. (2009). \textit{What is a p-value anyway?}
Boston: Pearson.} devoted to the subject, which means that whatever I choose to
write here can be nitpicked by statisticians a dozen different ways.

That's okay. I'm going to write it anyway, and this will serve you very well. 
Here goes: 

\index{$\alpha$ (alpha)}
\index{alpha@alpha ($\alpha$)}

\definecolor{shadecolor}{rgb}{.9,.9,.9}
\begin{shaded}
A \textit{p}-value is a number between 0 and 1. If you run a
statistical test on your bivariate data, and the \textit{p}-value is
\textbf{\textit{less than}} your $\alpha$ (``alpha'') value (normally, .05),
then there \textit{\textbf{is}} a statistically significant association between
the variables. Otherwise, there isn't. End of story.
\end{shaded}

Recall from section~\ref{alpha} (p.~\pageref{alpha}) that $\alpha$ is ``where
to set the bar'' to detect a meaningful association. It's essentially how often
we're willing to draw a false conclusion. For social science data (that is,
data involving humans), you should always choose .05 to avoid controversy. For
physical science data, you should always choose .01. 

The bottom line is this: if you spot a possible relationship between two of
your variables (like gender and IQ), run the appropriate statistical test (see
next chapter) and look at the $p$-value. If it's less than $\alpha$, then the
difference you thought you saw officially \textit{is} ``enough.'' You can
therefore declare ``yep, these two variables \textit{are} associated, to a
confidence level of $\alpha$.'' If it's not less than $\alpha$, then even
though you thought you saw a meaningful tendency in the data, you can
officially say, ``nope, it's not a stat sig diff. This is very likely just an
artifact of the particular data points I collected in my sample. Pay of no
mind.''

\section{Moving on}

Which statistical test is appropriate depends on your two variables' scales of
measure: in particular, whether they are categorical or numeric. There are
three scenarios for bivariate analysis: two categorical variables, two numeric
variables, or one of each. In the next chapter, in addition to learning how to
meaningfully plot all three cases, we'll learn how to run and interpret the
statistical test applicable to each case, in order to determine once and for
all whether ``enough'' is \textit{enough}.
