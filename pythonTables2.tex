
\chapter[Tables in Python (2 of 3)]{\huge\selectfont{Tables in Python (2 of
3)}}

\label{tablesInPython2}
\index{row (of a \texttt{DataFrame})}
\index{row (of a table)}
\index{column (of a table)}

It's easy to get tripped up on Pandas' syntax for accessing the individual bits
of \texttt{DataFrame}s. First, let's talk about rows and columns, and then
we'll talk about the atomic elements (``cells'') themselves.

\section{Accessing individual rows and columns}

\label{accessIndividualRowsColsOfDataFrame}

Suppose you have a \texttt{DataFrame} called \texttt{df}. Here's how you can
extract particular rows and columns:

\begin{compactitem}
\item \texttt{df.loc[}\textsl{i}\texttt{]} -- access the \textbf{row} with
\textbf{index} \textsl{i}
\item \texttt{df.iloc[}\textsl{n}\texttt{]} -- access \textbf{row} number
\textsl{n}
\item \texttt{df[}\textsl{c}\texttt{]} -- access \textbf{column} \textsl{c}
\end{compactitem}

\index{iloc@\texttt{.iloc} syntax (Pandas)}
\index{loc@\texttt{.loc} syntax (Pandas)}

The second of these is reminiscent of the \texttt{.iloc} syntax we learned for
\texttt{Series}es on p.~\pageref{iloc}. With it, we specify the \textit{number}
we want, rather than the index/key/label. That's not super common to do, but it
happens. More common is the first form: we specify the row we want by its
index.

The last one is tricky, because everyone (including me, several times a week,
it seems) assumes that just typing (say) ``\texttt{df['Bart']}'' would give you
\texttt{Bart}'s row. This is probably how it \textit{ought} to work, since
\texttt{Series}es worked that way. Alas, no: if you specify neither
\texttt{.loc} nor \texttt{.iloc}, you're asking for a \textit{column}, not a
row.

% WARNING: "bottom of" might not be correct if we change ch.16 text.

Yet another odd thing is how a single row is presented on the screen. Let's go
back to the \texttt{simpsons} data set (bottom of p.~\pageref{finalSimpsons}),
and access the \texttt{Bart} row the proper way (with \texttt{.loc}):

\begin{Verbatim}[fontsize=\small,samepage=true,frame=single,framesep=3mm]
print(simpsons.loc['Bart'])
\end{Verbatim}
\vspace{-.2in}

\begin{Verbatim}[fontsize=\small,samepage=true,frame=leftline,framesep=5mm,framerule=1mm]
species         human
age                10
gender              M
fave       skateboard
IQ                 90
hair             buzz
salary              0
Name: Bart, dtype: object
\end{Verbatim}

This bugs the heck out of me. Bart, like all other Simpsons, was a \textit{row}
in the original \texttt{DataFrame}, but here, it presents Bart's information
vertically instead of horizontally. I find it visually jarring. The reason
Pandas does it this way is that \textit{each row of a \texttt{DataFrame} is a
\texttt{Series}}, and the way Pandas displays \texttt{Series}es is vertically.
We'll deal somehow.

\index{double boxies@``double boxies''}

Btw, for any of the three options, you can provide a \textit{list} with
multiple things you want, instead of just one thing. You do so by using
\textit{double} boxies:

\begin{compactitem}
\item \texttt{df.loc[[}\textsl{i1,i2,i3,\dots}\texttt{]]} -- access the
row\textbf{s} with indices \textsl{i1, i2, i3,} \textit{etc.}
\item \texttt{df.iloc[[}\textsl{n1,n2,n3,\dots}\texttt{]]} -- access the row\textbf{s}
numbered \textsl{n1, n2, n3,} \textit{etc.}
\item \texttt{df[[}\textsl{c1,c2,c3,\dots}\texttt{]]} -- access the
column\textbf{s} names \textsl{c1, c2, c3,} \textit{etc.}
\end{compactitem}

\subsection{Examples}

To test your understanding of all of the above, confirm that you understand the
following examples:

\begin{Verbatim}[fontsize=\scriptsize,samepage=true,frame=single,framesep=3mm]
print(simpsons.iloc[3])
\end{Verbatim}
\vspace{-.2in}

\begin{Verbatim}[fontsize=\scriptsize,samepage=true,frame=leftline,framesep=5mm,framerule=1mm]
species        human
age                8
gender             F
fave       saxophone
IQ               200
hair           curly
salary             0
Name: Lisa, dtype: object
\end{Verbatim}

\medskip

\begin{Verbatim}[fontsize=\scriptsize,samepage=true,frame=single,framesep=3mm]
print(simpsons['age'])
\end{Verbatim}
\vspace{-.2in}

\begin{Verbatim}[fontsize=\scriptsize,samepage=true,frame=leftline,framesep=5mm,framerule=1mm]
name
Homer     36
Marge     34
Bart      10
Lisa       8
Maggie     1
SLH        4
Name: age, dtype: int64
\end{Verbatim}

\medskip
\begin{Verbatim}[fontsize=\scriptsize,samepage=true,frame=single,framesep=3mm]
print(simpsons.loc[['Lisa','Maggie','Bart']])
\end{Verbatim}
\vspace{-.2in}

\begin{Verbatim}[fontsize=\scriptsize,samepage=true,frame=leftline,framesep=5mm,framerule=1mm]
       species  age gender        fave     IQ   hair  salary
name                                                        
Lisa     human    8      F   saxophone  200.0  curly     0.0
Maggie   human    1      F    pacifier  100.0  curly     0.0
Bart     human   10      M  skateboard   90.0   buzz     0.0
\end{Verbatim}

\medskip
\begin{Verbatim}[fontsize=\scriptsize,samepage=true,frame=single,framesep=3mm]
print(simpsons.iloc[[1,3,4]])
\end{Verbatim}
\vspace{-.2in}

\begin{Verbatim}[fontsize=\scriptsize,samepage=true,frame=leftline,framesep=5mm,framerule=1mm]
       species  age gender            fave     IQ          hair  salary
name                                                                   
Marge    human   34      F  helping others  120.0  stacked tall     0.0
Lisa     human    8      F       saxophone  200.0         curly     0.0
Maggie   human    1      F        pacifier  100.0         curly     0.0
\end{Verbatim}

\medskip
\begin{Verbatim}[fontsize=\scriptsize,samepage=true,frame=single,framesep=3mm]
print(simpsons[['age','fave','IQ']])
\end{Verbatim}
\vspace{-.2in}

\begin{Verbatim}[fontsize=\scriptsize,samepage=true,frame=leftline,framesep=5mm,framerule=1mm]
        age            fave     IQ
name                              
Homer    36            beer   74.0
Marge    34  helping others  120.0
Bart     10      skateboard   90.0
Lisa      8       saxophone  200.0
Maggie    1        pacifier  100.0
SLH       4             NaN   30.0
\end{Verbatim}

Incidentally, you'll notice how the \texttt{name} values are treated
differently from all the other columns, since \texttt{name} is the
\texttt{DataFrame}'s index. For one thing, \texttt{name} \textit{always}
appears, even though it's not included among the columns we asked for. For
another, it's listed at the bottom of the single-row \texttt{Series} listings
rather than up with the other values in that row.

\section{Accessing individual elements}

\label{accessIndividualElementsOfDataFrame}

I mentioned above the eternal truth that \textit{each row of a
\texttt{DataFrame} is a \texttt{Series}}. Once you grasp this, you'll realize
that you can access an individual ``cell'' of a \texttt{DataFrame} simply by
getting the row you want, and then getting the specific value from that. A
two-step process for doing this would be:

\begin{Verbatim}[fontsize=\small,samepage=true,frame=single,framesep=3mm]
lisas_row = simpsons.loc['Lisa']
lisas_iq = lisas_row['IQ']
print(lisas_iq)
\end{Verbatim}
\vspace{-.2in}

\begin{Verbatim}[fontsize=\small,samepage=true,frame=leftline,framesep=5mm,framerule=1mm]
200.0
\end{Verbatim}

But a shorter, one-stepper just combines these two operations on the same line:

\begin{Verbatim}[fontsize=\small,samepage=true,frame=single,framesep=3mm]
lisas_iq = simpsons.loc['Lisa']['IQ']
print(lisas_iq)
\end{Verbatim}
\vspace{-.2in}

\begin{Verbatim}[fontsize=\small,samepage=true,frame=leftline,framesep=5mm,framerule=1mm]
200.0
\end{Verbatim}

\section{Accessing a \texttt{DataFrame}'s metadata}

\index{metadata}
\index{index@\texttt{.index} (little \texttt{i}) syntax}

We can get some meta-information about a \texttt{DataFrame} without even
looking at individual rows. If we want to know what the index values themselves
are, we use \texttt{.index}:

\begin{Verbatim}[fontsize=\small,samepage=true,frame=single,framesep=3mm]
print(simpsons.index)
\end{Verbatim}
\vspace{-.2in}

\begin{Verbatim}[fontsize=\small,samepage=true,frame=leftline,framesep=5mm,framerule=1mm]
Index(['Homer', 'Marge', 'Bart', 'Lisa', 'Maggie', 'SLH'],
    dtype='object', name='name')
\end{Verbatim}

That weird-looking output tells us several things. First, the index of this
\texttt{DataFrame} consists of \texttt{string}s (remember from
p.~\pageref{dtypeRules} that's what ``\texttt{dtype=\textquotesingle
object\textquotesingle}'' means). Second,
the \textit{name} of the index column is, ironically, ``\texttt{name}''. (It
could be named anything at all, of course.) Third, the actual index values are
\texttt{Homer}, \texttt{Marge}, and all the rest.

\index{columns@\texttt{.columns} (Pandas)}

That's the index, or the ``row names,'' if you will. To get the column names,
we use \texttt{.columns}:

\begin{Verbatim}[fontsize=\small,samepage=true,frame=single,framesep=3mm]
print(simpsons.columns)
\end{Verbatim}
\vspace{-.2in}

\begin{Verbatim}[fontsize=\small,samepage=true,frame=leftline,framesep=5mm,framerule=1mm]
Index(['species', 'age', 'gender', 'fave', 'IQ', 'hair',
    'salary'], dtype='object')
\end{Verbatim}

Interestingly, this too is an ``\texttt{Index}'' beast, also comprised of
\texttt{string}s. Pandas treats both ``axes'' of a \texttt{DataFrame}
similarly, in that both of them are the same type of thing (an
``\texttt{Index}''). Notice that \texttt{name} is not present in the column
names list, because as the \texttt{DataFrame}'s index it's a different sort of
thing.

\index{len@\texttt{len()}}

How many rows does a \texttt{DataFrame} have? This is answerable by using the
\texttt{len()} function again:

\begin{Verbatim}[fontsize=\small,samepage=true,frame=single,framesep=3mm]
print(len(simpsons))
\end{Verbatim}
\vspace{-.2in}

\begin{Verbatim}[fontsize=\small,samepage=true,frame=leftline,framesep=5mm,framerule=1mm]
6
\end{Verbatim}

This is our third use of the word \texttt{len()}: it can be used to find the
number of characters in a \texttt{string}, the number of key/value pairs of a
\texttt{Series}, and (here) the number of rows of a \texttt{DataFrame}.

\index{shape@\texttt{.shape} (Pandas)}
Finally, we often want to get a quick sense of how large a \texttt{DataFrame}
is, both in terms of rows and columns. The \texttt{.shape} syntax is handy
here:

\begin{samepage}
\begin{Verbatim}[fontsize=\small,samepage=true,frame=single,framesep=3mm]
print(simpsons.shape)
\end{Verbatim}
\vspace{-.2in}

\begin{Verbatim}[fontsize=\small,samepage=true,frame=leftline,framesep=5mm,framerule=1mm]
(6, 7)
\end{Verbatim}
\end{samepage}

This tells us that \texttt{simpsons} has six rows and seven columns. As I
mentioned previously (p.~\pageref{tallAndSkinny}) this is definitely not the
typical case: most \texttt{DataFrames} will have many more rows (thousands or
even millions) than columns (at most, dozens).

\section{Sorting \texttt{DataFrame}s}
\index{sorting@sorting (\texttt{DataFrame}s)}

Sorting a \texttt{DataFrame} is largely like sorting a \texttt{Series}, except
we have more choices: instead of just the keys and the values, we have the
index and potentially \textit{many} different columns.

\index{sort\_index@\texttt{.sort\_index()} method (Pandas)}
The \texttt{.sort\_index()} method works just like it did for
\texttt{Series}es:

\begin{samepage}
\begin{Verbatim}[fontsize=\small,samepage=true,frame=single,framesep=3mm]
print(simpsons.sort_index())
\end{Verbatim}
\vspace{-.2in}

\begin{Verbatim}[fontsize=\scriptsize,samepage=true,frame=leftline,framesep=5mm,framerule=1mm]
       species  age gender            fave     IQ          hair   salary
name                                                                    
Bart     human   10      M      skateboard   90.0          buzz      0.0
Homer    human   36      M            beer   74.0          none  52000.0
Lisa     human    8      F       saxophone  200.0         curly      0.0
Maggie   human    1      F        pacifier  100.0         curly      0.0
Marge    human   34      F  helping others  120.0  stacked tall      0.0
SLH        dog    4      M             NaN   30.0        shaggy      0.0
\end{Verbatim}
\end{samepage}

The result is rows sorted alphabetically by name. And I hate to keep repeating
myself, but remember that \texttt{.sort\_index()} returns a modified copy,
unless you pass the \texttt{inplace=True} argument. The
\texttt{ascending=False} argument is also allowed, and will sort by the index
highest-to-lowest instead of lowest-to-highest.

\index{sort\_values@\texttt{.sort\_values()} method (Pandas)}
To sort by one of the columns, we call \texttt{.sort\_values()} and pass it the
column name:

\begin{Verbatim}[fontsize=\small,samepage=true,frame=single,framesep=3mm]
print(simpsons.sort_index('IQ')
\end{Verbatim}
\vspace{-.2in}

\begin{Verbatim}[fontsize=\scriptsize,samepage=true,frame=leftline,framesep=5mm,framerule=1mm]
       species  age gender            fave     IQ          hair   salary
name                                                                    
SLH        dog    4      M             NaN   30.0        shaggy      0.0
Homer    human   36      M            beer   74.0          none  52000.0
Bart     human   10      M      skateboard   90.0          buzz      0.0
Maggie   human    1      F        pacifier  100.0         curly      0.0
Marge    human   34      F  helping others  120.0  stacked tall      0.0
Lisa     human    8      F       saxophone  200.0         curly      0.0
\end{Verbatim}

\index{tie-breaker}

Sometimes we want to include more than one column in the sort. Why? As a
tie-breaker. Consider sorting a roster for a student club, which has
\texttt{first\_name} and \texttt{last\_name} columns, among other things. We
might want to sort the list alphabetically by last name, but for students with
the same last name, we should go to the first name as a tie-breaker (so that
\texttt{Angela Smith} shows up after \texttt{Velma Patterson} but before
\texttt{Brad Smith}).

To do this, we pass a list of columns, instead of a single column:

\begin{samepage}
\begin{Verbatim}[fontsize=\small,samepage=true,frame=single,framesep=3mm]
print(simpsons.sort_values(['gender','hair','IQ']))
\end{Verbatim}
\vspace{-.2in}

\begin{Verbatim}[fontsize=\scriptsize,samepage=true,frame=leftline,framesep=5mm,framerule=1mm]
       species  age gender            fave     IQ          hair   salary
name                                                                    
Maggie   human    1      F        pacifier  100.0         curly      0.0
Lisa     human    8      F       saxophone  200.0         curly      0.0
Marge    human   34      F  helping others  120.0  stacked tall      0.0
Bart     human   10      M      skateboard   90.0          buzz      0.0
Homer    human   36      M            beer   74.0          none  52000.0
SLH        dog    4      M             NaN   30.0        shaggy      0.0
\end{Verbatim}
\end{samepage}

Here, we said ``sort the rows alphabetically by \texttt{gender}. For rows with
the same \texttt{gender}, use \texttt{hair} as a tie-breaker. And for rows with
the same \texttt{gender} \textit{and} the same \texttt{hair}, use \texttt{IQ}
as a second tie-breaker.'' Glance at that output and convince yourself that
it's correct.

We control the ``ascendingness'' of the multi-column sort by specifying a list
of \textit{each} \texttt{ascending} value, one for each column we're sorting
by. Consider this:

\begin{samepage}
\begin{Verbatim}[fontsize=\small,samepage=true,frame=single,framesep=3mm]
print(simpsons.sort_values(['gender','hair','IQ'],
    ascending=[False,True,False]))
\end{Verbatim}
\vspace{-.2in}

\begin{Verbatim}[fontsize=\scriptsize,samepage=true,frame=leftline,framesep=5mm,framerule=1mm]
       species  age gender            fave     IQ          hair   salary
name                                                                    
Bart     human   10      M      skateboard   90.0          buzz      0.0
Homer    human   36      M            beer   74.0          none  52000.0
SLH        dog    4      M             NaN   30.0        shaggy      0.0
Lisa     human    8      F       saxophone  200.0         curly      0.0
Maggie   human    1      F        pacifier  100.0         curly      0.0
Marge    human   34      F  helping others  120.0  stacked tall      0.0
\end{Verbatim}
\end{samepage}

Now we're saying ``sort \textit{reverse} alphabetically by \texttt{gender},
breaking ties by comparing \texttt{hair} alphabetically, and breaking further
ties by \texttt{reverse} sorted order by IQ.''

Oh, and the \texttt{inplace=True} argument works for all these examples as
well.

\section{Summary statistics for \texttt{DataFrame}s}

\index{summary statistics}

Summary statistics like the mean, median, minimum/maximum, and the like, can of
course all be computed on individual columns of a \texttt{DataFrame}, because
each column is a \texttt{Series}:

\index{median}

\begin{samepage}
\begin{Verbatim}[fontsize=\small,samepage=true,frame=single,framesep=3mm]
print(simpsons['IQ'].median())
\end{Verbatim}
\vspace{-.4in}

\begin{Verbatim}[fontsize=\small,samepage=true,frame=leftline,framesep=5mm,framerule=1mm]
95.0
\end{Verbatim}
\end{samepage}

\begin{samepage}
\begin{Verbatim}[fontsize=\small,samepage=true,frame=single,framesep=3mm]
print(simpsons['salary'].sum())
\end{Verbatim}
\vspace{-.4in}

\index{sum method@\texttt{.sum()} method (Pandas)}
\begin{Verbatim}[fontsize=\small,samepage=true,frame=leftline,framesep=5mm,framerule=1mm]
52000.0
\end{Verbatim}
\end{samepage}

\index{mean method@\texttt{.mean()} method (Pandas)}
You can also, believe it or not, compute the sum/mean/max/etc on the entire
\texttt{DataFrame}. This computes it on every column individually:

\begin{samepage}
\begin{Verbatim}[fontsize=\small,samepage=true,frame=single,framesep=3mm]
print(simpsons.mean())
\end{Verbatim}
\vspace{-.4in}

\begin{Verbatim}[fontsize=\small,samepage=true,frame=leftline,framesep=5mm,framerule=1mm]
age         15.500000
IQ         102.333333
salary    8666.666667
dtype: float64
\end{Verbatim}
\end{samepage}


Pandas left out the non-numeric columns (\texttt{species}, \texttt{gender},
\textit{etc.}) and computed the mean of each of the others, giving us a
\texttt{Series} containing their values.

\index{describe method@\texttt{.describe()} method (Pandas)}

Finally, I often find the \texttt{.describe()} method useful:

\begin{samepage}
\begin{Verbatim}[fontsize=\small,samepage=true,frame=single,framesep=3mm]
print(simpsons.describe())
\end{Verbatim}
\vspace{-.2in}

\begin{Verbatim}[fontsize=\small,samepage=true,frame=leftline,framesep=5mm,framerule=1mm]
count   6.000000    6.000000      6.000000
mean   15.500000  102.333333   8666.666667
std    15.436969   56.645094  21228.911104
min     1.000000   30.000000      0.000000
25%     5.000000   78.000000      0.000000
50%     9.000000   95.000000      0.000000
75%    28.000000  115.000000      0.000000
max    36.000000  200.000000  52000.000000
\end{Verbatim}
\end{samepage}

\index{quartile}
\index{standard deviation}

Neat! We get the number of values, the mean, the standard deviation, and all
the quartiles for each of the numeric columns. Lots of dashboard information at
a glance!

