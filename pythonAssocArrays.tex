
% Somewhere: .map() for recoding

\chapter{Associative arrays in Python (1 of 2)}
\label{ch:pythonAssocArrays}

\index{associative array}
\index{Pandas package}
Our next trick is to represent associative arrays (review
section~\ref{sec:assocArrays} if you need to) in Python. To do so, we will
use another package, which goes by the adorable name ``Pandas'':

\begin{Verbatim}[fontsize=\small,samepage=true,frame=single,framesep=3mm]
import pandas as pd
\end{Verbatim}

This code should go at the top of your first notebook cell, right under your
``\texttt{import numpy as np}'' line. The two go hand in hand.

\index{table}
\index{NumPy package}
\index{series@\texttt{Series} (Pandas)}
\index{dict@\texttt{dict} (dictionary)}
By the way, just as there were other choices besides NumPy \texttt{ndarray}s to
represent ordinary arrays, there are other choices in Python for associative
arrays. The native Python \texttt{dict} (``dictionary'') type is an obvious
candidate. Because this won't work well when the data gets huge, however, and
because using Pandas here will set up our usage of tables nicely in the next
few chapters, we're going to use the Pandas \textbf{Series} data type for our
associative arrays.


\section{Pandas \texttt{Series}}
