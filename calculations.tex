
% TODO:
%   .strip() string method
% memory diagrams -- diff between modifying in place and returning a copy (like
% .strip().)

\chapter{Calculations}

Our discipline obviously involves a lot of computation -- in fact, I expect the
first image that comes to mind when most people hear the words ``data science''
is one of numerical calculation. In this chapter, I'll lay out the Python
syntax for performing various mathematical operations on numbers, as well as
manipulating strings. These things appear in every program, and you'll find it
all straightforward.

And then I'll drop a bomb on you. I'll unveil a Python behavior which you'll
probably find completely unexpected, which flummoxes nearly every student who
first sees it, and yet which you must understand and master to succeed in
Python or any programming language.

\section{Mathematical operations}

\index{operator}


\index{bananas (parentheses)}
\index{()@\texttt{()} (bananas)}
\index{curlies (curly braces)}
\index{\{\}@\texttt{\{\}} (curlies)}
\index{boxies (square brackets)}
\index{[]@\texttt{[]} (boxies)}
\index{wakkas (angle brackets)}
\index{<>@\texttt{<>} (wakkas)}
\index{+@\texttt{+}}
\index{-@\texttt{-}}
\index{/@\texttt{/}}
\index{**@\texttt{**}}
First, the easy stuff. Python has a number of built-in \textbf{operators} to do
the familiar math stuff. Figure~\ref{fig:mathOps} has a table of the ones we'll
use. A few are mildly surprising (\texttt{*} instead of \texttt{X} for
multiplication; \texttt{/} instead of $\div$ for division, which I'll bet you
couldn't find on your keyboard anyway), and you have to remember to use only
bananas (not boxies \texttt{[]}, curlies \texttt{\{\}}, or wakkas \texttt{<>})
for grouping sub-expressions within a larger expression. Otherwise, it's a
piece of cake.

\begin{figure}[ht]
\centering
\begin{tabular}{c | l}
\hline
Operator & Operation \\
\hline
\texttt{+} & addition \\
\texttt{-} & subtraction \\
\texttt{*} & multiplication \\
\texttt{/} & division \\
\texttt{**} & exponentiation \\
\texttt{()} & grouping \\
\hline
\end{tabular}
\smallskip
\caption{Python's basic math operators.}
\label{fig:mathOps}
\end{figure}

All this stuff has to appear on the \textbf{right-hand side} of an equals sign,
by the way, never on the left. That may seem surprising, since in mathematics
the equations ``$x = y + 3$'' and ``$y + 3 = x$'' mean the same thing. Why does
it matter which order you write it in? The answer, you'll recall, is that in a
program the symbol ``\texttt{=}'' doesn't mean ``\textit{is} equal to'' but
rather ``\textit{make} equal to.'' It's not an equation; it's a command. And
you can't command ``$y+3$'' to be equal to anything. Therefore the only thing
permitted on the left-hand side of an equals sign is a single, plain-jane
variable name.

To test your understanding of the syntax, see if you agree that the following
math expression:

\begin{align*}
\textrm{gpa} = \frac{
\textrm{creds}_1 \cdot \textrm{gradepts}_1 +
\textrm{creds}_2 \cdot \textrm{gradepts}_2}
{\textrm{creds}_1 + \textrm{creds}_2}
\end{align*}

should look like this in Python:

\begin{Verbatim}[fontsize=\small,samepage=true,frame=single,framesep=3mm]
gpa = (creds1 * gradepts1 + creds2 * gradepts2) / (creds1 + creds2)
\end{Verbatim}

and that this one:

\begin{align*}
a = \frac{\lbrack x^2y(4-z) + (x+q)\cdot y \rbrack \times 2^{15y+2z}}
{19x^3 - (2yz)^{(y-1)^2}}
\end{align*}

should look like this:

\begin{Verbatim}[fontsize=\scriptsize,samepage=true,frame=single,framesep=3mm]
a = ( ((x**2)*y*(4-z) + (x+q)*y) * 2**(15*y+2*z) ) / ( 19*(x**3) - (2*y*z)**((y-1)**2) )
\end{Verbatim}

If so, you're good to go. It's tedious, but not complicated.

Python also has plenty of functions for absolute value, sin and cosine,
logarithms, square roots, and anything else you can think of. We'll learn all
those at the proper time, or they're all eminently Google-able if you want to
look them up now.


\section{Return values}

\index{calling a function@``calling'' a function}
\index{passing an argument@``passing'' an argument}
First, we're going to add another phrase to our already lengthy
function-calling mantra. You'll recall that we summarized this code (a function
call):

\begin{Verbatim}[fontsize=\small,samepage=true,frame=single,framesep=3mm]
len(movie_title)
\end{Verbatim}

with this English:

\begin{quote}
``We are calling the \texttt{len()} function, and passing it
\texttt{movie\_title} as an argument.''
\end{quote}

\index{calling a method@``calling'' a method (on a variable)}
\index{on@``on''}
And we summarized this code (a method call):

\begin{Verbatim}[fontsize=\small,samepage=true,frame=single,framesep=3mm]
message.format(name, age)
\end{Verbatim}

with this English:

\begin{quote}
``We are calling the \texttt{.format()} method \textbf{on} the \texttt{message}
variable, and passing it \texttt{name} and \texttt{age} as arguments.''
\end{quote}

