
\chapter{Scales of measure}

In the last chapter, we learned the Python verbiage for how to do arithmetic
operations. In this one, we zoom out and ask: when does it actually make
\textit{sense} to use those operations? The answer turns out to be: not always.

\index{syntax}
\index{semantics}
\index{meaning}
Another way to phrase this distinction is in terms of \textbf{syntax}
vs.~\textbf{semantics}. Syntax concerns the rules for combining various symbols
in a programming (or other) language. Semantics concerns the \textit{meaning}
of those symbols. This isn't something a programming language can tell us; only
a human who understands what all those symbols refer to can determine when a
particular combination actually relates to something meaningful. Let this
chapter be your guide.

\index{scales of measure}
\section{The four scales of measure}

Every variable we collect can have various values, and the nature of
information it contains can be described by its \textbf{scale of measure}.
There are four of these\footnote{According to psychologist Stanley Smith
Stevens in 1946. Other researchers have developed related, but different,
scales of measure.}, and each one determines which kinds of operations are
``legal'' (\textit{i.e.}, sensible) with that variable.

