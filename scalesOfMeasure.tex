
\chapter{Scales of measure}

In the last chapter, we learned the Python verbiage for how to do arithmetic
operations. In this one, we zoom out and ask: when does it actually make
\textit{sense} to use those operations? The answer turns out to be: not always.

\index{syntax}
\index{semantics}
\index{meaning}
Another way to phrase this distinction is in terms of \textbf{syntax}
vs.~\textbf{semantics}. Syntax concerns the rules for combining various symbols
in a programming (or other) language. Semantics concerns the \textit{meaning}
of those symbols. This isn't something a programming language can tell us. Only
a human who understands what all those symbols refer to can determine when a
particular combination actually relates to something meaningful. Let this
chapter be your guide.

\index{scales of measure}
\section{The four scales of measure}

Every variable we collect can have various values, and the nature of
information it contains can be described by its \textbf{scale of measure}.
There are four of these\footnote{According to psychologist Stanley Smith
Stevens in 1946. Other researchers have developed related, but different,
scales of measure.}, and each one determines which kinds of operations are
``legal'' (\textit{i.e.}, sensible) with that variable.

\index{categorical variable}
\index{nominal variable}
\subsection{Categorical/nominal}

The first one is the simplest, although it actually has two different names in
common use: \textbf{categorical} or \textbf{nominal} variables. It is the scale
of variables that represent ``one of a set of choices,'' where no choice is
``higher'' or ``greater'' than any other.

\index{order}
An example would be a \texttt{fave\_color} variable that holds the value of a
child's favorite color: legal values are \texttt{"red"}, \texttt{"blue"},
\texttt{"green"} or \texttt{"yellow"}. We know it's categorical from, among
other things, the fact that there's no one right way to \textbf{order} those
values. (Alphabetical, most-popular-first, and ordering according to the
sequence of the rainbow are three possibilities. You might think of others.)

Political affiliation would be another categorical variable. Its values (like
\texttt{"Democrat"}, \texttt{"Republican"}, and \texttt{"Green"}) aren't in any
particular order. (Although you might think of the traditional left-to-right
political spectrum, that's only one dimension of political party, and perhaps
not even the most important one.) Other examples include film's genre, a
student's nationality, and a football player's position.

Now you might be tempted to think, ``hmm...all the categorical examples so far
are textual, not numeric. Perhaps this scales of measure thing is just another
way of stating the variable type?'' Alas, no. For one, we'll see text variables
in the next category as well. For another, even data that on its surface seems
numeric can actually be categorical in disguise.

Consider the uniform number of an athlete. I might be interested in asking,
``which uniform number had the greatest professional athletes who chose it?''
24 is a good candidate: Willie Mays, Ken Griffey Jr., and Kobe Bryant all wore
that jersey number. Or maybe \#7 is the winner, with Mickey Mantle, John Elway,
and Cristiano Ronaldo. Either way, though, all that matters in this analysis is
\textit{which} uniform number an athlete chose, not ``how high'' or ``how
great'' that number is compared to others. No one in their right mind would say
that Peyton Manning (\#18) was ``twice the player'' Mia Hamm was (\#9), because
uniform numbers aren't really ``numbers'' at all: they're more like labels.

\subsubsection{Legal operations for categorical/nominal variables}

\index{mode}
When a variable is on a categorical scale, about the only things you can do are
compare for equality/inequality, count the occurrences of different values, and
compute the \textbf{mode} of the set. The mode simply means the value that
occurs \textit{the most often}.

So these things make sense to ask:

\begin{compactitem}
\item[\leftthumbsup] ``Is his favorite color the same as her favorite color?''
\item[\leftthumbsup] ``How many people have \texttt{"red"} as their favorite
color?''
\item[\leftthumbsup] ``What's the most popular favorite color?''
\end{compactitem}

while these do \textit{not}:

\begin{compactitem}
\item[\leftthumbsdown] ``Is his favorite color greater than her favorite
color?'' (??)
\item[\leftthumbsdown] ``What's Caitlin's favorite color minus Hannah's?'' (??)
\item[\leftthumbsdown] ``What's the `average' favorite color in this data
set?'' (??)
\end{compactitem}

