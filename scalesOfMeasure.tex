
% note that "variable" means something different in this chapter. Not a single
% value, but multiple measurements.

\chapter{Scales of measure}

In the last chapter, we learned the Python verbiage for how to do arithmetic
operations. In this one, we zoom out and ask: when does it actually make
\textit{sense} to use those operations? The answer turns out to be: not always.

\index{syntax}
\index{semantics}
\index{meaning}
Another way to phrase this distinction is in terms of \textbf{syntax}
vs.~\textbf{semantics}. Syntax concerns the rules for combining various symbols
in a programming (or other) language. Semantics concerns the \textit{meaning}
of those symbols. This isn't something a programming language can tell us. Only
a human who understands what all those symbols refer to can determine when a
particular combination actually relates to something meaningful. Let this
chapter be your guide.

\index{scales of measure}
\section{The four scales of measure}

Every variable we collect can have various values, and the nature of
information it contains can be described by its \textbf{scale of measure}.
There are four of these\footnote{According to psychologist Stanley Smith
Stevens in 1946. Other researchers have developed related, but different,
scales of measure.}, and each one determines which kinds of operations are
``legal'' (\textit{i.e.}, sensible) with that variable.

\index{categorical variable}
\index{nominal variable}
\subsection{Categorical/nominal}

The first one is the simplest, although it actually has two different names in
common use: \textbf{categorical} or \textbf{nominal} variables. It is the scale
of variables that represent ``one of a set of choices,'' where no choice is
``higher'' or ``greater'' than any other.

\index{order}
An example would be a \texttt{fave\_color} variable that holds the value of a
child's favorite color: legal values are \texttt{"red"}, \texttt{"blue"},
\texttt{"green"} or \texttt{"yellow"}. We know it's categorical from, among
other things, the fact that there's no one right way to \textbf{order} those
values. (Alphabetical, most-popular-first, and ordering according to the
sequence of the rainbow are three possibilities. You might think of others.)

Political affiliation would be another categorical variable. Its values (like
\texttt{"Democrat"}, \texttt{"Republican"}, and \texttt{"Green"}) aren't in any
particular order. (Although you might think of the traditional left-to-right
political spectrum, that's only one dimension of political party, and perhaps
not even the most important one.) Other examples include film's genre, a
student's nationality, and a football player's position.

Now you might be tempted to think, ``hmm...all the categorical examples so far
are textual, not numeric. Perhaps this scales of measure thing is just another
way of stating the variable type?'' Alas, no. For one, we'll see text variables
in the next category as well. For another, even data that on its surface seems
numeric can actually be categorical in disguise.

Consider the uniform number of an athlete. I might be interested in asking,
``which uniform number had the greatest professional athletes who chose it?''
24 is a good candidate: Willie Mays, Ken Griffey Jr., and Kobe Bryant all wore
that jersey number. Or maybe \#7 is the winner, with Mickey Mantle, John Elway,
and Cristiano Ronaldo. Either way, though, all that matters in this analysis is
\textit{which} uniform number an athlete chose, not ``how high'' or ``how
great'' that number is compared to others. No one in their right mind would say
that Peyton Manning (\#18) was ``twice the player'' Mia Hamm was (\#9), because
uniform numbers aren't really ``numbers'' at all: they're more like labels.

\subsubsection{Legal operations for categorical/nominal variables}

\index{mode}
\index{central tendency}
\index{measure of central tendency}
When a variable is on a categorical scale, about the only things you can do are
compare for equality/inequality, count the occurrences of different values, and
compute something called the \textbf{mode} of the set.

The mode simply means the value that occurs \textit{the most often}. It's the
first of the ``\textbf{measures of central tendency}'' we'll see: such measures
are a way of capturing something about the ``typical'' value of a variable. For
categorical variables, the only typical-ness is ``which one occurs the most
often?'' If we ask a bunch of people for their \texttt{fave\_color}, and we get
the answers \texttt{"blue"}, \texttt{"red"}, \texttt{"blue"}, \texttt{"blue"},
and \texttt{"yellow"}, then the mode is \texttt{"blue"}. It's that simple.

To wrap things up, these things make sense to ask of a \textit{categorical}
variable:

\begin{compactitem}
\item[\leftthumbsup] ``Is his favorite color the same as her favorite color?''
\item[\leftthumbsup] ``How many people have \texttt{"red"} as their favorite
color?''
\item[\leftthumbsup] ``What's the most popular favorite color?''
\end{compactitem}

while these do \textit{not}:

\begin{compactitem}
\item[\leftthumbsdown] ``Is his favorite color greater than her favorite
color?'' (??)
\item[\leftthumbsdown] ``What's Caitlin's favorite color minus Hannah's?'' (??)
\item[\leftthumbsdown] ``What's the `average' favorite color in this data
set?'' (??)
\end{compactitem}


\index{ordinal variable}
\index{order}
\subsection{Ordinal}

One step up on the food chain is an \textbf{ordinal} variable, which means that
its different possible values \textit{do} have some meaningful order.

Consider \texttt{education\_level}, a variable that contains the highest degree
a survey respondent has earned. Its values can be any of the following:
\texttt{"HS"}, \texttt{"Associates"}, \texttt{"Bachelors"}, \texttt{"Masters"},
and \texttt{"PhD"}. In some ways, this is like \texttt{fave\_color}: the
variable must take on one of a set of specific, prescribed values. However,
it's pretty clear that a High School degree is closer to (more similar to) an
Associates degree than it is to a Ph.D. Each successive value represents more
education, and so unlike categorical variables, it \textit{does} make sense to
compare them along greater-than-or-less-than lines.

In addition to the mode, another measure of central tendency available for
ordinal variables is the \textbf{median}. I think of the median as the
``middlest'' value: if you line up all the occurrences in a row -- in order of
the values -- it's the one that lies in the exact middle. Suppose our survey
respondents give these answers: \texttt{"Bachelors"}, \texttt{"HS"},
\texttt{"HS"}, \texttt{"Masters"}, \texttt{"Masters"}, \texttt{"Bachelors"},
and \texttt{"HS"}. To compute the median, we line them all up in order:

\begin{center}
\texttt{"HS"  "HS"  "HS"  "Bachelors"  "Bachelors"  "Masters"  "Masters"}
\end{center}

nd find the middlest one, which is \texttt{"Bachelors"}. So \texttt{"HS"} is
the mode of this variable, and \texttt{"Bachelors"} is the median.

Other examples of ordinal variables include an NCAA basketball team's top-25
ranking, a taxpayer's tax bracket, and survey questions asking whether you
\texttt{"strongly disagree"}, \texttt{"disagree"}, are \texttt{"neutral"},
\texttt{"agree"}, or \texttt{"strongly agree"} with a certain statement.

Again, a list. For \textit{ordinal} variables, these are okay:

\begin{compactitem}
\item[\leftthumbsup] ``Is his education level the same as her education level?''
\item[\leftthumbsup] ``How many people answered \texttt{"strongly disagree"} to
this question?''
\item[\leftthumbsup] ``Is UMW basketball ranked higher or lower than Messiah?''
\item[\leftthumbsup] ``What's the median tax bracket for this group of
employees?''
\end{compactitem}

and these are \textit{not}:

\begin{compactitem}
\item[\leftthumbsdown] ``Which looks like the bigger mismatch on paper: Duke
v.~Kentucky, or Villanova v.~Gonzaga?'' (??)
\item[\leftthumbsdown] ``What's Caitlin's education level minus Hannah's?'' (??)
\item[\leftthumbsdown] ``What's the `average' tax bracket for this group of
employees?''
\end{compactitem}

It's worth commenting on that second list, because you might have thought some
of those items were completely reasonable. For example, suppose that in the
latest AP poll, Duke is currently ranked \#1, Kentucky \#3, Villanova \#4, and
Gonzaga \#23. You might think that clearly the Villanova/Gonzaga matchup is the
most lopsided, since there's nineteen places between them, whereas Duke and
Kentucky are separated by just two.

But not necessarily. We know Duke is considered \textit{stronger} than
Kentucky, but not \textit{how much stronger}. It is almost certainly not the
case that the teams are exactly evenly spaced all the way down the list from
\#1 to \#25. Real life doesn't work like that. Instead, it might be the case
that Duke and Georgetown, the \#1 and \#2 teams in the country, are considered
\textit{far and away} the best two teams. And perhaps the next five or even
twenty teams on the list are considered very close, to the point where experts
disagree wildly on what order they should be in. If this is the case, then
mighty Duke vs.~(comparatively) lowly Kentucky might be an enormous mismatch,
while Villanova and Gonzaga is considered a tossup.

\index{order}
\index{spacing}
The bottom line is: although an ordinal variable's values are \textit{ordered},
there is no information at all about the \textit{spacing} between them. I'll
tell you from personal experience that the difference between a Bachelors and a
Masters degree is nuthin' compared to that between a Masters and a Ph.D. (You
can ask anyone who has earned the latter for confirmation.)

This leads into the second item on the no-no list: subtracting two ordinal
values. All you're going to get is ``the number of positions in the sequence by
which they differ,'' which tells you next to nothing. If I ask people to rate a
movie on a scale of \texttt{"POOR"}, \texttt{"FAIR"}, \texttt{"GOOD"}, and
\texttt{"EXCELLENT"}, the difference between \texttt{"POOR"} and
\texttt{"GOOD"} is likely to be a lot greater than that between \texttt{"FAIR"}
and \texttt{"EXCELLENT"}. This is true even though the ``difference'' between
them seems exactly the same: ``two ranking's worth.'' The fact is that humans
don't interpret those four adjectives as exactly equally spaced, and therefore
it's a fallacy to interpret their results as though they did.

Which leads to the third and last item: trying to take the ``average'' (adding
up all the scores and dividing by the total). It's tempting to say, ``let's
treat \texttt{"POOR"} as a 1, \texttt{"FAIR"} as a 2, \texttt{"GOOD"} as a 3,
and \texttt{"EXCELLENT"} as a 4. Then, we can just take the mean of all the
results to get the average rating! What's not to like?'' What's not to like is
that by assigning those numbers, you added spurious information and thereby
twisted the respondent's meaning into something they didn't necessarily intend.
They very likely didn't think of the four options as equally-spaced
numerically, and so this average is quite bogus. Instead, take the median.

