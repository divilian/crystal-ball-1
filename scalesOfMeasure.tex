
% note that "variable" means something different in this chapter. Not a single
% value, but multiple measurements.

% Give the hierarchy of categorical -> ord -> int -> rat, and how each one
% subsumes the other

\chapter{Scales of measure}
\label{ch:scalesOfMeasure}

In the last chapter, we learned the Python verbiage for how to do arithmetic
operations. In this one, we zoom out and ask: when does it actually make
\textit{sense} to use those operations? The answer turns out to be: not always.

\index{syntax}
\index{semantics}
\index{meaning}
Another way to phrase this distinction is in terms of \textbf{syntax}
vs.~\textbf{semantics}. Syntax concerns the rules for combining various symbols
in a programming (or other) language. Semantics concerns the \textit{meaning}
of those symbols. This isn't something a programming language can tell us. Only
a human who understands what all those symbols refer to can determine when a
particular combination actually relates to something meaningful.

\index{scales of measure}
\section{The four scales of measure}

\index{variable}
\index{variable!aggregate}
\label{variableDifferent}
Every variable\footnote{Note that our use of the term \textbf{variable} in this
chapter is different than how we used it in chapter~\ref{ch:atomicData}
(\textit{e.g.}, p.~\pageref{sec:envsAndVariables}) and throughout
chapter~\ref{ch:calculations}. Here, a variable is normally some measurable
aspect of \textit{every} object in our study. We might recruit participants to
a research experiment, and record their race, weight, and favorite breakfast
cereal. These are our three variables. Each of the three constitutes
\textit{many} values, since our participants have many races, weights, and
cereals. In programming terms, they will eventually become \textbf{aggregate} data types
of some kind.} we collect can have various values, and the nature of
information it contains can be described by its \textbf{scale of measure}.
There are four of these\footnote{According to psychologist Stanley Smith
Stevens in 1946. Other researchers have developed related, but different,
scales of measure.}, and each one determines which %kinds of
operations are ``legal'' (\textit{i.e.}, sensible) with that variable.

\index{categorical variable}
\index{nominal variable}
\subsection{Categorical/nominal}

The first one is the simplest, although it actually has two different names in
common use: \textbf{categorical} or \textbf{nominal} variables. It is the scale
of variables that represent ``one of a set of choices,'' where no choice is
``higher'' or ``greater'' than any other.

\index{order}
An example would be a \texttt{fave\_color} variable that holds the value of a
child's favorite color: legal values are \texttt{"red"}, \texttt{"blue"},
\texttt{"green"} or \texttt{"yellow"}. We know it's categorical from, among
other things, the fact that there's no one right way to \textbf{order} those
values. (Alphabetical, most-popular-first, and ordering according to the
sequence of the rainbow are three possibilities. You might think of others.)

Political affiliation would be another categorical variable. Its values (like
\texttt{"Democrat"}, \texttt{"Republican"}, and \texttt{"Green"}) aren't in any
particular order. (Although you might think of the traditional left-to-right
political spectrum, that's only one dimension of political party, and perhaps
not even the most important one.) Other examples include film's genre, a
student's nationality, and a football player's position.

Now you might be tempted to think, ``hmm...all the categorical examples so far
are textual, not numeric. Perhaps this scales of measure thing is just another
way of stating the variable type?'' Alas, no. For one, we'll see text variables
in the next category as well. For another, even data that on its surface seems
numeric can actually be categorical in disguise.

Consider the uniform number of an athlete. I might be interested in asking,
``which uniform number had the greatest professional athletes who chose it?''
24 is a good candidate: Willie Mays, Ken Griffey Jr., and Kobe Bryant all wore
that jersey number. Or maybe \#7 is the winner, with Mickey Mantle, John Elway,
and Cristiano Ronaldo. Either way, though, all that matters in this analysis is
\textit{which} uniform number an athlete chose, not ``how high'' or ``how
great'' that number is compared to others. No one in their right mind would say
that Peyton Manning (\#18) was ``twice the player'' Mia Hamm was (\#9), because
uniform numbers aren't really ``numbers'' at all: they're more like labels.

\subsubsection{Legal operations for categorical/nominal variables}

\index{mode}
\index{central tendency}
\index{measure of central tendency}
When a variable is on a categorical scale, about the only things you can do are
compare for equality/inequality, count the occurrences of different values, and
compute something called the \textbf{mode} of the set.

\label{mode}
The mode simply means the value that occurs \textit{the most often}. It's the
first of the ``\textbf{measures of central tendency}'' we'll see: such measures
are a way of capturing something about the ``typical'' value of a variable. For
categorical variables, the only typical-ness is ``which one occurs the most
often?'' If we ask a bunch of people for their \texttt{fave\_color}, and we get
the answers \texttt{"blue"}, \texttt{"red"}, \texttt{"blue"}, \texttt{"blue"},
and \texttt{"yellow"}, then the mode is \texttt{"blue"}. It's that simple.

To wrap things up, these things make sense to ask of a \textit{categorical}
variable:

\begin{compactitem}
\item[\leftthumbsup] ``Is his favorite color the same as her favorite color?''
\item[\leftthumbsup] ``How many people have \texttt{"red"} as their favorite
color?''
\item[\leftthumbsup] ``What's the most popular favorite color?''
\end{compactitem}

while these do \textit{not}:

\begin{compactitem}
\item[\leftthumbsdown] ``Is his favorite color greater than her favorite
color?'' (??)
\item[\leftthumbsdown] ``What's Caitlin's favorite color minus Hannah's?'' (??)
\item[\leftthumbsdown] ``What's the `average' favorite color in this data
set?'' (??)
\end{compactitem}


\index{ordinal variable}
\index{order}
\subsection{Ordinal}

One step up on the food chain is an \textbf{ordinal} variable, which means that
its different possible values \textit{do} have some meaningful order.

Consider \texttt{education\_level}, a variable that contains the highest degree
a survey respondent has earned. Its values can be any of the following:
\texttt{"HS"}, \texttt{"Associates"}, \texttt{"Bachelors"}, \texttt{"Masters"},
and \texttt{"PhD"}. In some ways, this is like \texttt{fave\_color}: the
variable must take on one of a set of specific, prescribed values. However,
it's pretty clear that a High School degree is closer to (more similar to) an
Associates degree than it is to a Ph.D. Each successive value represents more
education, and so unlike categorical variables, it \textit{does} make sense to
compare them along greater-than-or-less-than lines.

\index{median}

In addition to the mode, another measure of central tendency available for
ordinal variables is the \textbf{median}. I think of the median as the
``middlest'' value: if you line up all the occurrences in a row -- in order of
the values -- it's the one that lies in the exact middle. Suppose our survey
respondents give these answers: \texttt{"Bachelors"}, \texttt{"HS"},
\texttt{"HS"}, \texttt{"Masters"}, \texttt{"Masters"}, \texttt{"Bachelors"},
and \texttt{"HS"}. To compute the median, we line them all up in order:

\begin{center}
\small
\texttt{"HS"  "HS"  "HS"  "Bachelors"  "Bachelors"  "Masters"  "Masters"}
\end{center}

and find the middlest one, which is \texttt{"Bachelors"}. So \texttt{"HS"} is
the mode of this variable, and \texttt{"Bachelors"} is the median.

Other examples of ordinal variables include an NCAA basketball team's top-25
ranking, a taxpayer's tax bracket, and survey questions asking whether you
\texttt{"strongly disagree"}, \texttt{"disagree"}, are \texttt{"neutral"},
\texttt{"agree"}, or \texttt{"strongly agree"} with a certain statement.

Again, a list. For \textit{ordinal} variables, these are okay:

\begin{compactitem}
\item[\leftthumbsup] ``Is his education level the same as her education level?''
\item[\leftthumbsup] ``How many people answered \texttt{"strongly disagree"} to
this question?''
\item[\leftthumbsup] ``Is UMW basketball ranked higher or lower than Messiah?''
\item[\leftthumbsup] ``What's the median tax bracket for this group of
employees?''
\end{compactitem}

and these are \textit{not}:

\begin{compactitem}
\item[\leftthumbsdown] ``Which looks like the bigger mismatch on paper: Duke
v.~Kentucky, or Villanova v.~Gonzaga?'' (??)
\item[\leftthumbsdown] ``What's Caitlin's education level minus Hannah's?'' (??)
\item[\leftthumbsdown] ``What's the `average' tax bracket for this group of
employees?''
\end{compactitem}

It's worth commenting on that second list, because you might have thought some
of those items were completely reasonable. For example, suppose that in the
latest AP poll, Duke is currently ranked \#1, Kentucky \#3, Villanova \#4, and
Gonzaga \#23. You might think that clearly the Villanova/Gonzaga matchup is the
most lopsided, since there's nineteen places between them, whereas Duke and
Kentucky are separated by just two.

But not necessarily. We know Duke is considered \textit{stronger} than
Kentucky, but not \textit{how much stronger}. It is almost certainly not the
case that the teams are exactly evenly spaced all the way down the list from
\#1 to \#25. Real life doesn't work like that. Instead, it might be the case
that Duke and Georgetown, the \#1 and \#2 teams in the country, are considered
\textit{far and away} the best two teams. And perhaps the next five or even
twenty teams on the list are considered very close, to the point where experts
disagree wildly on what order they should be in. If this is the case, then
mighty Duke vs.~(comparatively) lowly Kentucky might be an enormous mismatch,
while Villanova and Gonzaga is considered a tossup.

\index{order}
\index{spacing}
The bottom line is: although an ordinal variable's values are \textit{ordered},
there is no information at all about the \textit{spacing} between them. I'll
tell you from personal experience that the difference between a Bachelors and a
Masters degree is nuthin' compared to that between a Masters and a Ph.D. (You
can ask anyone who has earned the latter for confirmation.)

This leads into the second item on the no-no list: subtracting two ordinal
values. All you're going to get is ``the number of positions in the sequence by
which they differ,'' which tells you next to nothing. If I ask people to rate a
movie on a scale of \texttt{"POOR"}, \texttt{"FAIR"}, \texttt{"GOOD"}, and
\texttt{"EXCELLENT"}, the difference between \texttt{"POOR"} and
\texttt{"GOOD"} is likely to be a lot greater than that between \texttt{"FAIR"}
and \texttt{"EXCELLENT"}. This is true even though the ``difference'' between
them seems exactly the same: ``two ranking's worth.'' The fact is that humans
don't interpret those four adjectives as exactly equally spaced, and therefore
it's a fallacy to interpret their results as though they did.

Which leads to the third and last item: trying to take the ``average'' (adding
up all the scores and dividing by the total). It's tempting to say, ``let's
treat \texttt{"POOR"} as a 1, \texttt{"FAIR"} as a 2, \texttt{"GOOD"} as a 3,
and \texttt{"EXCELLENT"} as a 4. Then, we can just take the mean of all the
results to get the average rating! What's not to like?'' What's not to like is
that by assigning those numbers, you added spurious information and thereby
twisted the respondent's meaning into something they didn't necessarily intend.
They very likely didn't think of the four options as equally-spaced
numerically, and so this average is quite bogus. Instead, take the median.


\index{interval variable}
\subsection{Interval}

Onward. Our next scale of measure is the \textbf{interval} scale, which
fulfills what was missing with ordinal variables. An interval variable
\textit{does} have meaningful and reliable differences between values, which
can be computed and analyzed.

Unlike the previous two scales, interval variables are always numeric by
nature. You can't subtract two words from one another, but you can do so with
numbers, and unlike our uniform number and NCAA hoops ranking examples, that
subtraction is a \textit{meaningful} operation.

An example of an interval variable might be the longitude (or latitude) of a
city. Not only can we ask whether two cities have the same longitude (as with
categorical), and whether one is east or west of another (as with ordinal), we
can now ask \textit{how far} east. Subtract one longitude from the other, and
boom. We have a reliable degree of difference.

This allows us to ask questions like ``are Dallas and Fort Worth farther apart
than Minneapolis and St.~Paul are?'' or ``is the temperature swing between
daytime and nighttime wider in Colorado than in Virginia?'' (Hint: yes.) Note
that we couldn't legally ask such questions of an ordinal variable, since there
was no way to really know how large the difference between \textbf{"GOOD"} and
\textbf{"EXCELLENT"} was, as opposed to that between \textbf{"FAIR"} and
\textbf{"GOOD"}.

Another example of an interval scale variable, besides the aforementioned
temperature, is the year an event takes place. We can say, for example, that
nearly two-thirds of our nation's history has occurred \textit{after} the Civil
War ($2019-1865=154$ years, versus $1861-1776=85$ years).

\index{mean}
The quintessential measure of central tendency for interval scale is the
arithmetic \textbf{mean}. Both the median and the mode are still permitted, and
they are sometimes quite useful. But often we're going to fall back on the
add-'em-up-and-divide-by-the-number-of-elements thing you learned in grade
school. In this case, it makes sense, because the values are at fixed,
meaningful, numerical positions and so adding them up is okay.

Here's our list of goods (for interval scale variables):

\begin{compactitem}
\item[\leftthumbsup] ``Was today's high temperature the same as yesterday's?''
\item[\leftthumbsup] ``Was Beethoven born before or after Napoleon?''
\item[\leftthumbsup] ``How many cities are at $40\deg$ latitude?''
\item[\leftthumbsup] ``What's the median year of birth for current U.S. Senators?''
\item[\leftthumbsup] ``Which is experiencing more global warming
(temperature difference) -- Greenland or France?''
\item[\leftthumbsup] ``What's London's latitude minus Boston's? How much
farther north is it?''
\item[\leftthumbsup] ``What was the average high temperature in Fredericksburg
in September?''
\end{compactitem}

and bads:

\begin{compactitem}
\item[\leftthumbsdown] ``Which cities are at least 20\% more east than
Chicago?'' (??)
\item[\leftthumbsdown] ``When was the first fall day which was half as hot as
it was on July 4th?'' (??)
\item[\leftthumbsdown] ``Was Lincoln born 5\% later than Washington?'' (??)
\end{compactitem}

\index{zero point}
Let's consider that bads list. With an interval scale variable, we can ask
almost anything we want to about it. Almost. The one fly in the ointment is
questions that have phrases like ``twice as'' or ``10\% less than.'' Those, we
cannot do. The reason is that an interval scale variable \textit{has no
meaningful \textbf{zero point}}.

In an interval scale, values have \textit{relative} distances from each other,
but not \textit{absolute} differences from some fixed reference point. Consider
years. Saying that the Cubs finally won the World Series 146 years after their
franchise was born is meaningful: the difference between 1870 and 2016 can be
measured. But what if we said ``they won the World Series 7.8\% later than
their franchise was born''? Could such a sentence possibly say anything useful?

\index{arbitrary}
The answer is no, and here's why. The ``zero point'' of our calendar system is
\textbf{arbitrary}. By that I mean that the year we might consider ``year
zero'' has nothing to do with the Cubs or baseball or America or anything else:
it was a guess as to the birth year of Jesus Christ, and a wrong one at
that.\footnote{Later historical discoveries have demonstrated that Herod the
Great died in what we now call 4 B.C. If you went to Sunday School, you might
recall that in a fit of jealousy, King Herod the Great ordered all the baby
boys in Bethlehem (two years old or younger) to be killed. (See Matthew
2:13-18.) He chose ``two years or younger'' as the cutoff because his goal was
to kill Jesus, who was about two years old at the time. Hence Jesus was most
likely born in the year which we have (incorrectly, it turns out) labeled as
``6 B.C.'' Fun facts.}

\index{relative difference}
\index{absolute difference}
We could, of course, have chosen to measure time relative to any other point
instead, like the birth of our own nation, the founding of Rome, the Cubs
franchise being founded, or anything else. If we had done that, all of the
\textit{relative} differences between years would have been the same: there
would still have been 85 years between the Declaration of Independence and the
Civil War, Barack Obama would still have been President for 8 years, and you
would still be the same age. But all the \textit{absolute} calculations that
implicitly make reference to the zero point -- like ``what percent later did
the Cubs won the Series than their franchise began?'' would suddenly become
radically different. If we measured years relative to 1776, then the Cubs'
victory would have been ``155.3\% later'' than their origin, instead of ``7.8\%
later!'' That betrays the fact that this is an utterly meaningless calculation.

Same thing with longitude. While latitude plausibly has a meaningful zero point
-- the equator -- and thus perhaps ``twice as north'' has some meaning to it
(``twice as far from the center of the planet'') longitude clearly does not.
Saying a city is ``twice as east'' as another is plain nonsense. That's because
the zero point for longitude is arbitrary: it's set at the Greenwich, England,
or some nonsense like that. Clearly only relative differences between longitude
have any meaning.

And the same thing with temperature. If yesterday was $40\deg$, and today was
$80\deg$, it's tempting to say that ````whew! It's twice as hot today!'' To see
that this is nonsense, consider that we could (and probably should, actually)
change to use the metric system like the rest of the world does, and measure
temperature in Celsius. Now if we did that, clearly we wouldn't start
experiencing heat waves or cold spells as a result! Hey we're just changing our
units, bro, not influencing the atmosphere. But realize that in Celsius,
yesterday's $40\deg$F day would become $4.4\deg$C, and today's $80\deg$F would
be $26.7\deg$C. So now, by changing our units, we would have to say ``oh golly,
I guess it's actually over \textit{six times} as hot today!'' This is why
multiplying and dividing with interval scale variables leads to madness.


\index{ratio variable}
\subsection{Ratio}

Which brings us to our last of the four scales: the ratio scale. In some ways
this is the easiest to understand, because of all the mathematical questions
we might want to ask, we \textit{can} ask them. Multiply, divide, make 
absolute statements like ``25\% greater than'' -- go crazy, man.

Salary has a meaningful, absolute zero point: namely, an unemployed (or
volunteer) worker earning \textit{zero} dollars. Since we have that
non-arbitrary standard, it makes perfect sense to say things like ``he makes
twice as much as she does.

The height of a person has a meaningful zero point as well: the ground. If
Tyrion Lannister rises $3\frac{1}{2}$ feet from the floor, and Gregor Clegane
stands a full 7 feet from that same floor, it makes all the sense in the world
to say ``Gregor is twice as tall as Tyrion.''

\index{mean}
As with interval scale variables, we often use the arithmetic \textbf{mean}  as
our measure of central tendency.\footnote{Interestingly, there are actually two
different kinds of means, one of which, called the ``geometric mean'' is only
applicable on the ratio data scale. It involves multiplying and taking roots
instead of adding and dividing, and is a useful operation in some niche
contexts.}

\section{Final word}
The lesson of this chapter is that Python will \textit{not} prevent you from
doing any of the above stupid things -- if we have an ordinal scale variable,
for instance, we can subtract values from one other until we're blue in the
face, not recognizing that the results we're producing are gibberish. It's all
on us to be responsible data citizens, and to only use operations that give
meaningful results.
