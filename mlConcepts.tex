
\chapter{Machine Learning: concepts}

When ordinary people hear the words ``Data Science,'' I'll bet the first images
that come to mind are of the closely-related fields of \textbf{data mining} and
\textbf{machine learning (ML)}, even if they don't know those terms. After all,
this is where all the sexy tech is, and the success stories too: Netflix
magically knowing which movies you'll like, grocery chains using data from
loyalty cards to optimally place products; the Oakland A's scouring minor
league stats to build a champion team with chump change
(see:~\textit{Moneyball}). There are also creepier applications of this
technology: Google placing personalized eye-catching ads in front of you using
data they mined from your email text, or Cambridge Analytica projecting from
voter personalities to the best ways to micro-target them.

All these examples have one thing in common: they actually \textit{make} the
discoveries and predictions from the data. They're the coup de gr\^{a}ce. They
take place after we've already acquired our data, imported it to an analysis
environment (like Python), stored it in the appropriate data structures (like
associative arrays or tables), recoded/transformed/pre-processed it as
necessary, and explored it enough to know what we want to ask. All that stuff
was mere prep work. This chapter is where we begin to really rock-and-roll.

\section{Data mining vs.~machine learning}
\index{inference}
\index{prediction}
\index{machine learning (ML)}
\index{data mining}

The terms ``data mining'' and ``ML'' have a lot of sloppy overlap, but one
distinction we can pick out is this. If someone says they're doing data mining,
their goal is normally \textbf{inference}: deriving high-level strategic
insights based on patterns in the data. Discovering that amateur pitching
performances translate more reliably to the major leagues than amateur batting
performances do, generally speaking, is an inference, and a potentially
valuable find.

If someone says they're doing ML, on the other hand, their goal is normally
\textbf{prediction}: making an educated guess about how a specific case will
turn out. When we forecast how many home runs we think a college prospect will
hit in his first two years in the majors, we're making a specific prediction
rather than inferring a general truth -- this, too, is potentially quite
valuable, as it may lead us to decide to sign the player or look at different
options.
