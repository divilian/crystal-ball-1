
\chapter{Recoding and transforming}

It's often the case that although a \texttt{DataFrame} contains the raw
information you need, it's not exactly in the form you need for your analysis.
Perhaps the data is in different units than you need -- meters instead of feet;
dollars instead of yen. Or perhaps you need some \textit{combination} of
available quantities -- miles per gallon instead of just miles and gallons
separately. Or perhaps you need to reframe a variable by binning it into
meaningful subdivisions -- categorizing a raw column of salaries into ``high,''
``medium,'' and ``low'' wage earners, for instance.

\index{derived column}
\index{column (of a table)}

In data science, these activities are known as \textbf{recoding} and/or
\textbf{transforming}. There's not a sharp division between the two; usually I
think of recoding as converting a single variable to one with different units
(as in the dollars-to-yen and high/medium/low earners examples) and
transforming as creating a new variable entirely out of a combination of
columns (like miles per gallon). In both cases, though, we'll be creating and
adding new columns to a \texttt{DataFrame}. These columns are sometimes called
\textbf{derived columns} since they're based on (derived from) existing columns
rather than containing independent information.

\section{Recoding with simple operations}

\index{soccer}
\index{US Women's National Team}

Consider the following soccer data set called \texttt{worldcup2019.csv}.
Each row of this data set represents one player's performance in a particular
2019 World Cup game. Notice that we have a couple of players with more than one
row (Megan Rapinoe and Rose Lavelle), and several rows for the same game (the
first four rows are all from the June 28th game, for instance):

\begin{Verbatim}[fontsize=\small,samepage=true,frame=lines,framesep=3mm]
last,first,date,in_time,out_time,goals,asst,tackles,shots
Morgan,Alex,28-Jun-2019,0.0,90.0,0,0,2,1
Rapinoe,Megan,28-Jun-2019,0.0,74.0,2,0,2,3
Press,Christen,28-Jun-2019,74.0,90.0,0,0,1,0
Lavelle,Rose,28-Jun-2019,0.0,90.0,0,1,3,0
Lavelle,Rose,7-Jul-2019,0.0,90.0,1,0,4,1
Rapinoe,Megan,7-Jul-2019,0.0,83.0,1,1,3,2
Lloyd,Carli,7-Jul-2019,87.0,90.0,0,0,1,0
Dunn,Crystal,23-Jun-2019,42.0,81.0,0,1,1,2
\end{Verbatim}

\index{set\_index@\texttt{.set\_index()} method (Pandas)}

The data set doesn't really have a meaningful index column, since none of the
columns are expected to be unique. So we'll leave off the
``\texttt{.set\_index()}'' method call when we read it in to Python:

\begin{samepage}
\begin{Verbatim}[fontsize=\footnotesize,samepage=true,frame=single,framesep=3mm]
wc = pd.read_csv('worldcup2019.csv')
print(wc)
\end{Verbatim}
\vspace{-.2in}

\begin{Verbatim}[fontsize=\scriptsize,samepage=true,frame=leftline,framesep=5mm,framerule=1mm]
     last    first   date in_mins in_secs out_mins out_secs goals asst tackles shots
0  Morgan     Alex 28-Jun       0       0       90        0     0    0       2     1
1 Rapinoe    Megan 28-Jun       0       0       74       27     2    0       2     3
2   Press Christen 28-Jun      74      27       90        0     0    0       1     0
3 Lavelle     Rose 28-Jun       0       0       90        0     0    1       3     0
4 Lavelle     Rose  7-Jul       0       0       90        0     1    0       4     1
5 Rapinoe    Megan  7-Jul       0       0       83       16     1    1       3     2
6   Lloyd    Carli  7-Jul      83      16       90        0     0    0       1     0
7    Dunn  Crystal 23-Jun      42      37       81        5     0    1       1     2
\end{Verbatim}
\end{samepage}

(As you can see, Pandas put in a numeric index column for us.)

Let's zero in on the columns with \texttt{mins} and \texttt{secs} in the names.
These columns show us the minute and second that the player went \texttt{in} to
the game, and the minute and second that they came \texttt{out}. For example,
Alex Morgan played the entire 90-minute match on June 28th. Rapinoe started
that game, but came out for a substitute at the 74:27 mark. Who replaced her?
Looks like Christen Press did, since she \textit{entered} the game at exactly
the same time. In most rows, the player either started the game, or ended the
game or both, but the last row (Crystal Dunn's June 23rd performance) has her
entering at 42:37 and exiting at 81:05.

Now the reason I bring this up is because one aspect of our analysis might be
computing statistics \textit{per minute} that each athlete played. If one
player scored 3 goals in 200 minutes, for example, and another scored 3 goals
in just 150 minutes, we could reasonably say that the second player was a more
prolific scorer in that World Cup.

\index{recoding}
\index{round@\texttt{round()} function (NumPy)}

This is hard to do with the data in the form that it stands. So we'll
\textbf{recode} a few of the columns. Let's collapse the minutes and seconds
for each of the two clock times into a single value, in minutes. For
readability, we'll also round this number to two decimal places using the
\texttt{round()} function we met on p.~\pageref{round}:

\begin{Verbatim}[fontsize=\small,samepage=true,frame=single,framesep=3mm]
wc['in_time'] = np.round(wc['in_mins'] + (wc['in_secs'] / 60),2)
wc['out_time'] = np.round(wc['out_mins'] + (wc['out_secs'] / 60),2)
\end{Verbatim}

\index{vectorized@``vectorized'' operation}

We're taking advantage of vectorized operations here. For each row, we need to
divide the \texttt{in\_secs} value by 60 (to convert it to minutes) and add it
to the \texttt{in\_mins} value. Pandas makes this super easy here, since we can
just write out those operations once, and it will compute it for every single
row!

\medskip
Let's delete the old, superfluous columns now and see what we've got:

\begin{samepage}
\begin{Verbatim}[fontsize=\small,samepage=true,frame=single,framesep=3mm]
del wc['in_mins']
del wc['in_secs']
del wc['out_mins']
del wc['out_secs']
print(wc)
\end{Verbatim}
\vspace{-.2in}

\begin{Verbatim}[fontsize=\footnotesize,samepage=true,frame=leftline,framesep=5mm,framerule=1mm]
     last    first   date goals asst tackles shots in_time out_time
0  Morgan     Alex 28-Jun     0    0       2     1    0.00    90.00
1 Rapinoe    Megan 28-Jun     2    0       2     3    0.00    74.45
2   Press Christen 28-Jun     0    0       1     0   74.45    90.00
3 Lavelle     Rose 28-Jun     0    1       3     0    0.00    90.00
4 Lavelle     Rose  7-Jul     1    0       4     1    0.00    90.00
5 Rapinoe    Megan  7-Jul     1    1       3     2    0.00    83.27
6   Lloyd    Carli  7-Jul     0    0       1     0   83.27    90.00
7    Dunn  Crystal 23-Jun     0    1       1     2   42.62    81.08
\end{Verbatim}
\end{samepage}

This is much less unwieldy than dealing with minutes and seconds separately.

% TODO: give warning about changing the values of an existing DF column -- you
% get the weird "changing on a slice/copy of a DF" warning, which sometimes
% means it doesn't work. 

\section{Transforming with simple operations}

\index{transforming}

\index{mins\_played@\texttt{mins\_played}}

Now that we've converted the awkward minutes-and-seconds columns to just
``\texttt{time}'' columns, all we need to do to complete our analysis is
\textbf{transform} this data by computing a new quantity entirely: the
\textit{total number of minutes played} for each player in each game. Again,
Pandas makes this easy:

\begin{Verbatim}[fontsize=\small,samepage=true,frame=single,framesep=3mm]
wc['mins_played'] = wc.out_time - wc.in_time
print(wc)
\end{Verbatim}
\vspace{-.2in}

\begin{Verbatim}[fontsize=\footnotesize,samepage=true,frame=leftline,framesep=5mm,framerule=1mm]
     last    first   date goals asst tackles shots in_time out_time mins_played
0  Morgan     Alex 28-Jun     0    0       2     1    0.00    90.00       90.00
1 Rapinoe    Megan 28-Jun     2    0       2     3    0.00    74.45       74.45
2   Press Christen 28-Jun     0    0       1     0   74.45    90.00       15.55
3 Lavelle     Rose 28-Jun     0    1       3     0    0.00    90.00       90.00
4 Lavelle     Rose  7-Jul     1    0       4     1    0.00    90.00       90.00
5 Rapinoe    Megan  7-Jul     1    1       3     2    0.00    83.27       83.27
6   Lloyd    Carli  7-Jul     0    0       1     0   83.27    90.00        6.73
7    Dunn  Crystal 23-Jun     0    1       1     2   42.62    81.08       38.46
\end{Verbatim}

\index{tackles-per-game}
\index{tkl\_per\_90@\texttt{tkl\_per\_90}}

Voil\`{a}. We now have the time-on-field for each player, which gives us a
whole new avenue of exploration. For example, any of the counting stats (goals,
assists, \textit{etc.}) can be converted into a ``per-minute'' version, showing
us how productive a player was while on the field. Let's do that for
\texttt{tackles}, and multiply by 90 to obtain a ``tackles-per-90-minutes''
statistic\footnote{I'm choosing 90 minutes here because that's how long a
regulation-length soccer match is. Therefore, our new \texttt{tkl\_per\_90}
column gives us ``number-of-tackles-per-complete-game,'' which is easier to
interpret than ``tackles-per-\textit{minute},'' which would be a miniscule
number for any player.}:

\begin{Verbatim}[fontsize=\small,samepage=true,frame=single,framesep=3mm]
wc['mins_played'] = wc['out_time'] - wc['in_time']
wc['tkl_per_90'] = np.round(wc['tackles'] / wc['mins_played'] * 90,2)
del wc['tackles']
\end{Verbatim}
\vspace{-.2in}

\begin{Verbatim}[fontsize=\footnotesize,samepage=true,frame=leftline,framesep=5mm,framerule=1mm]
     last    first   date goals asst shots in_time out_time mins_played tkl_per_90
0  Morgan     Alex 28-Jun     0    0     1    0.00    90.00       90.00       2.00
1 Rapinoe    Megan 28-Jun     2    0     3    0.00    74.45       74.45       2.42
2   Press Christen 28-Jun     0    0     0   74.45    90.00       15.55       5.79
3 Lavelle     Rose 28-Jun     0    1     0    0.00    90.00       90.00       3.00
4 Lavelle     Rose  7-Jul     1    0     1    0.00    90.00       90.00       4.00
5 Rapinoe    Megan  7-Jul     1    1     2    0.00    83.27       83.27       3.24
6   Lloyd    Carli  7-Jul     0    0     0   83.27    90.00        6.73      13.37
7    Dunn  Crystal 23-Jun     0    1     2   42.62    81.08       38.46       2.34
\end{Verbatim}

\subsection{Transforming grouped data}

The above example computed tackles-per-game all right, but it still left us
with one row for every player-performance. (In other words, the results had
\textit{two} rows for Rose Lavelle, one giving her \texttt{tkl\_per\_90} for
the June 28th game, and one giving it for the July 7th game.)

\index{groupby@\texttt{.groupby()} method (Pandas)}

We might instead be interested in a player-by-player analysis: overall in the
entire month-long World Cup, which players had the most tackles-per-game? This
is easy to do with the \texttt{.groupby()} method that we first encountered in
section~\ref{groupby} (p.~\pageref{groupby}). First, we group the rows by the
first \textit{two} columns (since first-and-last-names-together are needed to
uniquely identify a single player):

\index{grouped\_wc@\texttt{grouped\_wc}}

\begin{Verbatim}[fontsize=\footnotesize,samepage=true,frame=single,framesep=3mm]
grouped_wc = wc.groupby(['last','first'])
\end{Verbatim}

We then take our new, temporary \texttt{grouped\_wc} variable and extract the
\texttt{goals}, \texttt{asst}, \texttt{shots}, \texttt{tackles}, and
\texttt{mins\_played} columns from it, \textbf{summing} each of them to produce
the per-player values in the result:

\index{by\_player@\texttt{by\_player}}

\begin{Verbatim}[fontsize=\footnotesize,samepage=true,frame=single,framesep=3mm]
by_player = grouped_wc[['goals','asst','shots','tackles','mins_played']].sum()
\end{Verbatim}

This yields:

\begin{Verbatim}[fontsize=\small,samepage=true,frame=leftline,framesep=5mm,framerule=1mm]
                  goals  asst  shots  tackles  mins_played
last    first                                             
Dunn    Crystal       0     1      2        1        38.46
Lavelle Rose          1     1      1        7       180.00
Lloyd   Carli         0     0      0        1         6.73
Morgan  Alex          0     0      1        2        90.00
Press   Christen      0     0      0        1        15.55
Rapinoe Megan         3     1      5        5       157.72
\end{Verbatim}

Now, we're ready to compute a per-game analysis as before, but this time for
each player's entire World Cup games:

\begin{Verbatim}[fontsize=\small,samepage=true,frame=single,framesep=3mm]
by_player['tkl_per_90'] = (np.round(by_player['tackles'] /
    by_player['mins_played'] * 90,2))
del by_player['tackles']
\end{Verbatim}
\vspace{-.2in}

\begin{Verbatim}[fontsize=\small,samepage=true,frame=leftline,framesep=5mm,framerule=1mm]
                  goals  asst  shots  mins_played  tkl_per_90
last    first                                                
Dunn    Crystal       0     1      2        38.46        2.34
Lavelle Rose          1     1      1       180.00        3.50
Lloyd   Carli         0     0      0         6.73       13.37
Morgan  Alex          0     0      1        90.00        2.00
Press   Christen      0     0      0        15.55        5.79
Rapinoe Megan         3     1      5       157.72        2.85
\end{Verbatim}

\section{Transforming with more complex operations}

\index{vectorized@``vectorized'' operation}

In all the above examples, we took advantage of Pandas vectorized operations.
With just a single line of code like ``\texttt{wc[\textquotesingle
mins\_played\textquotesingle] = wc.out\_time - wc.in\_time}'', we could compute
our entire new transformed column in one fell swoop.

\index{loop}

Sometimes, we're not so lucky. In particular, if the computation of the
transformed column is more complicated than just numeric operations -- like, if
it involves branches, loops, or calling other functions -- we normally can't
compute it all at once. Instead, we have to resort to a loop.

Pandas makes this procedure a bit awkward in my opinion. But once you learn the
pattern, it's not hard to imitate. Here's the pattern for creating a
transformed/recoded column that requires more complex operations:

\definecolor{shadecolor}{rgb}{.9,.9,.9}
\begin{shaded}
\begin{compactenum}
\item Create a \textbf{function} that will compute the transformed value for a
\textit{single} row. Its arguments should be whatever column values are
necessary to derive the new value, and its return value should be the desired
transformation.
\item Create an \textit{empty} NumPy array to hold the row-by-row results, and
make sure it's the right type.
\item Write a loop that will iterate through all the rows of the original
\texttt{DataFrame}. For each row, pass the appropriate values to the function,
and then \textit{append} the return value to the ever-growing NumPy array.
\item Finally, slap that NumPy array on to the \texttt{DataFrame} as a new
column.
\end{compactenum}

\end{shaded}

Here's a couple examples. First, suppose we want to compute a shooting
percentage for each player; in other words, how many goals they scored per shot
they took. Now you might think we could simply use vectorized operations:

\begin{Verbatim}[fontsize=\small,samepage=true,frame=single,framesep=3mm]
wc['shots_per_goal'] = wc.goals / wc.shots
\end{Verbatim}

\index{cardinal sin (division by zero)}

The problem is, for players who never attempted a shot in the game, this would
result in dividing by zero, a cardinal sin. Sports convention says that if a
player makes 0 goals in 0 attempts, their shooting percentage is 0.00, even
though mathematically-speaking this is undefined.

Very well, following our procedure from above, we'll first define a function
\texttt{shooting\_perc()}:

\index{shooting\_perc@\texttt{shooting\_perc()}}
\begin{Verbatim}[fontsize=\small,samepage=true,frame=single,framesep=3mm]
def shooting_perc(goals, shots):
    if shots == 0:
        return 0.0
    else:
        return np.round(goals / shots * 100, 1)
\end{Verbatim}

\index{s\_perc@\texttt{s\_perc}}
Then, we create an empty NumPy array. Here's how:

\begin{Verbatim}[fontsize=\small,samepage=true,frame=single,framesep=3mm]
s_perc = np.array([])
\end{Verbatim}

\index{array@\texttt{array()} (NumPy)}
Looks weird, I know. But remember, the \texttt{array()} function (review
p.~\pageref{arrayFunction}) takes a boxie-enclosed list of elements. If we
enclose \textit{nothing} inside the boxies, that effectively makes it an empty
list.

And why would we want to do that? Because we need to continually add to this
array, one value for each row in the \texttt{DataFrame}. At the end, there must
be \textit{exactly} as many elements in \texttt{s\_perc} as there are rows in
\texttt{wc}, otherwise we won't be able to add it as a new column.

Here's the loop (step 3 from the shaded box):

\begin{Verbatim}[fontsize=\small,samepage=true,frame=single,framesep=3mm]
for row in wc.itertuples():
    new_s_perc = shooting_perc(row.goals, row.shots)
    s_perc = np.append(s_perc, new_s_perc)
\end{Verbatim}

I've chosen to create a temporary variable here (\texttt{new\_s\_perc}) for
readability. The first line of the loop body says to take the current
\texttt{row}'s \texttt{goals} and \texttt{shots} values, and send them as
arguments to the \texttt{shooting\_perc()} function. That function, which we
defined above, will return us a single number which is the shooting percentage
for \textit{that row}. The second line then appends that single
\texttt{new\_s\_perc} value to the end of the ever-growing \texttt{s\_perc}
array.

Finally, we add this new column to the \texttt{wc} \texttt{DataFrame} proper:

\begin{Verbatim}[fontsize=\small,samepage=true,frame=single,framesep=3mm]
wc['s_perc'] = s_perc
\end{Verbatim}

which gives us:

\begin{Verbatim}[fontsize=\footnotesize,samepage=true,frame=leftline,framesep=5mm,framerule=1mm]
     last    first   date goals asst shots in_time out_time mins_played s_perc
0  Morgan     Alex 28-Jun     0    0     1    0.00    90.00       90.00    0.0
1 Rapinoe    Megan 28-Jun     2    0     3    0.00    74.45       74.45   66.7
2   Press Christen 28-Jun     0    0     0   74.45    90.00       15.55    0.0
3 Lavelle     Rose 28-Jun     0    1     0    0.00    90.00       90.00    0.0
4 Lavelle     Rose  7-Jul     1    0     1    0.00    90.00       90.00  100.0
5 Rapinoe    Megan  7-Jul     1    1     2    0.00    83.27       83.27   50.0
6   Lloyd    Carli  7-Jul     0    0     0   83.27    90.00        6.73    0.0
7    Dunn  Crystal 23-Jun     0    1     2   42.62    81.08       38.46    0.0
\end{Verbatim}

Rose Lavelle's July 7th game was the only perfect shooting performance in this
data set -- who knew?

\bigskip

We'll complete this chapter with a slightly more complex example, but which
still follows the shaded box pattern.

\index{starter@\texttt{starter}}
Say we're also interested in which players \textit{started} which games (as
opposed to being a mid-game substitute). Obviously, a starter is someone who
entered the game at time 0. To create a new column for this, we'll need our
function to return the boolean value \texttt{True} if the player's
\texttt{in\_time} value was zero, and \texttt{False} otherwise. Here's
the complete code snippet for this transformation:

\begin{Verbatim}[fontsize=\small,samepage=true,frame=single,framesep=3mm]
def starter_func(in_time):
    if in_time == 0:
        return True
    else:
        return False

starter = np.array([]).astype(bool)

for row in wc.itertuples():
    starter = np.append(starter, starter_func(row.in_time))

wc['starter'] = starter
\end{Verbatim}
\vspace{-.2in}

\begin{Verbatim}[fontsize=\small,samepage=true,frame=leftline,framesep=5mm,framerule=1mm]
     last    first   date goals asst tackles shots mins_played s_perc starter
0  Morgan     Alex 28-Jun     0    0       2     1       90.00    0.0    True
1 Rapinoe    Megan 28-Jun     2    0       2     3       74.45   66.7    True
2   Press Christen 28-Jun     0    0       1     0       15.55    0.0   False
3 Lavelle     Rose 28-Jun     0    1       3     0       90.00    0.0    True
4 Lavelle     Rose  7-Jul     1    0       4     1       90.00  100.0    True
5 Rapinoe    Megan  7-Jul     1    1       3     2       83.27   50.0    True
6   Lloyd    Carli  7-Jul     0    0       1     0        6.73    0.0   False
7    Dunn  Crystal 23-Jun     0    1       1     2       38.46    0.0   False
\end{Verbatim}

\index{boolean value}
\index{float@\texttt{float}}

One subtle point that is easy to miss: when we first created the empty
\texttt{starter} array, we typed ``\texttt{.astype(bool)}'' at the end. This is
because by default, the values of a new empty array will be \textbf{float}s.
This worked fine for the shooting percentage example, because that's actually
what we wanted, but here we want \texttt{True}/\texttt{False} values instead
(for ``starter'' and ``non-starter.'')

Pretty cool, huh? The original \texttt{DataFrame} had the information we
wanted, but not in the form we really needed it. What we wanted was not the
entry time and exit time of each player (both in minutes and seconds) but
rather the total time that player was on the pitch, and whether or not they
started the game. We also wanted to convert several of the raw statistics into
per-complete game numbers, and to compute meaningful ratios like shooting
percentage or fouls per assist.

\bigskip

Recoding and transforming turn out to be common tasks for a simple reason:
\textit{whoever collects a data set can rarely predict how an analyst will
eventually use it.} We're very grateful to the author of the \texttt{.csv}
file, since it contains the raw material we need to evaluate our team's
performances; but how were they to know that length-of-time-on-the-field and
who-started-which-game was going to be important to us? They couldn't. But
thanks to recoding and transformation skills, we can cope.

