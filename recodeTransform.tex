
\chapter{Recoding and transforming}

It's often the case that although a \texttt{DataFrame} contains the raw
information you need, it's not exactly in the form you need for your analysis.
Perhaps the data is in different units than you need -- meters instead of feet;
dollars instead of yen. Or perhaps you need some \textit{combination} of
available quantities -- miles per gallon instead of just miles and gallons
separately. Or perhaps you need to reframe a variable by binning it into
meaningful subdivisions -- categorizing a raw column of salaries into ``high,''
``medium,'' and ``low'' wage earners, for instance.

In data science, these activities are known as \textbf{recoding} and/or
\textbf{transforming}. There's not a sharp division between the two; usually I
think of recoding as converting a single variable to one with different units
(as in the dollars-to-yen and high/medium/low earners examples) and
transforming as creating a new variable entirely out of a combination of
columns (like miles per gallon). In both cases, though, we'll be creating and
adding new columns to a \texttt{DataFrame}.

\section{Recoding with simple operations}

\index{US Women's National Team}

Consider the following soccer data set called \texttt{worldcup2019.csv}.
Each row of this data set represents one player's performance in a particular
2019 World Cup game. Notice that we have a couple of players with more than one
row (Megan Rapinoe and Rose Lavelle), and several rows for the same game (the
first four rows are all from the June 28th game, for instance):

\pagebreak
\begin{Verbatim}[fontsize=\small,samepage=true,frame=lines,framesep=3mm]
last,first,date,in_time,out_time,goals,asst,tackles,fouls
Morgan,Alex,28-Jun-2019,0.0,90.0,0,0,2,1
Rapinoe,Megan,28-Jun-2019,0.0,74.0,2,0,2,2
Press,Christen,28-Jun-2019,74.0,90.0,0,0,1,0
Lavelle,Rose,28-Jun-2019,0.0,90.0,0,1,3,0
Lavelle,Rose,7-Jul-2019,0.0,90.0,1,0,4,1
Rapinoe,Megan,7-Jul-2019,0.0,83.0,1,1,3,2
Lloyd,Carli,7-Jul-2019,87.0,90.0,0,0,1,0
Dunn,Crystal,23-Jun-2019,42.0,81.0,0,1,1,2
\end{Verbatim}

\index{set\_index@\texttt{.set\_index()} method (Pandas)}

The data set doesn't really have a meaningful index column, since none of the
columns are expected to be unique. So we'll leave off the
``\texttt{.set\_index()}'' method call when we read it in to Python:

\begin{samepage}
\begin{Verbatim}[fontsize=\footnotesize,samepage=true,frame=single,framesep=3mm]
wc = pd.read_csv('worldcup2019.csv')
print(wc)
\end{Verbatim}
\vspace{-.2in}

\begin{Verbatim}[fontsize=\scriptsize,samepage=true,frame=leftline,framesep=5mm,framerule=1mm]
     last    first   date in_mins in_secs out_mins out_secs goals asst tackles fouls
0  Morgan     Alex 28-Jun       0       0       90        0     0    0       2     1
1 Rapinoe    Megan 28-Jun       0       0       74       27     2    0       2     2
2   Press Christen 28-Jun      74      27       90        0     0    0       1     0
3 Lavelle     Rose 28-Jun       0       0       90        0     0    1       3     0
4 Lavelle     Rose  7-Jul       0       0       90        0     1    0       4     1
5 Rapinoe    Megan  7-Jul       0       0       83       16     1    1       3     2
6   Lloyd    Carli  7-Jul      83      16       90        0     0    0       1     0
7    Dunn  Crystal 23-Jun      42      37       81        5     0    1       1     2
\end{Verbatim}
\end{samepage}

(As you can see, Pandas put in a numeric index column for us.)

Let's zero in on the columns with \texttt{mins} and \texttt{secs} in the names.
These columns show us the minute and second that the player went \texttt{in} to
the game, and the minute and second that they came \texttt{out}. For example,
Alex Morgan played the entire 90-minute match on June 28th. Rapinoe started
that game, but came out for a substitute at the 74:27 mark. Who replaced her?
Looks like Christen Press did, since she \textit{entered} the game at exactly
the same time. In most rows, the player either started the game, or ended the
game or both, but the last row (Crystal Dunn's June 23rd performance) has her
entering at 42:37 and exiting at 81:05.

Now the reason I bring this up is because one aspect of our analysis might be
computing statistics \textit{per minute} that each athlete played. If one
player scored 3 goals in 200 minutes, for example, and another scored 3 goals
in just 150 minutes, we could reasonably say that the second player was a more
prolific scorer in that World Cup.

\index{recoding}
\index{round@\texttt{round()} function (NumPy)}

This is hard to do with the data in the form that it stands. So we'll
\textbf{recode} a few of the columns. Let's collapse the minutes and seconds
for each of the two clock times into a single value, in minutes. For
readability, we'll also round this number to two decimal places using the
\texttt{round()} function we met on p.~\pageref{round}:

\begin{Verbatim}[fontsize=\small,samepage=true,frame=single,framesep=3mm]
wc['in_time'] = np.round(wc['in_mins'] + (wc['in_secs'] / 60),2)
wc['out_time'] = np.round(wc['out_mins'] + (wc['out_secs'] / 60),2)
\end{Verbatim}

\index{vectorized@``vectorized'' operation}

We're taking advantage of vectorized operations here. For each row, we need to
divide the \texttt{in\_secs} value by 60 (to convert it to minutes) and add it
to the \texttt{in\_mins} value. Pandas makes this super easy here, since we can
just write out those operations once, and it will compute it for every single
row!

\medskip
Let's delete the old, superfluous columns now and see what we've got:

\begin{samepage}
\begin{Verbatim}[fontsize=\small,samepage=true,frame=single,framesep=3mm]
del wc['in_mins']
del wc['in_secs']
del wc['out_mins']
del wc['out_secs']
print(wc)
\end{Verbatim}
\vspace{-.2in}

\begin{Verbatim}[fontsize=\footnotesize,samepage=true,frame=leftline,framesep=5mm,framerule=1mm]
     last    first   date goals asst tackles fouls in_time out_time
0  Morgan     Alex 28-Jun     0    0       2     1    0.00    90.00
1 Rapinoe    Megan 28-Jun     2    0       2     2    0.00    74.45
2   Press Christen 28-Jun     0    0       1     0   74.45    90.00
3 Lavelle     Rose 28-Jun     0    1       3     0    0.00    90.00
4 Lavelle     Rose  7-Jul     1    0       4     1    0.00    90.00
5 Rapinoe    Megan  7-Jul     1    1       3     2    0.00    83.27
6   Lloyd    Carli  7-Jul     0    0       1     0   83.27    90.00
7    Dunn  Crystal 23-Jun     0    1       1     2   42.62    81.08
\end{Verbatim}
\end{samepage}

This is much less unwieldy than dealing with minutes and seconds separately.

\section{Transforming with simple operations}

\index{transforming}

\index{mins\_played@\texttt{mins\_played}}

Now that we've converted the awkward minutes-and-seconds columns to just
``\texttt{time}'' columns, all we need to do to complete our analysis is
\textbf{transform} this data by computing a new quantity entirely: the
\textit{total number of minutes played} for each player in each game. Again,
Pandas makes this easy:

\begin{Verbatim}[fontsize=\small,samepage=true,frame=single,framesep=3mm]
wc['mins_played'] = wc['out_time'] - wc['in_time']
print(wc)
\end{Verbatim}
\vspace{-.2in}

\begin{Verbatim}[fontsize=\footnotesize,samepage=true,frame=leftline,framesep=5mm,framerule=1mm]
     last    first   date goals asst tackles fouls in_time out_time mins_played
0  Morgan     Alex 28-Jun     0    0       2     1    0.00    90.00       90.00
1 Rapinoe    Megan 28-Jun     2    0       2     2    0.00    74.45       74.45
2   Press Christen 28-Jun     0    0       1     0   74.45    90.00       15.55
3 Lavelle     Rose 28-Jun     0    1       3     0    0.00    90.00       90.00
4 Lavelle     Rose  7-Jul     1    0       4     1    0.00    90.00       90.00
5 Rapinoe    Megan  7-Jul     1    1       3     2    0.00    83.27       83.27
6   Lloyd    Carli  7-Jul     0    0       1     0   83.27    90.00        6.73
7    Dunn  Crystal 23-Jun     0    1       1     2   42.62    81.08       38.46
\end{Verbatim}

\index{tackles-per-game}
\index{tkl\_per\_90@\texttt{tkl\_per\_90}}

Voil\`{a}. We now have the time-on-field for each player, which gives us a
whole new avenue of exploration. For example, any of the counting stats (goals,
assists, \textit{etc.}) can be converted into a ``per-minute'' version, showing
us how productive a player was while on the field. Let's do that for
\texttt{tackles}, and multiply by 90 to obtain a ``tackles-per-90-minutes''
statistic\footnote{I'm choosing 90 minutes here because that's how long a
regulation-length soccer match is. Therefore, our new \texttt{tkl\_per\_90}
column gives us ``number-of-tackles-per-complete-game,'' which is easier to
interpret than ``tackles-per-\textit{minute},'' which would be a miniscule
number for any player.}:

\begin{Verbatim}[fontsize=\small,samepage=true,frame=single,framesep=3mm]
wc['mins_played'] = wc['out_time'] - wc['in_time']
wc['tkl_per_90'] = np.round(wc['tackles'] / wc['mins_played'] * 90,2)
del wc['tackles']
\end{Verbatim}
\vspace{-.2in}

\begin{Verbatim}[fontsize=\footnotesize,samepage=true,frame=leftline,framesep=5mm,framerule=1mm]
     last    first   date goals asst fouls in_time out_time mins_played tkl_per_90
0  Morgan     Alex 28-Jun     0    0     1    0.00    90.00       90.00       2.00
1 Rapinoe    Megan 28-Jun     2    0     2    0.00    74.45       74.45       2.42
2   Press Christen 28-Jun     0    0     0   74.45    90.00       15.55       5.79
3 Lavelle     Rose 28-Jun     0    1     0    0.00    90.00       90.00       3.00
4 Lavelle     Rose  7-Jul     1    0     1    0.00    90.00       90.00       4.00
5 Rapinoe    Megan  7-Jul     1    1     2    0.00    83.27       83.27       3.24
6   Lloyd    Carli  7-Jul     0    0     0   83.27    90.00        6.73      13.37
7    Dunn  Crystal 23-Jun     0    1     2   42.62    81.08       38.46       2.34
\end{Verbatim}

\subsection{Transforming grouped data}

The above example computed tackles-per-game all right, but it still left us
with one row for every player-performance. (In other words, the results had
\textit{two} rows for Rose Lavelle, one giving her \texttt{tkl\_per\_90} for
the June 28th game, and one giving it for the July 7th game.)

\index{groupby@\texttt{.groupby()} method (Pandas)}

We might instead be interested in a player-by-player analysis: overall in the
entire month-long World Cup, which players had the most tackles-per-game? This
is easy to do with the \texttt{.groupby()} method that we first encountered in
section~\ref{groupby} (p.~\pageref{groupby}). First, we group the rows by the
first \textit{two} columns (since first-and-last-names-together are needed to
uniquely identify a single player):

\index{grouped\_wc@\texttt{grouped\_wc}}

\begin{Verbatim}[fontsize=\footnotesize,samepage=true,frame=single,framesep=3mm]
grouped_wc = wc.groupby(['last','first'])
\end{Verbatim}

We then take our new, temporary \texttt{grouped\_wc} variable and extract the
\texttt{goals}, \texttt{asst}, \texttt{fouls}, \texttt{tackles}, and
\texttt{mins\_played} columns from it, \textbf{summing} each of them to produce
the per-player values in the result:

\index{by\_player@\texttt{by\_player}}

\begin{Verbatim}[fontsize=\footnotesize,samepage=true,frame=single,framesep=3mm]
by_player = grouped_wc[['goals','asst','fouls','tackles','mins_played']].sum()
\end{Verbatim}

This yields:

\begin{Verbatim}[fontsize=\small,samepage=true,frame=leftline,framesep=5mm,framerule=1mm]
                  goals  asst  fouls  tackles  mins_played
last    first                                             
Dunn    Crystal       0     1      2        1        38.46
Lavelle Rose          1     1      1        7       180.00
Lloyd   Carli         0     0      0        1         6.73
Morgan  Alex          0     0      1        2        90.00
Press   Christen      0     0      0        1        15.55
Rapinoe Megan         3     1      4        5       157.72
\end{Verbatim}

Now, we're ready to compute a per-game analysis as before, but this time for
each player's entire World Cup games:

\begin{Verbatim}[fontsize=\small,samepage=true,frame=single,framesep=3mm]
by_player['tkl_per_90'] = (np.round(by_player['tackles'] /
    by_player['mins_played'] * 90,2))
del by_player['tackles']
\end{Verbatim}
\vspace{-.2in}

\begin{Verbatim}[fontsize=\small,samepage=true,frame=leftline,framesep=5mm,framerule=1mm]
                  goals  asst  fouls  mins_played  tkl_per_90
last    first                                                
Dunn    Crystal       0     1      2        38.46        2.34
Lavelle Rose          1     1      1       180.00        3.50
Lloyd   Carli         0     0      0         6.73       13.37
Morgan  Alex          0     0      1        90.00        2.00
Press   Christen      0     0      0        15.55        5.79
Rapinoe Megan         3     1      4       157.72        2.85
\end{Verbatim}


\index{starter@\texttt{starter}}
\index{where@\texttt{where()} function (NumPy)}



% START HERE: instead of np.where, use a function with an if statement.

Before removing the \texttt{in\_time} and \texttt{out\_time} columns (which we
could keep around, but I want this to fit on the page) let's do something extra
snazzy. We might also be interested in which players \textit{started} which
games (as opposed to being a mid-game substitute). Obviously, a starter is
someone who entered the game at time 0. To create a new column for this, let's
use the NumPy \texttt{where()} function, which I find immeasurably useful.

The \texttt{np.where()} function works as follows. You give it three arguments.
The first is a query condition, using the same query syntax we learned back in
section~\ref{seriesQueries}. The second and third arguments are the values you
want to use \textit{if the query evaluates to \texttt{True}, or \texttt{False},
respectively}, for a particular row. Check it out:

\begin{samepage}
\begin{Verbatim}[fontsize=\small,samepage=true,frame=single,framesep=3mm]
wc['starter'] = np.where(wc['in_time'] == 0, True, False)
del wc['in_time']
del wc['out_time']
print(wc)
\end{Verbatim}
\end{samepage}

\begin{Verbatim}[fontsize=\footnotesize,samepage=true,frame=leftline,framesep=5mm,framerule=1mm]
     last    first   date goals asst tackles fouls mins_played starter
0  Morgan     Alex 28-Jun     0    0       2     1       90.00    True
1 Rapinoe    Megan 28-Jun     2    0       2     2       74.45    True
2   Press Christen 28-Jun     0    0       1     0       15.55   False
3 Lavelle     Rose 28-Jun     0    1       3     0       90.00    True
4 Lavelle     Rose  7-Jul     1    0       4     1       90.00    True
5 Rapinoe    Megan  7-Jul     1    1       3     2       83.27    True
6   Lloyd    Carli  7-Jul     0    0       1     0        6.73   False
7    Dunn  Crystal 23-Jun     0    1       1     2       38.46   False
\end{Verbatim}

Pretty cool, huh? The original \texttt{DataFrame} had the information we
wanted, but not in the form we really needed it. What we wanted was not the
entry time and exit time of each player (both in minutes and seconds) but
rather the total time that player was on the pitch, and whether or not they
started the game.

Recoding and transforming turn out to be common tasks for a simple reason:
\textit{whoever collects a data set can rarely predict how an analyst will
eventually use it.} We're very grateful to the author of the \texttt{.csv}
file, since it contains the raw material we need to evaluate our team's
performances; but how were they to know that length-of-time-on-the-field and
who-started-which-game was going to be important to us? They couldn't. But
thanks to recoding and transformation skills, we can cope.

