
\chapter[Tables in Python (1 of 3)]{\huge\selectfont{Tables in Python (1 of
3)}}

\label{tablesInPython1}
\label{dataframes}

\index{table}
\index{DataFrame@\texttt{DataFrame} (Pandas)}

The third of our three aggregate data types from waaaay back in
Chapter~\ref{ch:aggregateData} was the \textbf{table}. Don't worry: we haven't
forgotten about him. In this chapter, we'll implement him by means of the
Pandas \texttt{DataFrame}, the most important data type in this entire book.

\section{Reading a \texttt{DataFrame} from a \texttt{.csv} file}

\index{CSV@CSV (comma-separated values format)}

Unlike NumPy arrays and Pandas \texttt{Series}es, which we learned several
different ways to create, we're only going to learn one way to create a
\texttt{DataFrame}. That's because \texttt{DataFrame}s are normally big enough
that it's just too tedious to ever type them in manually. Instead, we'll read
them from an external source; a \texttt{.csv} file.

\index{read\_csv@\texttt{read\_csv()} function (Pandas)}
\index{davieses@\texttt{davieses}}

We'll actually use the same \texttt{read\_csv()} function that we used in
section~\ref{read_csv} (p.~\pageref{read_csv}), although oddly, this time we
won't need to specify as many arguments. Let's say we have a
``\texttt{davieses.csv}'' file with these contents:

\index{file!\texttt{.csv}}
\vspace{-.2in}

\begin{Verbatim}[fontsize=\small,samepage=true,frame=lines,framesep=3mm]
person,age,gender,height,instrument
Dad,50,M,73,piano
Mom,49,F,66,flute
Lizzy,19,F,63,guitar
TJ,18,M,71,trombone
Johnny,15,M,69,euphonium
\end{Verbatim}

We can read it into a \texttt{DataFrame} with this code:

\begin{Verbatim}[fontsize=\footnotesize,samepage=true,frame=single,framesep=3mm]
my_first_df = pd.read_csv("davieses.csv").set_index('person')
print(my_first_df)
\end{Verbatim}
\vspace{-.2in}

\begin{Verbatim}[fontsize=\small,samepage=true,frame=leftline,framesep=5mm,framerule=1mm]
        age gender  height instrument
person                               
Dad      50      M      73      piano
Mom      49      F      66      flute
Lizzy    19      F      63     guitar
TJ       18      M      71   trombone
Johnny   15      M      69  euphonium
\end{Verbatim}

\index{header row (of a \texttt{.csv} file)}
\index{row (of a table)}

A couple things. First, you may have noticed that the \texttt{davieses.csv}
file had a ``header'' row. This means that the first line of the file is not
like the others: instead of containing information on a specific family member,
it contains the \textit{kind} of information for \textit{every} family member.
It looked like this:

\vspace{-.1in}
\begin{center}
\texttt{person,age,gender,height,instrument}
\end{center}
\vspace{-.1in}

\index{column (of a table)}
\index{metadata}

and you'll notice that these words (except for the first one; more on that in a
moment) became the \textit{column names} when we imported the data. This sort
of information, by the way, is called ``\textbf{metadata},'' a geeky-sounding
word that basically means ``data about data.'' If ``Lizzy plays the guitar'' is
a piece of data, then ``family members play instruments'' is a piece of
\textit{meta}data.

\index{read\_csv@\texttt{read\_csv()} function (Pandas)}
\index{set\_index@\texttt{.set\_index()} method (Pandas)}

% TODO: make clear that .set_index() is optional.
Second, don't miss the ending I tacked on to the \texttt{read\_csv()} line,
where I called the \texttt{.set\_index()} method on the \texttt{DataFrame}.
This tells Pandas that one of the columns in the \texttt{DataFrame} should be
designated as the \textbf{index} (or the \textbf{keys}).

\index{uniqueness!of values in \texttt{DataFrame} index}

Back on p.~\pageref{tablesHaveNoKey} I asserted that unlike associative arrays,
tables didn't have keys. And that's true of the general ``table'' concept. But
Pandas designed their \texttt{DataFrame}s to behave in the same way as their
\texttt{Series}es: one uniquely-valued column will be used to identify each
row.

\index{Foreman, George}
\index{George}

This choice is usually easy; if you glance back to Figure~\ref{fig:table}
(p.~\pageref{fig:table}), we'd probably want to choose the \textsf{screenname}
as the index (although a case could be made for the \textsf{real name} column
instead). For the table in Figure~\ref{fig:aggregateMemory}
(p.~\pageref{fig:aggregateMemory}), it would be the \textsf{item} column. In
the \texttt{DataFrame} we just created above, obviously \texttt{person} is the
correct choice -- it's the only one sure to be unique.\footnote{With apologies
to boxing legend George Foreman, who named all four of his sons ``George.''}

Anyway, designating a column as the index in this way sort of removes it from
the other, ``ordinary'' columns. In the output, above, you may notice that the
word ``\texttt{person}'' is printed somewhat lower than the other column names
are. It turns out that if we want to talk about the index column specifically,
we'll need to use a slightly different technique than we do for the other
columns. More on that next chapter.

\section{Missing values}

Let's change the example to a different family, and a slightly bigger
\texttt{DataFrame}. The ``\texttt{simpsons.csv}'' file is reproduced below. Do
you notice anything odd about it?

\index{simpsons@The Simpsons}
\index{Santa's Little Helper}
\index{file!\texttt{.csv}}

\vspace{-.15in}
\begin{Verbatim}[fontsize=\small,samepage=true,frame=lines,framesep=3mm]
name,species,age,gender,fave,IQ,hair,salary
Homer,human,36,M,beer,74,,52000
Marge,human,34,F,helping others,120,stacked tall,
Bart,human,10,M,skateboard,90,buzz,
Lisa,human,8,F,saxophone,200,curly,
Maggie,human,1,F,pacifier,100,curly,
SLH,dog,4,M,,,shaggy,
\end{Verbatim}
\vspace{-.15in}

\index{double comma@``double comma''}

What I mean is the positioning of some of the commas. The sharp-eyed reader
will see a ``double comma'' in Homer's row. Even a dull-eyed reader will notice
several commas in a row in SLH's\footnote{The Simpson's dog was named ``Santa's
Little Helper.''} row. And nearly \textit{every} row (the exception being
Homer's) \textit{ends} with a comma, which just looks messed up.

\index{missing value}

This weird punctuation implies the existence of \textbf{missing values}, which
means just what it sounds like: there's simply no data for certain columns of 
certain rows. Homer doesn't have a ``\texttt{hair}'' value, no one \textit{but}
Homer has a ``\texttt{salary}'' value, and SLH is missing all kinds of stuff.

When we read this into a Pandas \texttt{DataFrame} a la:

\begin{Verbatim}[fontsize=\small,samepage=true,frame=single,framesep=3mm]
simpsons = pd.read_csv("simpsons.csv").set_index('name')
\end{Verbatim}

the result looks like this:

\begin{Verbatim}[fontsize=\scriptsize,samepage=true,frame=leftline,framesep=5mm,framerule=1mm]
       species  age gender            fave     IQ          hair   salary
name                                                                    
Homer    human   36      M            beer   74.0           NaN  52000.0
Marge    human   34      F  helping others  120.0  stacked tall      NaN
Bart     human   10      M      skateboard   90.0          buzz      NaN
Lisa     human    8      F       saxophone  200.0         curly      NaN
Maggie   human    1      F        pacifier  100.0         curly      NaN
SLH        dog    4      M             NaN    NaN        shaggy      NaN
\end{Verbatim}

\index{nan@\texttt{NaN} (``not a number'')}

The missing values come up as \texttt{NaN}'s, the same value you may remember
from p.~\pageref{NaN}. The monker ``not a number'' makes sense for the
\texttt{salary} case, although I think it's a bit weird for Homer's
\texttt{hair} (not a number? is hair \textit{supposed} to be a number?...)
At any rate, we can expect that this will be the case for many real-world data
sets.

\index{objects (of a study)}

``Missing'' can mean quite a few subtly different things, actually. Maybe it
means that the value for that object of study was collected, but lost. Maybe it
means it was never collected at all. Maybe it means that variable doesn't
really make \textit{sense} for that object, as in the case of a dog's IQ.
Ultimately, if we want to use the other values in that row, we'll have to come
to terms with what the missing values \textit{mean}. For now, let's just learn
a couple of coarse ways of dealing with them.

\index{dropna@\texttt{.dropna()} method (Pandas)}

One (sometimes) handy method is \texttt{.dropna()}. If you call it, it will
return a modified copy of the \texttt{DataFrame} in which any row with an
\texttt{NaN} is removed. This turns out to be overkill in the Simpson's case,
though:

\begin{Verbatim}[fontsize=\small,samepage=true,frame=single,framesep=3mm]
print(simpsons.dropna())
\end{Verbatim}
\vspace{-.2in}

\begin{Verbatim}[fontsize=\small,samepage=true,frame=leftline,framesep=5mm,framerule=1mm]
Empty DataFrame
Columns: [species, age, gender, fave, IQ, hair, salary]
Index: []
\end{Verbatim}

In other words, nothing's left. (Every row had at least one \texttt{NaN} in
it, so nothing survived.)

We could pass an optional argument to \texttt{.dropna()} called
``\texttt{how}'', and set it equal to \texttt{"all"}: in this case only rows
with \textit{all} \texttt{NaN} values are removed. Sometimes that's
``underkill,'' as in our Simpson's example: after all, none of the rows are
\textit{entirely} \texttt{NaN}'s, so calling \texttt{.dropna(how="all")}
would leave everything intact.

\index{fillna@\texttt{.fillna()} method (Pandas)}
\index{default value}

Another option is the \texttt{.fillna()} method, which takes a ``default
value'' argument: any \texttt{NaN} value is replaced with the default in the
modified copy returned. Let's try it with the string \texttt{"none"} as the
default value:

\begin{Verbatim}[fontsize=\small,samepage=true,frame=single,framesep=3mm]
print(simpsons.fillna("none"))
\end{Verbatim}
\vspace{-.2in}

\begin{Verbatim}[fontsize=\scriptsize,samepage=true,frame=leftline,framesep=5mm,framerule=1mm]
       species  age gender            fave    IQ          hair salary
name                                                                 
Homer    human   36      M            beer    74          none  52000
Marge    human   34      F  helping others   120  stacked tall   none
Bart     human   10      M      skateboard    90          buzz   none
Lisa     human    8      F       saxophone   200         curly   none
Maggie   human    1      F        pacifier   100         curly   none
SLH        dog    4      M            none  none        shaggy   none
\end{Verbatim}

This is possibly useful, but in this case it's not a perfect fit because
different columns call for different defaults. The \texttt{fave} and
\texttt{hair} columns could well have ``\texttt{none}'' (indicating no
favorite thing, and no hair, respectively) but we might want the default
\texttt{salary} to be 0. The way to accomplish that is to change the individual
columns of the \texttt{DataFrame}. Here goes:

\begin{Verbatim}[fontsize=\small,samepage=true,frame=single,framesep=3mm]
simpsons['salary'] = simpsons['salary'].fillna(0)
simpsons['IQ'] = simpsons['IQ'].fillna(100)
simpsons['hair'] = simpsons['hair'].fillna("none")
print(simpsons)
\end{Verbatim}
\vspace{-.2in}

\label{finalSimpsons}

\begin{Verbatim}[fontsize=\scriptsize,samepage=true,frame=leftline,framesep=5mm,framerule=1mm]
       species  age gender            fave     IQ          hair   salary
name                                                                    
Homer    human   36      M            beer   74.0          none  52000.0
Marge    human   34      F  helping others  120.0  stacked tall      0.0
Bart     human   10      M      skateboard   90.0          buzz      0.0
Lisa     human    8      F       saxophone  200.0         curly      0.0
Maggie   human    1      F        pacifier  100.0         curly      0.0
SLH        dog    4      M             NaN  100.0        shaggy      0.0
\end{Verbatim}

Here we've assumed that the default IQ, for someone who hasn't taken the test,
is 100 (the average). I left the \texttt{NaN} in \texttt{fave} as is, since
that seemed appropriate.

\begin{samepage}
By the way, that code is actually more than it may appear at first. When we
execute a line like:

\vspace{-.1in}
\begin{center}
\texttt{simpsons[\textquotesingle salary\textquotesingle] =
simpsons[\textquotesingle salary\textquotesingle].fillna(0)}
\end{center}
\vspace{-.1in}
\end{samepage}

we're really saying ``please \textit{replace} the \texttt{salary} column of the
\texttt{simpsons} \texttt{DataFrame} with a new column. That new column should
be -- wait for it -- the existing \texttt{salary} column but with zeros
replacing the \texttt{NaN}'s.''

We'll see many more cases of changing \texttt{DataFrame} columns wholesale in
the following chapters.

\section{Removing rows/columns}

Finally, after reading a \texttt{.csv} file into a \texttt{DataFrame}, there
are times when you want to manually delete certain rows and/or columns that are
not going to be of interest.


\begin{samepage}

\index{drop@\texttt{.drop()} method (Pandas)}
\index{Santa's Little Helper}
The easiest syntax for deleting a row (say, Santa's Little Helper) is:

\begin{Verbatim}[fontsize=\small,samepage=true,frame=single,framesep=3mm]
simpsons = simpsons.drop('SLH')
\end{Verbatim}
\end{samepage}

The \texttt{.drop()} method takes the index of the undesired row as an
argument, Like most of the methods we've seen so far, it returns a modified
copy of the \texttt{DataFrame} it's called on, so you have to reassign this to
the original variable (or use the \texttt{inplace=True} argument).

\index{boxies (square brackets)}
\index{[]@\texttt{[]} (boxies)}

\begin{samepage}
You can even delete multiple rows at the same time by enclosing the undesired
indices in boxies:

\begin{Verbatim}[fontsize=\small,samepage=true,frame=single,framesep=3mm]
simpsons = simpsons.drop(['Homer','Marge','SLH'])
\end{Verbatim}
\end{samepage}

\begin{samepage}

\index{del operator@\texttt{del} operator (Pandas)}
Deleting a column is even more common, since many tables ``in the wild'' have
many, many columns, only a few of which you may care about in your analysis.
You can whack one entirely with the \texttt{del} operator, just like we did for
\texttt{Series}es (p.~\pageref{delOp}):

\begin{Verbatim}[fontsize=\small,samepage=true,frame=single,framesep=3mm]
del simpsons['IQ']
\end{Verbatim}
\end{samepage}
