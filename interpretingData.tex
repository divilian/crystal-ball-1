
\chapter{Interpreting Data}

Let's take an intermission from the nitty-gritty Python stuff and talk about
how to properly \textit{interpret} the data we're working with; specifically,
how to draw correct conclusions from what we've collected.

\section{Independent and dependent variables}

\index{variable!dependent}
\index{variable!independent}
\index{independent variable}
\index{dependent variable}
\index{cause}
\index{causal}
\index{hiccup}
\index{orgasm}
\index{greenhouse gas}
\index{smoking}
\index{lung cancer}
\index{global warming}
You've undoubtedly seen countless studies that claim to reveal important truths
about the world, such as that smoking can cause lung cancer, greenhouse gas
emissions can cause higher global temperatures, or orgasms can cure hiccups.
Much of the time, scientists try to find a \textbf{causal} factor that links
one variable to another: they suspect that the value of a variable $A$ (often
called the \textbf{independent variable}) is a \textit{reason}, or
\textbf{cause}, of a certain value in another variable $B$ (the
\textbf{dependent variable}).

\index{variable}
\index{objects (of a study)}

Importantly, we're using the word \textbf{variable} here in a different, but
related way than we used it in chapters~\ref{ch:atomicData},
\ref{ch:arraysInPython1}, and \ref{ch:arraysInPython2}. As we did in
chapter~\ref{ch:scalesOfMeasure} (see p.~\pageref{variableDifferent} footnote),
we use ``variable'' here to mean a specific aspect of the \textbf{objects of a
study} that can differ, or ``vary.'' The objects in our study (often people,
but sometimes companies, organizations, environments, nations, \textit{etc.})
\textit{each} have a value for the variable. Thus if you think of a
``per-capita income'' variable, you might think of an entire \textit{array} of
floats, each of which represented the average income per resident of a single
nation (measured in dioxide emissions of a single nation (measured in
gigatonnes $\textrm{CO}_2$).

\index{scales of measure}
\index{categorical variable}

The variables in question can be from any of the scales of measure from
chapter~\ref{ch:scalesOfMeasure}. Take the smoking example, with patients as
the object of study. We might say that independent variable $A$ is categorical,
with values \texttt{SMOKER} and \texttt{NON-SMOKER}. The dependent variable $B$
is also categorical: \texttt{CANCER} and \texttt{NO-CANCER}. The key question
is: do people with $A=\texttt{SMOKER}$ also have $B=\texttt{CANCER}$
\textit{more often} (a higher percentage of the time) than people with
$A=\texttt{NON-SMOKER}$ do?

\index{ratio variable}
\index{interval variable}

In the greenhouse gas emissions example, our objects of study might be
\textit{years}. Our variables are numeric, with $A$ (a measure of yearly
greenhouse gas emissions, measured in gigatonnes $\textrm{CO}_2$) on the ratio
scale, and $B$ (average worldwide temperature increase/decrease) on an interval
scale. Here, the question would be: do years in which $A$ is relatively high
more often also have $B$ relatively high? Put another way: do years in which
earthlings have released more gas into the atmosphere tend to correspond with
years in which the temperature increased?

And of course, we might have one categorical variable and one numeric. Perhaps
our objects of study are American adults, and while our categorical $A$
variable has values \texttt{DEMOCRAT}, \texttt{REPUBLICAN}, \texttt{OTHER}, and
\texttt{INDEPENDENT}, our numerical $B$ is yearly income. Our question would
be: do adherents of one political party tend to be more wealthy than those of
another?

Or, flipping sides, the independent variable $A$ could be numeric while the
dependent variable $B$ is categorical. Our objects of study might be high
school seniors applying to UMW. Let $A$ be the number of different colleges a
student applied to, and $B$ a categorical variable with values
\texttt{ADMITTED-TO-UMW} and \texttt{NOT-ADMITTED-TO-UMW}. The question of
interest is here is: do students who apply to more colleges tend to get in to
UMW more often?

\section{Association and causality}

All of the above questions can be answered with data. In future chapters, we'll
learn the exact Python commands to ask them, and how to interpret the answers.

\index{association}
\index{correlation}
\index{dependent}

For now, I merely want to draw your attention to the fact that these are all
questions of \textbf{association}, not causation. An association between
variables merely means that they are \textbf{correlated} in some way
statistically.\footnote{Another way to put this is to say that the variables
are \textbf{dependent} on each other, although this is confusing because we're
already using the word ``dependent'' to refer to one of the variables.} If
$A=\texttt{SMOKER}$ goes with $B=\texttt{CANCER}$ more often than
$A=\texttt{NON-SMOKER}$ does, then there \textit{is} an association between the
two, period. If yearly income $B$ is on average higher for
$A=\texttt{REPUBLICAN}$ than for $A=\texttt{DEMOCRAT}$, then there \textit{is}
an association between the two, period.

(By the way, a key nuance will turn out to be: \textit{how much} more often
does $A=\texttt{SMOKER}$ need to go with $B=\texttt{CANCER}$ in order for us to
be confident that there is a true association? Or \textit{how much} more
wealthy do the $A=\texttt{REPUBLICAN}$s need to be on average for us to have
confidence? That one's a little tricky, and we'll postpone addressing it for
now.)

\index{causality}

So anyway, the question of association turns out to be pretty straightforward
to answer. Python will simply tell us if variables are associated or not. More
difficult, however, is determining \textbf{causality}
(a.k.a.~\textbf{causation}). Does a person's political affiliation influence
how much wealth they have? Or is it the other way around: does a person's
wealth cause them to vote a certain way? Or is it neither of these, with some
third factor (perhaps values, or life philosophy) helping determine
\textit{both} variables?

\index{$\rightarrow$@$\rightarrow$ (causality)}

If the first of these three is the case, we would write ``$A \rightarrow B$,''
pronounced ``$A$ causes $B$''. If the second, we'd write, ``$B \rightarrow
A$,'' and for the third, we'd write ``$C \rightarrow A, B$'' for some other
(possibly yet to be determined) variable $C$. Determining which (if any) of
these is true calls for some careful thinking, intuition, and additional kinds
of statistical tests.

In fact, just to blow your mind, Figure~\ref{fig:causalityTypes} gives a
partial list of the various types of causation that \textit{could} be the true
explanation, once we find out that $A$ and $B$ have an association.

\begin{figure}[ht]
\small
\centering
\begin{tabular}{|c|c|p{3.3in}|}
\hline
Symbology & Name & Example \\
\hline
\multirow[c]{2}{*}{$A \rightarrow B$} & \multirow[c]{2}{*}{causation} & Regular
exercise does indeed normally lead to a lower resting heart rate. \\
\hline
\multirow[c]{2}{*}{$B \rightarrow A$} & \multirow[c]{2}{*}{reverse causation} & Smoking doesn't cause depression;
depression causes smoking. \\
\hline
\multirow[c]{2}{*}{$C \rightarrow A, B$} & \multirow[c]{2}{*}{\makecell{external causation \\ (confounding
factor)}} & Ice cream sales don't cause shark attacks; high temperatures boost both ice
cream sales and ocean swimming. \\
\hline
\multirow[c]{2}{*}{$A \rightarrow B$ \& $B \rightarrow C$} &
\multirow[c]{2}{*}{multiple causation }& A liberal arts
education does improve critical thinking skills, but lots of other things do
too. \\
\hline
\multirow[c]{2.5}{*}{$A,C \rightarrow B$} & \multirow[c]{2.5}{*}{joint causation} &
Just being tall doesn't necessarily make you a good basketball player, but if
your height is accompanied by another factor as well (athleticism), then you
will be. \\
\hline
\multirow[c]{2.5}{*}{$A \rightarrow C \rightarrow B$} &
\multirow[c]{2.5}{*}{indirect causation} & People who use antiperspirant tend to
get more dates, but it's not because of the antiperspirant \textit{per se};
it's because they don't have an unpleasant odor. \\
\hline
\multirow[c]{3}{*}{$A \not\rightarrow B$} & \multirow[c]{3}{*}{spurious
association} & Although for many years the outcome of the Washington Redskins
game immediately preceding a Presidential election predicted the election's
outcome, that was by coincidence. \\
\hline
\end{tabular}
\medskip
\caption{Various types of causality that could be the underlying reason why an
association between $A$ and $B$ exists.}
\label{fig:causalityTypes}
\normalsize
\end{figure}

\index{variable!confounding}
\index{confounding variable}
