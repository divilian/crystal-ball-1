
\chapter{Interpreting Data}

Let's take an intermission from the nitty-gritty Python stuff and talk about
how to properly \textit{interpret} the data we're working with; specifically,
how to draw correct conclusions from what we've collected.

\index{variable!dependent}
\index{variable!independent}
\index{independent variable}
\index{dependent variable}
\index{cause}
\index{causal}
\index{hiccup}
\index{orgasm}
\index{greenhouse gas}
\index{smoking}
\index{lung cancer}
\index{global warming}
You've undoubtedly seen countless studies that claim to reveal important truths
about the world, such as that smoking can cause lung cancer, greenhouse gas
emissions can cause higher global temperatures, or orgasms can cure hiccups.
Much of the time, scientists try to find a \textbf{causal} factor that links
one variable to another: they suspect that the value of a variable $A$ (often
called the \textbf{independent variable}) is a \textit{reason}, or
\textbf{cause}, of a certain value in another variable $B$.

\index{variable}

Importantly, we're using the word \textbf{variable} here in a different, but
related way than we used it in chapters~\ref{ch:atomicData},
\ref{ch:arraysInPython1}, and \ref{ch:arraysInPython2}. As we did in
chapter~\ref{ch:scalesOfMeasure} (see p.~\pageref{variableDifferent} footnote),
we use ``variable'' here to mean a specific aspect of the objects in our study
that can differ, or ``vary.'' The objects of our study (often people, but
sometimes companies, organizations, environments, nations, \textit{etc.})
\textit{each} have a value for the variable. Thus if you think of a
``greenhouse gas emissions'' variable, you might think of an entire
\textit{array} of floats, each of which represented the total yearly carbon
dioxide emissions of a single nation (measured in gigatonnes $\textrm{CO}_2$).

% This could be framed in several different ways. Take the smoking example. We
% might say that independent variable $A$ is 


\index{variable!confounding}
\index{confounding variable}
