
\chapter{Branching}
\label{ch:branching}

\index{non-linear}
\index{branching}
\index{conditional execution}

In this chapter, we'll learn our next programming trick: how to execute code
\textit{conditionally}. This is called \textbf{branching}. It's another variant
of non-linear programming, like the loops from chapter~\ref{ch:loops}: it
enables something other than the strict, start-to-finish, line-by-line
execution of a program. In particular, branching allow us to designate certain
lines of code to be executed ``only sometimes.''

\section{The \texttt{if} statement}

\index{if statement@\texttt{if} statement}

The main branching statement in Python and most languages is the
\textbf{\texttt{if} statement}. Here's a couple of them in action:

\index{cash\_on\_hand@\texttt{cash\_on\_hand}}

\begin{Verbatim}[fontsize=\small,samepage=true,frame=single,framesep=3mm]
 1: name = "Horace"
 2: cash_on_hand = 100000
 3: IQ = 90
 4: print("Nice to meet you, {}!".format(name))
 5: if cash_on_hand > 5000:
 6:     print("Wow, you're rich! Gimme a fiver.")
 7:     cash_on_hand = cash_on_hand - 5
 8: if IQ > 100:
 9:     print("Wow, you're smart! Read a book.")
10:     IQ = IQ + 5
11: print("{}'s IQ is {} and he has ${}.".format(name, IQ, cash_on_hand))
\end{Verbatim}
\vspace{-.2in}

Even without any explanation, you might be able to figure out that the output
of the code snippet above is:

\begin{Verbatim}[fontsize=\small,samepage=true,frame=leftline,framesep=5mm,framerule=1mm]
Nice to meet you, Horace!
Wow, you're rich! Gimme a fiver.
Horace's IQ is 90 and he has $99995.
\end{Verbatim}

If not, stay tuned.

\index{loop!header}
\index{loop!body}
\index{indentation}

Just like a \texttt{for} loop, every \texttt{if} statement has a header and a
body. And just like a \texttt{for} loop, the determining factor of which lines
constitute the body depends on the indentation:

\begin{compactitem}
\item[\leftpointright] The first \texttt{if} statement's header is line \textbf{5}.
\item[\leftpointright] The first \texttt{if} statement's body is lines \textbf{6 and 7}.
\item[\leftpointright] The second \texttt{if} statement's header is line \textbf{8}.
\item[\leftpointright] The second \texttt{if} statement's body is lines
\textbf{9 and 10}.
\end{compactitem}

\index{condition (of an \texttt{if} statement)}

When an \texttt{if} statement is reached, its \textbf{condition} is evaluated;
in the first case, the condition ``\texttt{cash\_on\_hand > 500}'' is evaluated
to \texttt{True}, and in the second case, ``\texttt{IQ > 100}'' is determined
to be \texttt{False}. Then, \textit{only if} the condition is true will the
body of the \texttt{if} statement execute. Otherwise, it'll be skipped over.

Thus, the lines of the above program execute in this order: 1, 2, 3, 4, 5, 6,
7, 8, 11. Lines 9 and 10 are skipped entirely, since Horace's IQ wasn't above
average. Observe that the \texttt{cash\_on\_hand} variable was updated inside
the body of the first \texttt{if} statement, but that \texttt{IQ} was not.

\subsection{Compound conditions}

\index{compound condition}

Conditions can be more complicated than the ones above; just as with queries
(p.~\pageref{seriesCompoundQueries}) they can contain more than one component:

\begin{Verbatim}[fontsize=\small,samepage=true,frame=single,framesep=3mm]
if cash_on_hand > 10000 and IQ < 50:
    print("Wow, some dumb people are sure rich!")
\end{Verbatim}


You might have been surprised to see the word ``\texttt{and}'' in that
\texttt{if} statement instead of the character ``\texttt{\&}''. I feel you.
It's totally inconsistent, but nevertheless true: although in a query, you must
use the symbols \texttt{\&}, \texttt{|}, and \texttt{\textasciitilde}, in an
\texttt{if} condition, you must use the words \texttt{and}, \texttt{or}, and
\texttt{not}. (In other news, the bananas around the components of an
\texttt{if} condition aren't necessary, but you can include them if you want.)

\index{double-equals (\texttt{==})}
\index{==@\texttt{==} (double-equals)}

For your convenience, the \texttt{if} condition operators are listed in
Figure~\ref{fig:ifStatementOps}. (\textbf{Remember} the double-equals!!)

% in and .isin()

\begin{figure}[ht]
\centering
\small
\bigskip
\begin{tabular}{c|c}
Symbol & Meaning \\
\hline
\texttt{>} & greater than \\
\hline
\texttt{<} & less than \\
\hline
\texttt{>=} & greater than or equal to \\
\hline
\texttt{<=} & less than or equal to \\
\hline
\texttt{!=} & \textit{not} equal to \\
\hline
\texttt{==} & equal to \\
\hline
\texttt{and} & and \\
\hline
\texttt{or} & or \\
\hline
\texttt{not} & not \\
\end{tabular}

\bigskip
\normalsize
\caption{Operators in \texttt{if} statements: simple and compound.}
\label{fig:ifStatementOps}
\end{figure}

% else

% elif

% nested
