
\chapter{Three kinds of atomic data}

\section{Atomic data}

When we say that some data is ``\textbf{atomic},'' we don't mean it's
radioactive; we mean it's \textit{indivisible}.

The ancients spoke of ``atoms'' as the smallest possible bits of matter. If you
divide up any physical object -- say, an apple -- into parts, you get its
components: a stalk, a stem, skin, seeds, and the sweet juicy stuff. Cut up any
of \textit{those} pieces with a knife and you get smaller pieces. If you
continue to split and split and split, philosophers like Democritus reasoned,
you'll eventually get to tiny indivisible bits that can not be further
dissected. This is where the physical world bottoms out at the finest degree of
granularity.

Similarly, a piece of atomic data is typically treated as an entire unit, not
as something with internal structure that can be broken down. In the next
chapter we'll learn about various ways that these atoms of data can be strung
together and organized into larger wholes; for now, though, we're just looking
at the atoms themselves.

\section{Environments and variables}

A data analysis program -- of which we will write many in this course -- makes
use of an \textbf{environment} as it runs. ``Environment'' just means ``all the
data that is currently in view, and which the program can access.'' The
environment consists of \textbf{variables}, each of which (usually) has a
\textbf{name} and a \textbf{value}. For example, we might have a variable named
\texttt{age} whose value is 21, and a variable named \texttt{slogan} whose
value is \texttt{"Finger lickin' good"}.

Each variable in the environment must have a \textit{distinct} name
(\textit{i.e.}, no two variables can share the same name). Also, importantly,
the reason these building blocks are called ``variables'' is that their value
can \textit{change} as the program executes. Although we may initially create
an \texttt{age} variable with the value \texttt{21}, later on in the program
the variable's value might change to \texttt{22}, or \texttt{50}, or
\texttt{0}. The variable's \textit{name} never changes, though.

\section{Atomic data types}

There's one other thing that a variable has in addition to its name and value:
a \textbf{type}.\footnote{Strictly speaking, although in languages like Java
variables indeed have types, in Python the \textit{values} have types, not the
variables. This distinction will never be important for us though.}  In a
programming language like Python, every piece of data has a specific type,
which is necessary for determining how it behaves and what all you can do to
it. A question you should ask yourself a lot is: ``okay, I've got a variable in
my environment called \texttt{x}...now what is its type?'' You might have
guessed (correctly) that our \texttt{age} and \texttt{slogan} variables from
the previous section are of different types: one is a number, and the other is
a phrase.

In this course, we'll principally deal with three types of atomic data, all of
which will be familiar to you.


\subsection{Whole numbers}

One very common type of data is whole numbers, or integers. These are usually
positive, but can be negative, and have no decimal point. Things like a
person's birth year, a candidate's vote total, or a social media post's number
of ``likes'' are represented with this data type.


\subsection{Real (fractional) numbers}

You may remember from high school math that the so-called ``real numbers''
include not only integers, but also numbers with digits after the decimal
point. This type can therefore be used to store interest rates, temperature
readings, and average movie ratings on a 1-to-5 scale.

Since all whole numbers are themselves real numbers, you might wonder why we
bother to define two different types for these. Why not just give both kinds of
variables the same real number type? Basically, the answer is that something
``feels wrong'' about that to the Data Science community. A Facebook user might
have 240 friends, or 241, but it would never make sense for them to have 240.3
friends. A consensus has thus arisen: variables that would only ever store
whole numbers really ought to be of a type that's devoted to only whole
numbers. You can violate this convention, but you'll be thought weird by your
colleagues if you do so.


\subsection{Text}

Lastly, some values obviously aren't numeric at all, like a customer's name, a
show title, or a tweet. So our third type of data is textual. Variables of this
type have a sequence of characters as values. These characters are most often
English letters, but can also include spaces, punctuation, and characters from
other alphabets.

By the way, this third data type can tiptoe right up to the ``atomic'' line and
sometimes cross it. In other words, we will occasionally work with text values
\textit{non}-atomically, by splitting them up into their constituent words or
even letters. Most of the time, though, we'll treat a character sequence like
\texttt{"Avengers:\ Endgame"} as a single, indivisible chunk just like we treat
a number like \texttt{42}.

What about other things a computer can store: images, song files, videos? It
turns out that through clever tricks, all these kinds of media and more can be
boiled down to a large number of integers, and stored in an aggregate data
structure like those discussed in the next chapter. At the atomic level, we'll
really only ever need to deal with the three types above.

\section{The three kinds in Python}

\subsection{Whole numbers: \texttt{int}}

\subsection{Real (fractional) numbers: \texttt{float}}

\subsection{Text: \texttt{string}}

