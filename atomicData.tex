
\chapter{Three kinds of atomic data}

\section{Atomic data types}

When we say that data is ``\textbf{atomic},'' we don't mean it's radioactive;
we mean it's \textit{indivisible}.

The ancients spoke of ``atoms'' as the smallest possible bits of matter. If you
divided up any physical object -- say, an apple -- into parts, you got its
components: a stalk, a stem, skin, seeds, and the sweet juicy stuff. Cut up
\textit{those} pieces with a knife and you get smaller pieces. If you continue
to split and split and split, philosophers like Democritus reasoned, you'd
eventually get to tiny indivisible bits that could not be further divided. This
is where the physical world bottomed out at the finest degree of granularity.

Similarly, a piece of atomic data is (typically) treated as an entire unit, not
as something with internal structure that can be broken down. In this course,
we'll principally deal with three of these, all of which will be familiar to
you.

\subsection{Whole numbers}

\subsection{Real (fractional) numbers}

\subsection{Text}

\section{The three kinds in Python}

\subsection{Whole numbers: \texttt{int}}

\subsection{Real (fractional) numbers: \texttt{float}}

\subsection{Text: \texttt{string}}

