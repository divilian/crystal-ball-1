
% +, -, *, /
% np.sqrt()
% np.round( , num_decimals)
% np query syntax
% np.append
% np.flip
% two kinds of sorting
% remove (by index  and by value?)

% indexing, slices

\chapter{Arrays in Python (2 of 2)}

\index{array!in NumPy}
\index{ndarray@\texttt{ndarray} (NumPy)}

Now that we know several options for how to \textit{create} \texttt{ndarrays},
what can we do with them? Many and sundry things.

\section{Getting the array size}

To learn how long an array is (\textit{i.e.}, how many elements) we use the
\texttt{len()} function, kind of like we did for strings. Refer back to
Figure~\ref{fig:arraysInMemory} (p.~\pageref{fig:arraysInMemory}) and consider
this code:

\begin{Verbatim}[fontsize=\small,samepage=true,frame=single,framesep=3mm]
num_players = len(roster)
big_number = len(photo_likes)
print("There are {} players on the USWNT.".format(num_players))
print(big_number)
print("We've had {} presidential elections.".format(len(prez_elections)))
\end{Verbatim}

\begin{Verbatim}[fontsize=\small,samepage=true,frame=single,framesep=3mm]
There are 24 players on the USWNT.
40000000000
We've had 59 elections for president.
\end{Verbatim}

\index{overloading}
\index{return@``returning'' a value}
\index{array!length}
\index{len@\texttt{len()}}
\index{element}
This is an example of Python \textbf{overloading} function names, which just
means that the same name is used for two different functions. When you pass a
string to \texttt{len()}, you get the number of characters; but when you pass
an array to \texttt{len()}, you get the number of elements it has.
(The \texttt{roster} array had way more than 24 \textit{letters} in it, notice
-- but \texttt{len()} returned 24 since that was the number of strings.)

\section{Accessing individual elements}

\subsection{Retrieving an element}

\index{boxies (square brackets)}
\index{[]@\texttt{[]} (boxies)}
\index{element}
To get the value of a specific element from an array, we use ``boxie notation''
with the index number:

\begin{Verbatim}[fontsize=\small,samepage=true,frame=single,framesep=3mm]
print(prez_elections[0])
third_year = usa_years[2]
print("{} was the third year of our country.".format(third_year))
print("The highest-numbered player is {}".format(roster[len(roster)-1]))
\end{Verbatim}

\begin{Verbatim}[fontsize=\small,samepage=true,frame=leftline,framesep=5mm,framerule=1mm]
1788.0
1778 was the third year of our country.
The highest-numbered player is Christen Press.
\end{Verbatim}

Remember, indices start at zero (not one) so that's why the first line has a
\texttt{0} in it.

Now examine that last line, which is kind of tricky. Whenever you have boxies,
you have to first evaluate the code \textit{inside} them to get a number. That
number is then the index number Python will look up in the array. In the last
line above, the code \textit{inside} the boxies is:

\quad\quad\quad   ...\texttt{len(roster)-1}...

Breaking it down, we know that \texttt{len(roster)} is 24, which means
\texttt{len(roster)-1} must be 23, and so \texttt{roster[len(roster)-1]} is
\texttt{Christen Press}. It's a common pattern to get the last element of an
array.\footnote{Fun fact: you can also use \textit{negative} indices to mean
``from the end of the array, rather than the beginning.'' So
\texttt{roster[-1]} will also give you \texttt{Christen Press},
\texttt{roster[-2]} is \texttt{Jessica McDonald}, 
\texttt{roster[-5]} is \texttt{Crystal Dunn}, \textit{etc.} (see
p.~\pageref{rosterNames} for the values). I find this a bit obscure, though, so
I don't normally use this feature. (Negative indices also mean a completely
different thing in the R language, which is another reason I eschew them in
both R and Python.)}

To test your understanding, figure out what the following code will print:

\label{indexTest}
\begin{Verbatim}[fontsize=\small,samepage=true,frame=single,framesep=3mm]
q = 2
r = np.array([45,17,39,99])
s = 1
print(r[q-s+1]+3)
\end{Verbatim}

The answer is at the end of the chapter.

\subsection{Changing an element}

To modify an element of an array, just use the equals sign like we do for
variables:

\begin{Verbatim}[fontsize=\small,samepage=true,frame=single,framesep=3mm]
stooges = np.array(['Larry','Beavis','Moe'])
print(stooges)
stooges[1] = 'Curly'
print(stooges)
\end{Verbatim}

\begin{Verbatim}[fontsize=\small,samepage=true,frame=leftline,framesep=5mm,framerule=1mm]
['Larry' 'Beavis' 'Moe']
['Larry' 'Curly' 'Moe']
\end{Verbatim}

After all, an individual element like \texttt{stooges[1]} is itself a variable
pretty much like any other.

% Omit: slices

\section{``Vectorized'' arithmetic operators}

Recall our table of Python math operators (Figure~\ref{fig:mathOps} on
p.~\pageref{fig:mathOps}). What do those things do if we use them on aggregate,
instead of atomic data? The answer is: something super cool and useful.

\subsection{Operating on an array and a single value}

Consider the following code:

\label{vectorizedArrayIntExample}
\begin{Verbatim}[fontsize=\small,samepage=true,frame=single]
num_likes_today = np.array([6,61,0,0,14])
num_likes_tomorrow = num_likes_today + 3
print(num_likes_tomorrow)
\end{Verbatim}

\begin{Verbatim}[fontsize=\small,samepage=true,frame=leftline,framesep=5mm,framerule=1mm]
[ 9 64 3 3 17 ]
\end{Verbatim}

See what happened? ``Adding 3'' to the \textit{array} means adding 3 \textit{to
each element}. All in one compact line of code, we can do five -- or even five
billion -- operations. This works for all the other Figure~\ref{fig:mathOps}
operators as well.

\index{vectorized@``vectorized'' operation}

For somewhat geeky reasons, this sort of thing is called a \textbf{vectorized}
operation. All you need to know is that this means \textbf{fast}. And that's
``fast'' in two different ways: fast to write the code (since instead of using
a \textbf{loop}, which we'll cover in \ref{ch:loops}, you just write a single
statement with \texttt{+} and \texttt{=} signs), and more importantly, fast to
\textit{execute}. For more geeky reasons, the above code will run lightning
fast even if \texttt{num\_likes\_today} had five hundred million elements
instead of just five. As you'll learn if you ever try it, a Python loop is much
slower.\footnote{I just ran that comparison on my laptop, and here are the
results. Using the plain-ol' ``\texttt{+}'' vectorized operator, my machine
added the number 3 to an array with five hundred million elements in just 1.51
seconds. A loop, by contrast, took 2.8 \textit{minutes} to do the same thing.}

Don't get me wrong: there are times we'll have to use a loop because we have no
choice. But the general rule with Python is: if you can figure out how to
perform a calculation without using a loop, always do it!


\subsection{Operating on two arrays}

\index{vectorized@``vectorized'' operation}

Possibly even cooler, we can even ``\texttt{+}'' (or ``\texttt{-}'', or
``\texttt{*}'', or...) two entire \textit{arrays} together. Example:

\label{vectorizedArrayArrayExample}
\begin{Verbatim}[fontsize=\small,samepage=true,frame=single,framesep=3mm]
salaries = np.array([38000, 102000, 55750, 29500, 250000])
raises = np.array([1000, 4000, 2000, 1000, 2000])
salaries = salaries + raises
print(salaries)
\end{Verbatim}

This code produces:

\begin{Verbatim}[fontsize=\small,samepage=true,frame=leftline,framesep=5mm,framerule=1mm]
[ 39000 106000 57750 30500 252000]
\end{Verbatim}

Can you see why? ``Adding'' the two arrays together performed addition
\textit{element-by-element}. The result is a new array with $38000+1000$ as the
first element, $102000+4000$ as the second, \textit{etc.} This, too, is a
lightning-fast, vectorized operation, and it too works with all the other math
operators.

Just to re-emphasize one point before we go on. In the example back on
p.~\pageref{vectorizedArrayIntExample}, we assigned the result of the operation
to a new variable, \texttt{num\_likes\_tomorrow}. This means that
\texttt{num\_likes\_today} itself was \textit{unchanged} by the code. In
contrast, in the example we just did, we assigned the result of the operation
back into an \textit{existing} variable (\texttt{salaries}). So
\texttt{salaries} has itself been updated as a result of that code.

% TODO: could include memory picture to this effect.




\subsubsection{The answer}

Oh, and the answer to the puzzle on p.~\pageref{indexTest} -- and also the
answer to Life, the Universe, and Everything, as it turns out -- is 42.
