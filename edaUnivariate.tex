
\chapter{Exploratory Data Analysis: univariate data}
\label{ch:edaUnivariate}

\index{Exploratory Data Analysis (EDA)}

The fancy term ``\textbf{Exploratory Data Analysis}'' (EDA) basically just
means getting acquainted with your data. After importing a new data set into
Python, the first thing you normally do is poke around to get an idea of what
it contains. You may not even know what questions you eventually want to ask --
let alone what the answers are -- but sizing up the data is a necessary
precursor to those activities.

\index{univariate}

In this chapter, we'll learn some basic EDA techniques for \textbf{univariate
data}, which is really all we've studied so far. ``Univariate'' means to
consider just one variable at a time, rather than possible relationships
between variables. A single (one-dimensional) NumPy array or Pandas
\texttt{Series} is a univariate data set, if you treat it in isolation. As it
turns out, there's quite a few interesting things you can do with even
something that simple.

\section{Summary statistics}

\index{summary statistics}

\index{scales of measure}
\index{categorical variable}
\index{nominal variable}

\textbf{Summary statistics} are a way to capture the general features of a data
set, so you can see the forest instead of just a bunch of trees. Which type of
summary information is appropriate depends on whether you're dealing with
categorical or numeric data.

\subsection{Categorical data: counts of occurrences}

\index{faves@\texttt{faves}}
Let's say you had access to a poll on people's favorite pop stars. You import
this into a big ol' Pandas \texttt{Series} called \texttt{faves}:

\begin{Verbatim}[fontsize=\small,samepage=true,frame=single,framesep=3mm]
print(faves)
\end{Verbatim}

\begin{Verbatim}[fontsize=\small,samepage=true,frame=leftline,framesep=5mm,framerule=1mm]
0          Katy Perry
1             Rihanna
2       Justin Bieber
3               Drake
4             Rihanna
5        Taylor Swift
6               Adele
7               Adele
8        Taylor Swift
9       Justin Bieber
...
1395       Katy Perry
dtype: object
\end{Verbatim}

\index{value\_counts@\texttt{.value\_counts()}}

That's great, but it's also kinda TMI. You probably don't care who the
\textit{first} person's idol is, nor the fifteenth, nor the last. Much more
interesting is simply \textit{how many times} each value appears in the
\texttt{Series}. This information is available from the Pandas
\texttt{.value\_counts()} method:

\begin{Verbatim}[fontsize=\small,samepage=true,frame=single,framesep=3mm]
counts = faves.value_counts()
print(counts)
\end{Verbatim}

\begin{Verbatim}[fontsize=\small,samepage=true,frame=leftline,framesep=5mm,framerule=1mm]
Taylor Swift     388
Katy Perry       265
Drake            261
Adele            212
Rihanna          136
Justin Bieber    134
dtype: int64
\end{Verbatim}

The \texttt{.value\_counts()} method returns another \texttt{Series}, but the
\textit{values} of the original \texttt{Series} become the \textit{keys} of the
new one. This tells us at a glance how popular each answer is relative to the
others.

To get percentages instead of totals, just divide by the total and multiply by
100, of course:

\begin{Verbatim}[fontsize=\small,samepage=true,frame=single,framesep=3mm]
print(counts / len(counts) * 100)
\end{Verbatim}

\begin{Verbatim}[fontsize=\small,samepage=true,frame=leftline,framesep=5mm,framerule=1mm]
Taylor Swift     0.277937
Katy Perry       0.189828
Drake            0.186963
Adele            0.151862
Rihanna          0.097421
Justin Bieber    0.095989
dtype: float64
\end{Verbatim}

\index{mode}
\index{central tendency}
\index{measure of central tendency}

Recall (p.~\pageref{mode}) that the \textbf{mode} is the only measure of
central tendency that makes sense for categorical data. And all you have to do
is call \texttt{.value\_counts()} and look at the top result. (In this case,
\texttt{Taylor Swift}.)

Note that \texttt{.value\_counts()} is a Pandas \texttt{Series} method, not a
NumPy method. If you find yourself with a NumPy array instead, you can just
\textbf{wrap} it in a \texttt{Series} as we did in Section~\ref{wrap}
(p.~\pageref{wrap}):

\begin{Verbatim}[fontsize=\small,samepage=true,frame=single,framesep=3mm]
my_array = np.array(['red','blue','red','green','green','green','blue'])
print(pd.Series(my_array).value_counts())
\end{Verbatim}

\begin{Verbatim}[fontsize=\small,samepage=true,frame=leftline,framesep=5mm,framerule=1mm]
green    3
red      2
blue     2
dtype: int64
\end{Verbatim}

